% Options for packages loaded elsewhere
\PassOptionsToPackage{unicode}{hyperref}
\PassOptionsToPackage{hyphens}{url}
\PassOptionsToPackage{dvipsnames,svgnames,x11names}{xcolor}
%
\documentclass[
  letterpaper,
  DIV=11,
  numbers=noendperiod]{scrartcl}

\usepackage{amsmath,amssymb}
\usepackage{iftex}
\ifPDFTeX
  \usepackage[T1]{fontenc}
  \usepackage[utf8]{inputenc}
  \usepackage{textcomp} % provide euro and other symbols
\else % if luatex or xetex
  \usepackage{unicode-math}
  \defaultfontfeatures{Scale=MatchLowercase}
  \defaultfontfeatures[\rmfamily]{Ligatures=TeX,Scale=1}
\fi
\usepackage{lmodern}
\ifPDFTeX\else  
    % xetex/luatex font selection
\fi
% Use upquote if available, for straight quotes in verbatim environments
\IfFileExists{upquote.sty}{\usepackage{upquote}}{}
\IfFileExists{microtype.sty}{% use microtype if available
  \usepackage[]{microtype}
  \UseMicrotypeSet[protrusion]{basicmath} % disable protrusion for tt fonts
}{}
\makeatletter
\@ifundefined{KOMAClassName}{% if non-KOMA class
  \IfFileExists{parskip.sty}{%
    \usepackage{parskip}
  }{% else
    \setlength{\parindent}{0pt}
    \setlength{\parskip}{6pt plus 2pt minus 1pt}}
}{% if KOMA class
  \KOMAoptions{parskip=half}}
\makeatother
\usepackage{xcolor}
\setlength{\emergencystretch}{3em} % prevent overfull lines
\setcounter{secnumdepth}{-\maxdimen} % remove section numbering
% Make \paragraph and \subparagraph free-standing
\makeatletter
\ifx\paragraph\undefined\else
  \let\oldparagraph\paragraph
  \renewcommand{\paragraph}{
    \@ifstar
      \xxxParagraphStar
      \xxxParagraphNoStar
  }
  \newcommand{\xxxParagraphStar}[1]{\oldparagraph*{#1}\mbox{}}
  \newcommand{\xxxParagraphNoStar}[1]{\oldparagraph{#1}\mbox{}}
\fi
\ifx\subparagraph\undefined\else
  \let\oldsubparagraph\subparagraph
  \renewcommand{\subparagraph}{
    \@ifstar
      \xxxSubParagraphStar
      \xxxSubParagraphNoStar
  }
  \newcommand{\xxxSubParagraphStar}[1]{\oldsubparagraph*{#1}\mbox{}}
  \newcommand{\xxxSubParagraphNoStar}[1]{\oldsubparagraph{#1}\mbox{}}
\fi
\makeatother


\providecommand{\tightlist}{%
  \setlength{\itemsep}{0pt}\setlength{\parskip}{0pt}}\usepackage{longtable,booktabs,array}
\usepackage{calc} % for calculating minipage widths
% Correct order of tables after \paragraph or \subparagraph
\usepackage{etoolbox}
\makeatletter
\patchcmd\longtable{\par}{\if@noskipsec\mbox{}\fi\par}{}{}
\makeatother
% Allow footnotes in longtable head/foot
\IfFileExists{footnotehyper.sty}{\usepackage{footnotehyper}}{\usepackage{footnote}}
\makesavenoteenv{longtable}
\usepackage{graphicx}
\makeatletter
\newsavebox\pandoc@box
\newcommand*\pandocbounded[1]{% scales image to fit in text height/width
  \sbox\pandoc@box{#1}%
  \Gscale@div\@tempa{\textheight}{\dimexpr\ht\pandoc@box+\dp\pandoc@box\relax}%
  \Gscale@div\@tempb{\linewidth}{\wd\pandoc@box}%
  \ifdim\@tempb\p@<\@tempa\p@\let\@tempa\@tempb\fi% select the smaller of both
  \ifdim\@tempa\p@<\p@\scalebox{\@tempa}{\usebox\pandoc@box}%
  \else\usebox{\pandoc@box}%
  \fi%
}
% Set default figure placement to htbp
\def\fps@figure{htbp}
\makeatother
% definitions for citeproc citations
\NewDocumentCommand\citeproctext{}{}
\NewDocumentCommand\citeproc{mm}{%
  \begingroup\def\citeproctext{#2}\cite{#1}\endgroup}
\makeatletter
 % allow citations to break across lines
 \let\@cite@ofmt\@firstofone
 % avoid brackets around text for \cite:
 \def\@biblabel#1{}
 \def\@cite#1#2{{#1\if@tempswa , #2\fi}}
\makeatother
\newlength{\cslhangindent}
\setlength{\cslhangindent}{1.5em}
\newlength{\csllabelwidth}
\setlength{\csllabelwidth}{3em}
\newenvironment{CSLReferences}[2] % #1 hanging-indent, #2 entry-spacing
 {\begin{list}{}{%
  \setlength{\itemindent}{0pt}
  \setlength{\leftmargin}{0pt}
  \setlength{\parsep}{0pt}
  % turn on hanging indent if param 1 is 1
  \ifodd #1
   \setlength{\leftmargin}{\cslhangindent}
   \setlength{\itemindent}{-1\cslhangindent}
  \fi
  % set entry spacing
  \setlength{\itemsep}{#2\baselineskip}}}
 {\end{list}}
\usepackage{calc}
\newcommand{\CSLBlock}[1]{\hfill\break\parbox[t]{\linewidth}{\strut\ignorespaces#1\strut}}
\newcommand{\CSLLeftMargin}[1]{\parbox[t]{\csllabelwidth}{\strut#1\strut}}
\newcommand{\CSLRightInline}[1]{\parbox[t]{\linewidth - \csllabelwidth}{\strut#1\strut}}
\newcommand{\CSLIndent}[1]{\hspace{\cslhangindent}#1}

\KOMAoption{captions}{tableheading}
\makeatletter
\@ifpackageloaded{caption}{}{\usepackage{caption}}
\AtBeginDocument{%
\ifdefined\contentsname
  \renewcommand*\contentsname{Table of contents}
\else
  \newcommand\contentsname{Table of contents}
\fi
\ifdefined\listfigurename
  \renewcommand*\listfigurename{List of Figures}
\else
  \newcommand\listfigurename{List of Figures}
\fi
\ifdefined\listtablename
  \renewcommand*\listtablename{List of Tables}
\else
  \newcommand\listtablename{List of Tables}
\fi
\ifdefined\figurename
  \renewcommand*\figurename{Figure}
\else
  \newcommand\figurename{Figure}
\fi
\ifdefined\tablename
  \renewcommand*\tablename{Table}
\else
  \newcommand\tablename{Table}
\fi
}
\@ifpackageloaded{float}{}{\usepackage{float}}
\floatstyle{ruled}
\@ifundefined{c@chapter}{\newfloat{codelisting}{h}{lop}}{\newfloat{codelisting}{h}{lop}[chapter]}
\floatname{codelisting}{Listing}
\newcommand*\listoflistings{\listof{codelisting}{List of Listings}}
\makeatother
\makeatletter
\makeatother
\makeatletter
\@ifpackageloaded{caption}{}{\usepackage{caption}}
\@ifpackageloaded{subcaption}{}{\usepackage{subcaption}}
\makeatother

\usepackage{bookmark}

\IfFileExists{xurl.sty}{\usepackage{xurl}}{} % add URL line breaks if available
\urlstyle{same} % disable monospaced font for URLs
\hypersetup{
  pdftitle={On a New Family of Stirling-like triangular arrays and Bell-like numbers},
  colorlinks=true,
  linkcolor={blue},
  filecolor={Maroon},
  citecolor={Blue},
  urlcolor={Blue},
  pdfcreator={LaTeX via pandoc}}


\title{On a New Family of Stirling-like triangular arrays and Bell-like
numbers}
\author{Vencislav Popov}
\date{}

\begin{document}
\maketitle


\usepackage{amsmath}
\usepackage{logix}
\newcommand{\Stirling}[0]{\genfrac\{\}{0pt}{}}
\newcommand{\mStirling}[0]{\genfrac{\lfloor}{\rfloor}{0pt}{}}

\subsection{1. Introduction}\label{introduction}

Given a sequence \((a_n)_{n \geq 0} = (a_0, a_1, \ldots)\), the binomial
transform is the procedure that maps it to a new sequence
\((b_n)_{n \geq 0} = (b_0,b_1,...)\) as follows:

\[
b_n = \sum_{k=0}^{n} \binom{n}{k} a_k
\]

Repeated applications of the binomial transform on the resulting
sequence can be summarized with a single sum:

\[
b_n = \sum_{k=0}^{n} \binom{n}{k} m^{n-k} a_k,
\]

where \(m\) is an integer that represents the number of times the
binomial transform has been applied. Bernstein and Sloane {[}1{]}
studied a number of what they call ``Eigen sequences'' of various such
transformations - sequences \((a_n)\) which when transformed one or more
times shift by one or more places but are otherwise preserved. Such
sequences show a ``self-similarity'' under an iterated transform and
understanding why this self-similarity occurs often reveals new
properties or relations between different integer sequences.

Perhaps the most famous case of a binomial-transform invariant sequence
is that of the Bell numbers (\href{https://oeis.org/A000110}{A000110}:
\(1, 1, 2, 5, 15, 52, 203, \ldots\)), which shift one place to the left
after a single binomial transformation:

\begin{equation}\phantomsection\label{eq-bell-recurrence}{
B_{n+1} = \sum_{k=0}^{n} \binom{n}{k} B_k
}\end{equation}

The Bell numbers count the total number of partitions of an n-element
set and they are part of a rich combinatorial structure that involves
the Stirling numbers the of the second kind
(\href{https://oeis.org/A008277}{A008277}), Touchard polynomials
{[}2{]}, the exponential function and linear operators acting on it
{[}3{]}.

In this paper I study a new generalization of the Bell numbers defined
by the following property:

\begin{equation}\phantomsection\label{eq-mbell-recurrence}{
B^{(m)}_{n+m} = \sum_{k=0}^n \binom{n}{k} m^{n-k} B_k^{(m)}
}\end{equation}

These are sequences that shift to the left by \(m\) places after \(m\)
applications of the binomial transform. The case \(m=1\) corresponds to
the Bell numbers, and \(m=2\) corresponds to sequences
\href{https://oeis.org/A007472}{A007472},
\href{https://oeis.org/search?q=1,0,1,2,5,16&language=english&go=Search}{A351143}
and \href{https://oeis.org/A351028}{A351028}, which shift by 2 places
left after 2 binomial transformations (sequences for \(m>2\) are not
currently present in OEIS). Although the sequences for \(m=2\) are
listed in OEIS, little is known about their properties.

In the remainder of this paper I show that the m-fold
binomial--shift-invariance property that characterizes these sequences
arises from a combinatorial structure that mirrors that of the regular
Bell numbers. Each such \emph{m-Bell} sequence corresponds to the row
sums of new Stirling-like arrays, motivating the name \emph{m-Stirling
numbers} (which come in dual pairs). The m-Bell numbers have explicit
exponential generating functions (e.g.f.) that are the solutions of
ordinary differential equations of order \(m\). Each of the associated
m-Stirling arrays arises as the coefficients of polynomials that result
from the application of the exponential shift operator to hypergeometric
functions, which have simple forms for \(m=1\) (in terms of the
exponential function) and \(m=2\) (in terms of modified Bessel functions
of the first and second kinds). I conclude by providing a combinatorial
interpretation for the general case as well as showing a connection to
the Conway-Maxwell-Poisson distribution {[}4{]}.

To clearly establish the analogue between the Bell-Stirling-Touchard
framework and the novel results, I begin with a review of standard
results {[}5{]} {[}6{]}.

\subsection{2. Background - Bell numbers, Stirling numbers, Touchard
polynomials and operator
calculus}\label{background---bell-numbers-stirling-numbers-touchard-polynomials-and-operator-calculus}

Recall that the Bell numbers are the row sums of a triangle array formed
by the Stirling numbers of the second kind:

\[
\begin{align}
B_n = \sum_{k=0}^{n}\genfrac\{\}{0pt}{}{n}{k},
\end{align}
\]

where the Stirling numbers of the second kind satisfy the recurrence:

\[
\genfrac\{\}{0pt}{}{n+1}{k} = k \, \genfrac\{\}{0pt}{}{n}{k} + \genfrac\{\}{0pt}{}{n}{k-1}, \,\,\,\,\, (n, k ≥ 1)\\
\]

The Stirling numbers of the second kind count the number of ways to
partition n labeled objects into k unlabeled subsets and are the
coefficients of the Touchard (also known as Bell or exponential)
polynomials:

\[
T_n(x) = \sum_{k=0}^{n} \genfrac\{\}{0pt}{}{n}{k} x^k,
\]

whose bi-variate e.g.f. is below (and is the e.g.f. of the Bell numbers
when evaluated at \(x=1\)):

\[
\sum_{n=0}^{\infty} T_n(x) \frac{t^n}{n!} = e^{x(e^t-1)}
\]

Many useful identities involving Touchard polynomials and Stirling
numbers can be shown via the action of the exponential scaling operator:

\[
e^{txD_x}f(x)=f(xe^t)
\]

where \(D_x\) is the derivative operator with respect to \(x\).
Specifically, by applying this operator to the standard exponential
function \(e^x\) we get precisely the e.g.f. for Touchard polynomials
{[}3{]}:

\[
e^{-x}e^{txD_x}e^x = e^{x(e^t-1)}
\]

A final useful relation is the Dobinski formula that let us express
\(T_n(x)\) and the Bell numbers \(B_n = T_n(1)\) as an infinite sum:

\[
T_n(x)=e^{-x}\sum_{k=0}^{\infty}\frac{x^k k^n}{k!}
\]

\subsection{3. The case m = 2}\label{the-case-m-2}

The following three sequences all share the property that they shift by
2 places left after two binomial transformations:

\begin{itemize}
\item
  \href{https://oeis.org/A007472}{A007472}: \$1, 1, 1, 3, 9, 29, 105,
  431, 1969, \ldots \$
\item
  \href{https://oeis.org/A351143}{A351143}:
  \(1, 0, 1, 2, 5, 16, 61, 258, 1177, \ldots\)
\item
  \href{https://oeis.org/A351028}{A351028}: \$0, 1, 0, 1, 4, 13, 44,
  173, 792, 4009, \ldots \$
\end{itemize}

Define the following two-term recurrence relation where
\(\lfloor k/m\rfloor\) is the floor function:

\[
\genfrac{\lfloor}{\rfloor}{0pt}{}{n+1}{k}_m = m\lfloor k/m \rfloor \, \genfrac{\lfloor}{\rfloor}{0pt}{}{n}{k}_m + \genfrac{\lfloor}{\rfloor}{0pt}{}{n}{k-1}_m, \,\,\,\,\, n, k ≥ 1
\]

with initial conditions
\(\genfrac{\lfloor}{\rfloor}{0pt}{}{n}{k}_m = \delta_{n,k}\)

For \(m=1\) we obtain the standard Stirling numbers of the first kind.
Let us first consider the case for \(m=2\) in detail:

\[
\genfrac{\lfloor}{\rfloor}{0pt}{}{n+1}{k}_2 = 2\lfloor k/2 \rfloor \, \genfrac{\lfloor}{\rfloor}{0pt}{}{n}{k}_2 + \genfrac{\lfloor}{\rfloor}{0pt}{}{n}{k-1}_2, \,\,\,\,\, (n, k ≥ 1)\\
\]

This recurrence relation is almost identical to that for the Stirling
numbers of the second kind, except that the coefficient in front of
Z(n,k) has parity - it is always even, with k rounded down to the
nearest even integer.

The first few rows of this array are:

\begin{longtable}[]{@{}
  >{\raggedright\arraybackslash}p{(\linewidth - 18\tabcolsep) * \real{0.0794}}
  >{\raggedright\arraybackslash}p{(\linewidth - 18\tabcolsep) * \real{0.0794}}
  >{\raggedright\arraybackslash}p{(\linewidth - 18\tabcolsep) * \real{0.0794}}
  >{\raggedright\arraybackslash}p{(\linewidth - 18\tabcolsep) * \real{0.0794}}
  >{\raggedright\arraybackslash}p{(\linewidth - 18\tabcolsep) * \real{0.0794}}
  >{\raggedright\arraybackslash}p{(\linewidth - 18\tabcolsep) * \real{0.0794}}
  >{\raggedright\arraybackslash}p{(\linewidth - 18\tabcolsep) * \real{0.0794}}
  >{\raggedright\arraybackslash}p{(\linewidth - 18\tabcolsep) * \real{0.0794}}
  >{\raggedright\arraybackslash}p{(\linewidth - 18\tabcolsep) * \real{0.0794}}
  >{\raggedright\arraybackslash}p{(\linewidth - 18\tabcolsep) * \real{0.2857}}@{}}
\toprule\noalign{}
\begin{minipage}[b]{\linewidth}\raggedright
\end{minipage} & \begin{minipage}[b]{\linewidth}\raggedright
k=0
\end{minipage} & \begin{minipage}[b]{\linewidth}\raggedright
k=1
\end{minipage} & \begin{minipage}[b]{\linewidth}\raggedright
k=2
\end{minipage} & \begin{minipage}[b]{\linewidth}\raggedright
k=3
\end{minipage} & \begin{minipage}[b]{\linewidth}\raggedright
k=4
\end{minipage} & \begin{minipage}[b]{\linewidth}\raggedright
k=5
\end{minipage} & \begin{minipage}[b]{\linewidth}\raggedright
k=6
\end{minipage} & \begin{minipage}[b]{\linewidth}\raggedright
k=7
\end{minipage} & \begin{minipage}[b]{\linewidth}\raggedright
\(\sum_{k=0}^{n}\)
\end{minipage} \\
\midrule\noalign{}
\endhead
\bottomrule\noalign{}
\endlastfoot
n=0 & 1 & & & & & & & & 1 \\
n=1 & 0 & 1 & & & & & & & 1 \\
n=2 & 0 & 0 & 1 & & & & & & 1 \\
n=3 & 0 & 0 & 2 & 1 & & & & & 3 \\
n=4 & 0 & 0 & 4 & 4 & 1 & & & & 9 \\
n=5 & 0 & 0 & 8 & 12 & 8 & 1 & & & 29 \\
n=6 & 0 & 0 & 16 & 32 & 44 & 12 & 1 & & 105 \\
n=7 & 0 & 0 & 32 & 80 & 208 & 92 & 18 & 1 & 431 \\
\end{longtable}

The row sums of this array yield the sequence
\href{https://oeis.org/A007472}{A007472}

\[
z_n = \sum_{k=0}^{n}\genfrac{\lfloor}{\rfloor}{0pt}{}{n}{k}_2 = (1, 1, 1, 3, 9, 29, 105, 431, 1969, 9785, 52145, 296155, 1787385, ...)
\]

which as stated earlier has a property analogous to the Bell numbers: it
shifts by 2 places left after 2 applications of the binomial transform:

\[
\mathscr{z}_{n+2}=\sum_{k=0}^{n}\binom{n}{k}2^{n-k}z_k
\]

where we use the following identity that relates a double binomial
transform to a single transform with an extra factor of \(2^{n-k}\):

\[
\sum_{k=0}^{n}\binom{n}{k}\sum_{j=0}^{k}\binom{k}{j}a_j = \sum_{k=0}^{n}\binom{n}{k}2^{n-k}a_k
\]

\phantomsection\label{refs}
\begin{CSLReferences}{0}{1}
\bibitem[\citeproctext]{ref-bernstein1995}
\CSLLeftMargin{{[}1{]} }%
\CSLRightInline{Bernstein, M and Sloane, N J A (1995 ). Some canonical
sequences of integers. \emph{Linear Algebra and its Applications}.
\textbf{226-228} 57--72.
\url{https://linkinghub.elsevier.com/retrieve/pii/0024379594002459}}

\bibitem[\citeproctext]{ref-weisstein}
\CSLLeftMargin{{[}2{]} }%
\CSLRightInline{Weisstein, E W Bell Polynomial.
\url{https://mathworld.wolfram.com/BellPolynomial.html}}

\bibitem[\citeproctext]{ref-dattoliTouchardPolynomialsGeneralized2010}
\CSLLeftMargin{{[}3{]} }%
\CSLRightInline{Dattoli, G, Germano, B, Martinelli, M R and Ricci, P E
(2010 ). Touchard like polynomials and generalized {Stirling} numbers.
\url{http://arxiv.org/abs/1010.5934}}

\bibitem[\citeproctext]{ref-shmueliUsefulDistributionFitting2005}
\CSLLeftMargin{{[}4{]} }%
\CSLRightInline{Shmueli, G, Minka, T P, Kadane, J B, Borle, S and
Boatwright, P (2005 ). A {Useful Distribution} for {Fitting Discrete
Data}: {Revival} of the {Conway}--{Maxwell}--{Poisson Distribution}.
\emph{Journal of the Royal Statistical Society Series C: Applied
Statistics}. \textbf{54} 127--42.
\url{https://doi.org/10.1111/j.1467-9876.2005.00474.x}}

\bibitem[\citeproctext]{ref-comtet1974}
\CSLLeftMargin{{[}5{]} }%
\CSLRightInline{Comtet, L (1974 ). \emph{Advanced Combinatorics}.
Springer Netherlands, Dordrecht.
\url{http://link.springer.com/10.1007/978-94-010-2196-8}}

\bibitem[\citeproctext]{ref-suxe1ndor2004}
\CSLLeftMargin{{[}6{]} }%
\CSLRightInline{Sándor, J and Crstici, B (2004 ). Stirling, bell,
bernoulli, euler and eulerian numbers. Springer Netherlands, Dordrecht.
459--618. \url{https://doi.org/10.1007/1-4020-2547-5_5}}

\end{CSLReferences}




\end{document}
