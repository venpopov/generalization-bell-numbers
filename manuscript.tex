\documentclass[a4paper]{amsart}
\usepackage{biblatex}
\usepackage{amsaddr}
\usepackage{hyperref}
\usepackage{amsmath} % Needed for align, aligned, equation*
\usepackage{mathrsfs} % Needed for \mathscr
\usepackage{array} % Needed for table column specifiers

% Custom commands
\newcommand{\Stirling}[0]{\genfrac\{\}{0pt}{}}
\newcommand{\mStirling}[0]{\genfrac{\lfloor}{\rfloor}{0pt}{}}
\newcommand{\floor}[1]{\left\lfloor #1 \right\rfloor}
\newcommand{\ceil}[1]{\left\lceil #1 \right\rceil}
\newcommand{\R}{\mathbf{R}} % The real numbers.
%\DeclareMathOperator{\dist}{dist} % The distance.

% Formatting
\RequirePackage[left=1.5in,right=1.5in,top=1in,bottom=1in]{geometry}
%\setlength{\parindent}{0mm}
\newtheorem{theorem}{Theorem}
\newtheorem{definition}[theorem]{Definition}
\newtheorem{proposition}[theorem]{Proposition}
\newtheorem{conjecture}[theorem]{Conjecture}
\newtheorem{lemma}[theorem]{Lemma}
\newtheorem{corollary}[theorem]{Corollary}
\newtheorem{remark}[theorem]{Remark}


%% Meta
\title{On a New Family of Bell-like numbers and Stirling-like triangular arrays}
\author{Vencislav Popov}
\email{vencislav.popov@gmail.com}
\address{Department of Psychology, University of Zurich}
\date{}

% Biblio
\addbibresource{references.bib}
% \bibliography{references} % Removed as addbibresource is used with biblatex

\begin{document}
\begin{abstract}
I introduce $m$-Bell numbers $B^{(m)}_{n}$, a generalization of the Bell numbers ($m=1$). They are characterized by the operator identity $\mathrm{BINOM}^{m}\!\circ a=\mathrm{L}^{m}\!\circ a$ for a sequence $a=(a_{n})_{n\ge0}$, i.e.\ invariance under $m$ binomial transforms up to an $m$-place left shift.  Their exponential generating functions (EGFs) satisfy $m$-th‑order ordinary differential equations whose solutions are hypergeometric functions; for $m=2$ these specialize to modified Bessel functions, yielding the “Bessel‑Bell’’ numbers. Analogous to the Bell–Stirling correspondence, there exist \(m\)-Stirling triangular arrays arising from the two‑term recurrence  $S_{n+1,k}=m\!\lfloor k/m\rfloor\,S_{n,k}+S_{n,k-1},$ whose row sums reproduce $B^{(m)}_{n}$. These numbers admit combinatorial interpretations through parity‑constrained partitions and are linked to moments of the Conway–Maxwell–Poisson distribution.

\bigskip
\noindent\textbf{Keywords.} Bell numbers, Stirling numbers, Conway–Maxwell–Poisson distribution, modified Bessel functions
\end{abstract}

\maketitle

\section{Introduction}\label{sec-introduction}
\noindent Given a sequence of numbers $(a_n)_{n \geq 0} = (a_0, a_1, \ldots)$, the binomial transform maps it to a new sequence $(b_n)_{n \geq 0} = (b_0,b_1,...)$ as follows:
\begin{equation*}
    b_n = \sum_{k=0}^{n} \binom{n}{k} a_k.
\end{equation*}

Repeated applications of the binomial transform on the resulting sequence can be represented with a single sum (see \cite{spiveyKbinomialTransformsHankel2006}):
\begin{equation*}
    b_n = \sum_{k_1=0}^{n} \binom{n}{k_1}\sum_{k_2=0}^{n} \binom{k_1}{k_2} \dots \sum_{k_{m}=0}^{k_{m-1}} \binom{k_{m-1}}{k_m} a_{k_m} = \sum_{k=0}^{n} \binom{n}{k} m^{n-k} a_k,
\end{equation*}
where $m$ is an integer that represents the number of times the binomial transform has been applied. Bernstein and Sloane \cite{bernstein1995} studied a number of what they call "Eigen sequences" of various such transformations - sequences $(a_n)$ which when transformed one or more times shift by one or more places but are otherwise preserved. Such sequences show a "self-similarity" under an iterated transform and understanding why this self-similarity occurs often reveals new properties or relations between different integer sequences and the combinatorial structures they enumerate.

Perhaps the most famous case of a binomial-transform invariant sequence is that of the Bell numbers, entry \href{https://oeis.org/A000110}{A000110} in the Online Encyclopedia of Integer Sequences (OEIS): $1, 1, 2, 5, 15, 52, 203, 877, 4140, \ldots$ Applying the binomial transformation to the Bell numbers produces an identical sequence, with the first element omitted and each subsequent element's index shifted once to the left:
\begin{equation}\label{eq-bell-recurrence}
B_{n+1} = \sum_{k=0}^{n} \binom{n}{k} B_k
\end{equation}

The Bell numbers count the total number of partitions of an n-element set and they are part of a rich combinatorial structure that involves the Stirling numbers the of the second kind (\href{https://oeis.org/A008277}{A008277}), Touchard polynomials \cite{weisstein}, the exponential function and linear operators acting on it \cite{dattoliTouchardPolynomialsGeneralized2010}.

To simplify the notation, following Bernstein and Sloane \cite{bernstein1995}, define the following operators on sequences:
\[
\begin{aligned}
\mathrm{BINOM} \circ [a_0,a_1,a_2, \dots] &= \bigg[\binom{0}{0}a_0,\binom{1}{0}a_0+\binom{1}{1}a_1, \ldots, \sum_{k=0}^n\binom{n}{k}a_k, \ldots \bigg] \\
L \circ [a_0,a_1,a_2, \ldots] &= [a_1,a_2, \ldots]
\end{aligned}
\]
where $\mathrm{BINOM}$ is the binomial transform operator applied to sequence $a$, whereas $L$ is the left-shift operator. Then the Bell numbers form the unique sequence that satisfies the following equality with initial condition $a_0 = 1:$
\[
\mathrm{BINOM} \circ a = \mathrm{L} \circ a
\]

Consider a new generalization of the Bell numbers defined by the following property:
\begin{equation}\label{eq-mbell-recurrence}
B^{(m)}_{n+m} = \sum_{k=0}^n \binom{n}{k} m^{n-k} B_k^{(m)}
\end{equation}
where the upper index $(m)$ has no algebraic meaning and should be read merely as an "m-Bell" number. Equivalently, these sequences satisfy the operator equation:
\begin{equation}\label{eq-mbell-operator}
\mathrm{BINOM^m} \circ a = \mathrm{L}^m \circ a
\end{equation}

These are sequences that shift to the left by $m$ places after $m$ applications of the binomial transform. The case $m=1$ corresponds to the Bell numbers, and $m=2$ corresponds to sequences \href{https://oeis.org/A007472}{A007472}, \href{https://oeis.org/search?q=1,0,1,2,5,16&language=english&go=Search}{A351143} and \href{https://oeis.org/A351028}{A351028}, which shift by 2 places left after 2 binomial transformations (sequences for $m>2$ are not currently present in OEIS). Although the sequences for $m=2$ are listed in OEIS, little is known about their properties.

In this paper I show that the m-fold binomial–shift-invariance property that characterizes these sequences arises from a combinatorial structure that mirrors that of the regular Bell numbers (Section \ref{sec-background}). The $m$-Bell numbers have exponential generating functions (e.g.f.) that are the solutions of ordinary differential equations of order $m$. The solutions to these equations are hypergeometric functions \cite{tauberGeneralizationsExponentialFunction1960}, which reduce to the exponential function for $m=1$ (the 1-Bell numbers) and modified Bessel functions of the first and second kinds for $m=2$ (motivating the name Bessel-Bell or 2-Bell numbers; Section \ref{sec-m2}; Theorem \ref{thm-m2-egf}). These generalized Bell numbers admit Dobiński-like formula representations (Theorem \ref{thm-bessel-dobiski}). Each such \textit{m-Bell} sequence corresponds to the row sums of new Stirling-like arrays (Section \ref{sec-2-stirling}), motivating the name \textit{m-Stirling }numbers, which, as we will see, come in dual pairs just like the regular Stirling numbers of the first and second kind. Each of the associated $m$-Stirling arrays arises as the coefficients of polynomials that result from the application of the exponential shift operator to the hypergeometric e.g.f's. The $m$-Stirling and $m$-Bell numbers count parity-constrained partitions, resulting in a novel general combinatorial structure that greatly expands the Bell-Stirling-Touchard story (Section \ref{sec-combinatorial}). Finally, just like their standard counterparts are related to moments of the Poisson distribution, the $m$-Bell numbers turn out to be related to the moments of a generalized Poisson distribution known as the Conway-Maxwell-Poisson distribution \cite{shmueliUsefulDistributionFitting2005} (Section \ref{sec-cmp}).

To clearly ground the analogue between the Bell-Stirling-Touchard framework and the novel results, I begin with a review of standard results and notation \cite{comtet1974, sándor2004}.

\section{Background}\label{sec-background}
\noindent The exponential generating function (e.g.f.) $\mathcal{A}(x)$ of a sequence $(a_n)_{n \geq 0}$ is a formal power series in $x$:
\[
\mathcal{A}(x) = \sum_{n=0}^\infty a_n \frac{x^n}{n!}
\]

\begin{proposition}[Functional equation for the e.g.f. of $m$-Bell numbers]\label{prop:mbell-egf}
Let $\mathcal{A}(x)$ be the exponential generating function of a sequence \( a = (a_n) \) satisfying \eqref{eq-mbell-operator}.
Then \( \mathcal{A}(x) \) satisfies the $m$-th order ODE:
\[
\left(\frac{\mathrm{d}}{\mathrm{d}x}\right)^m \mathcal{A}(x) - e^{mx} \mathcal{A}(x) = 0.
\]
\end{proposition}

\begin{proof}
It is well known that both the binomial transform and the left-shift operator acting on $a$ have simple effects on the e.g.f., $\mathcal{B}(x)$, of the resulting sequence $b$ (e.g. \cite{bernstein1995}):
\begin{equation}\label{eq-egf-transform}
\begin{aligned}
b = \mathrm{BINOM} \circ a \quad &\iff\quad \mathcal{B}(x) = e^x \mathcal{A}(x) \\
b = \mathrm{L} \circ a \quad &\iff\quad \mathcal{B}(x) = \mathcal{A}'(x)
\end{aligned}
\end{equation}
Applying $\mathrm{BINOM}$ $m$ times corresponds to multiplying $\mathcal{A}(x)$ by $e^{mx}$. Applying $\mathrm{L}$ $m$ times corresponds to taking the $m$-th derivative. Therefore, if \(\mathrm{BINOM}^m \circ a = \mathrm{L}^m \circ a\), then:
\[
\mathcal{A}^{(m)}(x) = e^{mx} \mathcal{A}(x).
\]
Rewriting this gives the result.
\end{proof}

The solution of the ODE for $m=1$ gives the well known offset double exponential generating function for the Bell numbers (with $x = e^t$):
\begin{equation}\label{eq-bell-egf}
    e^{e^t-1} = \sum_{n=0}^{\infty} B_n \frac{t^n}{n!}.
\end{equation}

Recall that the Bell numbers are the row sums of a triangle array formed by the Stirling numbers of the second kind:
\begin{align*}
B_n = \sum_{k=0}^{n}\genfrac\{\}{0pt}{}{n}{k},
\end{align*}
where the Stirling numbers of the second kind satisfy the two-term recurrence:
\[
\genfrac\{\}{0pt}{}{n+1}{k} = k \, \genfrac\{\}{0pt}{}{n}{k} + \genfrac\{\}{0pt}{}{n}{k-1}, \,\,\,\,\, n, k \ge 1\\.
\]

The Stirling numbers of the second kind count the number of ways to partition n labeled objects into k unlabeled subsets and are the coefficients of the Touchard (also known as Bell or exponential) polynomials:
\begin{equation}\label{eq-touchard-def}
    T_n(x) = \sum_{k=0}^{n} \genfrac\{\}{0pt}{}{n}{k} x^k,
\end{equation}

whose e.g.f. is a bi-variate generalization of the Bell numbers' e.g.f.:
\begin{equation}\label{eq-touchard-egf}
    \sum_{n=0}^{\infty} T_n(x) \frac{t^n}{n!} = e^{x(e^t-1)}.
\end{equation}
Many useful identities involving Touchard polynomials and Stirling numbers can be shown via the action of the exponential scaling operator:
\begin{equation}\label{eq-scaling-operator}
    e^{txD_x}f(x)=f(xe^t)
\end{equation}

where $D_x$ is the derivative operator with respect to $x$. Specifically, by applying this operator to the standard exponential function $e^x$ we get precisely the e.g.f. for Touchard polynomials \cite{dattoliTouchardPolynomialsGeneralized2010}:
\[
e^{-x}e^{txD_x}e^x = e^{x(e^t-1)}
\]

A final useful relation is the celebrated Dobiński formula\cite{wilfGeneratingfunctionologyThirdEdition2005} that lets us express $T_n(x)$ and the Bell numbers $B_n = T_n(1)$ as an infinite sum:
\begin{equation}\label{eq-bell-dobinski}
\begin{aligned}
T_n(x) & =e^{-x}\sum_{k=0}^{\infty}\frac{x^k k^n}{k!} \\
B_n & = \frac{1}{e}\sum_{k=0}^{\infty}\frac{k^n}{k!}
\end{aligned}
\end{equation}

In the remainder of the paper we will see that the sequences for $m$-Bell numbers have many analogous properties as those defined above, starting with the special case $m = 2$.

\section{The Bessel-Bell numbers ($m=2$)}\label{sec-m2}
\subsection{Deriving the e.g.f.}
\noindent Before solving the general case, let us focus on the $m=2$ generalization of the Bell numbers, which satisfy the property that they shift by 2 places left after two binomial transformations:
\begin{equation}\label{eq-2bell-recurrence}
B^{(2)}_{n+2} = \sum_{k=0}^n \binom{n}{k} 2^{n-k} B_k^{(2)}
\end{equation}
The following three sequences listed in OEIS satisfy this property:
\begin{itemize}
    \item \href{https://oeis.org/A007472}{A007472}: $1, 1, 1, 3, 9, 29, 105, 431, 1969, \ldots$
    \item \href{https://oeis.org/A351143}{A351143}: $1, 0, 1, 2, 5, 16, 61, 258, 1177, \ldots$
    \item \href{https://oeis.org/A351028}{A351028}: $0, 1, 0, 1, 4, 13, 44, 173, 792, 4009, \ldots$
\end{itemize}

The main difference between the three sequences are the initial conditions $(a_0, a_1)$: $(1, 1); (1,0); (0,1)$. We will see in Section \ref{sec-general-case} that for any $m$ there are $m$ \textbf{primitive} solutions, whose initial conditions are a sequence of $m$ values, exactly one of which is 1 and the rest are 0. In the $m=2$ case the primitive solutions are A351143 and A351028, whose element-wise sum produces the sequence A007472. Owing to the form of their e.g.f.s, I will refer to the composite sequence A007472 as the Bessel-Bell numbers (not to be confused with Bessel numbers \cite{cheonGeneralizedBesselNumbers2013a}).

\begin{theorem}\label{thm-m2-egf}
The exponential generating functions for 2-Bell numbers are solutions to the modified Bessel ODE - a linear combination of modified Bessel functions of the first ($I_0$) and second kind ($K_0$) of order 0, whose weights are uniquely determined by the first two elements of the corresponding sequence. Specifically:
\[
\mathcal{A}(x) = p\,I_0(e^x)+q\,K_0(e^x)
\]

\begin{center}
\begin{tabular}{lrr}
\hline
Sequence & Coefficient \(p\) & Coefficient \(q\) \\
\hline
351143 & \(K_1(1) \approx 0.601907\) & \(I_1(1) \approx 0.565159\) \\
351008 & \(K_0(1) \approx 0.421024\) & \(-I_0(1) \approx 1.266066\) \\
007472 & \(K_0(1) + K_1(1)\approx 1.022932\) & \(I_1(1) - I_0(1)\approx -0.700907\) \\
\hline
\end{tabular}
\end{center}
\smallskip
\end{theorem}

\begin{proof}
By Proposition \ref{prop:mbell-egf}, the e.g.f. $\mathcal{A}(x)$ satisfies the linear ODE
\[
\mathcal{A}''(x)-e^{2x}\,\mathcal{A}(x)=0.
\]
Set $t=e^{x}$ (so $\mathrm{d}t/\mathrm{d}x=t$) and define $f(t)=\mathcal{A}(x)$.  By the chain rule,
\[
\begin{aligned}
\mathcal{A}'(x) &= t\,f'(t),\\
\mathcal{A}''(x)&= t\,f'(t)+t^{2}f''(t),\\
e^{2x}\,\mathcal{A}(x) &= t^{2}f(t).
\end{aligned}
\]
Substituting in yields the standard form of the modified Bessel equation for the case of $n=0$ (see \href{https://dlmf.nist.gov/10.25.E1}{10.25.1}) \cite{NIST:DLMF}
\begin{equation}\label{eq-m2-ode}
t^{2}f''+t\,f'-t^{2}f=0,
\end{equation}
whose general solution is:
\[
f(t)=p\,I_0(t)+q\,K_0(t).
\]
with $I_n$ and $K_n$ being the modified Bessel functions of the first and second kind, and $p$ and $q$ are constants determined by the initial conditions. Therefore, we obtain the general solution for $\mathcal{A}(x)$, the e.g.f. of sequences \href{https://oeis.org/A007472}{A007472}, \href{https://oeis.org/A351143}{A351143} and \href{https://oeis.org/A351028}{A351028}, by substituting back $t=e^x$:
\begin{equation}\label{eq-m2-ODE-solution}
\mathcal{A}(x)=p\, I_o(e^x) + q\,K_0(e^x)
\end{equation}
To determine $p$ and $q$ we use the known initial conditions. Sequence A007472 is the element-wise sum of A351143 and A351028, so it is sufficient to determine $p$ and $q$ for the latter 2 sequences only. For A351143 we have $a_0 = 1$ and $a_1 = 0$. Evaluating the Maclaurin series for $\mathcal{A}(x)$ yields the following system of equations:
\[
\begin{aligned}
\mathcal{A}(0) & = p\, I_0(1) + q\,K_0(1) = 1 \\
\mathcal{A}'(x)\bigr|_{x=0} &=p\, I_0'(e^x)\bigr|_{x=0} + q\,K_0'(e^x)\bigr|_{x=0} = 0
\end{aligned}
\]
where $f'(x)\bigr|_{x=0}$ is the first derivative of $f$ evaluated at 0. The derivative of $I_0$ and $K_0$ are thankfully straightforward (\href{https://dlmf.nist.gov/10.29.E3}{10.29.3}) \cite{NIST:DLMF}:
\begin{equation}\label{eq-bessel-deriv1}
\begin{aligned}
I_0'(x) &= I_1(x) \\ 
K_0'(x) &= -K_1(x). \\
\end{aligned}
\end{equation}
By the chain rule we get:
\begin{equation*}
    \begin{aligned}
        I_0'\bigl(e^x\bigr) &= e^xI_1\bigl(e^x\bigr) \\
        K_0'\bigl(e^x\bigr) &= -e^xK_1\bigl(e^x\bigr) \\
    \end{aligned}
\end{equation*}
Evaluating at $x=0$ gives:
\[
\begin{aligned}
A(0) & = p \,I_0(1)+q\,K_0(1) = 1 \\
A'(0) & = p\,I_1(1)-q K_1(1) = 0
\end{aligned}
\]
After some standard algebraic torture we get the following expressions for p and q:
\[
\begin{aligned}
p &= \frac{1-qK_0(1)}{I_0(1)} \\
q &= \frac{I_1(1)}{I_1(1)K_0(1)+I_0(1)K_1(1)}
\end{aligned}
\]
These expressions can be simplified further due to the following Bessel identity concerning the Wronskian of the modified Bessel functions (see \href{https://dlmf.nist.gov/10.28.E2}{10.28.2}):
\begin{equation}\label{eq-bessel-wronskian-general}
    I_v(z)K_{v+1}(z)+I_{v+1}(z)K_v(z) = 1/z
\end{equation}

which holds for any complex $v$ and $z$. In the special case when $v = 0$ and $z=1$:
\begin{equation}\label{eq-bessel-wronskian}
I_0(1)K_1(1)+I_1(1)K_0(1) = 1
\end{equation}
Therefore the constants for the e.g.f. of sequence A351143 are:
\[
\begin{aligned}
q &= I_1(1) \\
p &= \frac{1 - I_1(1)K_0(1)}{I_0(1)} = \frac{I_0(1)K_1(1)+I_1(1)K_0(1) - I_1(1)K_0(1)}{I_0(1)} = K_1(1)
\end{aligned}
\]
With a very similar manipulation for the initial conditions of A351028 and A007472 we get the coefficients in Theorem~\ref{thm-m2-egf} which concludes the proof.
\end{proof}

\subsection{Why do the e.g.f.s of the 2-Bell numbers produce integer coefficients?}\label{2-bell-integer-coefs}
The e.g.f.s of the 2-Bell numbers are admittedly more complicated than the e.g.f. $e^{e^x-1}$ of the Bell numbers \eqref{eq-bell-egf} (although the form is quite similar, as I discuss at the end of this section). How is it that such relatively complicated expression, involving four different special functions, produce integer coefficients? Let us explore the series expansion for the derived e.g.f.s. Consider first the case of sequence A351143:
\[
\mathcal{A}_{351143}(x) = K_1(1)I_0(e^x) + I_1(1)K_0(e^x)
\]
We have the following standard series for $I_0$ and $K_0$ (see \href{https://dlmf.nist.gov/10.25.E2}{10.25.2} and \href{https://dlmf.nist.gov/10.31.E2}{10.31.2}) \cite{NIST:DLMF}:
\begin{equation}\label{eq-i-k-0-series}
    \begin{aligned}
        I_0(z) &= \sum_{k=0}^{\infty} \frac{(\frac{1}{2}z)^{2k}}{k!k!} \\
        K_0(z) &= -\log\biggl(\frac{z}{2}\biggr)I_0(z) + \sum_{k=0}^{\infty}\frac{\psi(k+1)(\frac{1}{2}z)^{2k}}{k!k!}
    \end{aligned}
\end{equation}
where $\psi$ is the digamma function. Let us focus on the $I_0$ function. Substituting $z=e^x$ and then expanding the Taylor series for the exponential function we get:
\begin{equation}\label{eq-besseli-exp-maclaurin}
\begin{aligned}
I_0(e^x) &= \sum_{k=0}^{\infty} \frac{e^{2xk}}{2^{2k}k!k!} = \sum_{k=0}^{\infty} \frac{1}{2^{2k}k!k!}\sum_{n=0}^{\infty}\frac{x^n(2k)^n}{n!} = \sum_{n=0}^{\infty}\frac{x^n}{n!}\sum_{k=0}^{\infty} \frac{(2k)^n}{2^{2k}k!k!}
\end{aligned}
\end{equation}
Let the inner sum be represented by $S(n) = \sum_{k=0}^{\infty} \frac{(2k)^n}{2^{2k}k!k!}$. This formula is reminiscent of the Dobiński formula for the standard Bell numbers \eqref{eq-bell-dobinski} with an extra factorial in the denominator and extra powers of 2. Indeed, we can state an equivalent theorem:

\begin{theorem}[Dobiński-like formula for the Bessel-Bell numbers]\label{thm-bessel-dobiski}
\[
\begin{aligned}
I_0(e^x) &= \sum_{n=0}^\infty S(n) \frac{x^n}{n!} \\
S(n) &= \sum_{k=0}^{\infty} \frac{(2k)^n}{(2^{k}k!)^2}=v_n I_0(1) + u_n I_1(1) \\
A351143(n) &= v_n \\
A351028(n) &= u_n \\
A007472(n) &= v_n+u_n
\end{aligned}
\]
\end{theorem}

\begin{proof}
Proceed by induction. First, establish the base cases:
\[
\begin{aligned}
S(0) &= \sum_{k=0}^{\infty} \frac{1}{2^{2k}k!k!} = I_0(1) \\
S(1) &= \sum_{k=0}^{\infty} \frac{2k}{2^{2k}k!k!} = \sum_{k=1}^{\infty} \frac{1}{2^{2k-1}(k-1)!k!} = \sum_{k=0}^{\infty} \frac{1}{2^{2k+1}k!(k+1)!} = I_1(1)\\
S(2) &= \sum_{k=0}^{\infty} \frac{2^2k^2}{2^{2k}k!k!} = \sum_{k=1}^{\infty} \frac{1}{2^{2k-2}(k-1)!(k-1)!} = \sum_{k=0}^{\infty} \frac{1}{2^{2k}k!k!} = I_0(1)\\
\end{aligned}
\]
where in the case $S(1)$ we used the expansion for a general order modified Bessel function:
\[
I_v = \sum_{k=0}^\infty \frac{(x/2)^{2v+1}}{\Gamma(k+1)\Gamma(k+v+1)}
\]
Induction hypothesis: for every $m \leq n$, it holds that $S(m) = v_m I_0(1) + u_m I_1(1)$. Then:
\begin{equation}\label{eq-2bell-dobinski-proof-step1}
\begin{aligned}
S(n+2) &= \sum_{k=0}^{\infty} \frac{(2k)^{n+2}}{2^{2k}k!k!} = \sum_{k=1}^{\infty} \frac{(2k)^{n}}{2^{2k-2}(k-1)!(k-1)!} = \sum_{k=0}^{\infty} \frac{(2k+2)^{n}}{2^{2k}k!k!}\\
&= \sum_{k=0}^{\infty}\sum_{m=0}^{n} \binom{n}{m}2^{n-m}\frac{(2k)^m}{2^{2k}k!k!} = \sum_{m=0}^{n}\binom{n}{m}2^{n-m} \sum_{k=0}^{\infty}\frac{(2k)^m}{2^{2k}k!k!} \\
&= \sum_{m=0}^{n}\binom{n}{m}2^{n-m} S(m)
\end{aligned}
\end{equation}
First, notice that \eqref{eq-2bell-dobinski-proof-step1} has exactly the same form as the recurrence relation for the 2-Bell numbers as defined in \eqref{eq-2bell-recurrence}. Second, use the induction hypothesis and substitute $S(m)$:
\[
\begin{aligned}
S(n+2) &= \sum_{m=0}^{n}\binom{n}{m}2^{n-m} (v_m I_0(1)+u_m I_1(1)) \\
&= I_0(1)\bigg[\sum_{m=0}^{n}\binom{n}{m}2^{n-m}v_m\bigg] + I_1(1)\bigg[\sum_{m=0}^{n}\binom{n}{m}2^{n-m}u_m\bigg] \\
&= v_{n+2}I_0(1)+u_{n+2}I_1(1)
\end{aligned}
\]
which completes the proof. Indeed, if we were to enumerate a few more cases, we would see that $v_n$ and $u_n$ precisely match the sequences A351143 and A351028.

\begin{center}
\begin{tabular}{|c|l|c|c|c|}
\hline
$n$ & $S(n)$                & $v_n$ & $u_n$ & $v_n+u_n$ \\
\hline
0   & $I_0(1)$              & 1     & 0     & 1         \\
1   & $I_1(1)$              & 0     & 1     & 1         \\
2   & $I_0(1)$              & 1     & 0     & 1         \\
3   & $2I_0(1)+I_1(1)$    & 2     & 1     & 3         \\
4   & $5I_0(1)+4I_1(1)$   & 5     & 4     & 9         \\
5   & $16I_0(1)+13I_1(1)$ & 16    & 13    & 29        \\
6   & $61I_0(1)+44I_1(1)$ & 61    & 44 &    105 \\ 
\hline
\end{tabular}
\end{center}
\end{proof}

The major difference between Dobiński-like Equation \eqref{eq-2bell-dobinski-proof-step1} and the standard Dobiński formula for the Bell numbers is that there are two different modified Bessel functions acting as carriers for the coefficients of the two primitive Bessel-Bell sequences, rather than a single exponential function. In the standard Dobiński formula one multiplies the infinite sum by $1/e$ to cancel the exponential carrier function, which here is not possible. The recurrence relation and the coefficients just derived show that in principle the e.g.f. for these sequences can be as simple as $I_0(e^x)$, which is very close to the e.g.f. for the Bell numbers $\exp(-1)\exp(e^x)$. A careful examination of the coefficients in the full series expansion of the e.g.f.s in Theorem~\ref{thm-m2-egf} will reveal that the added factors in the e.g.f. serve the same role as $\exp(-1)$ in the Bell numbers e.g.f. - due to Wronskian identity described in Equation \eqref{eq-bessel-wronskian}, all Bessel functions get canceled out and we remain with a pure power series in $x$ without any special functions.

Due to the complicated series expansion of $K_0$ shown in Equation \eqref{eq-i-k-0-series}, this cancellation property is easier to see if we use the Maclaurin series for $I$ and $K$. Let $f(x) = I_0(e^x)$ and $g(x) = K_0(e^x)$. Then
\[
\begin{aligned}
f(x) &= \sum_{n=0}^\infty \frac{x^n}{n!} f^{(n)}(x)|_{x=0} \\
g(x) &= \sum_{n=0}^\infty \frac{x^n}{n!} g^{(n)}(x)|_{x=0}
\end{aligned}
\]
By comparing the coefficients of $f(x)$ with \eqref{eq-besseli-exp-maclaurin}, we know that $f^{(n)}(x)|_{x=0} = S(n) = v_n I_0(1)+u_nI_1(1)$ where $v_n$ and $u_n$ are as defined in Theorem~\ref{thm-bessel-dobiski}. What about the coefficients in $g(x)$? They are derivatives of $K_0(e^x)$, and since derivatives of $K$ are very similar to those of $I$ (Equation \ref{eq-bessel-deriv1}), we will end up with similar coefficients up to a sign change. Specifically we have the partner theorem to Theorem~\ref{thm-bessel-dobiski}:

\begin{theorem}\label{thm-besselk-dobinski}
\[
K_0(e^x) = \sum_{n=0}^\infty \frac{x^n}{n!}(v_n K_0(1) - u_n K_1(1))
\]
\end{theorem}

\begin{proof}
We need to prove that $K_0^{(n)}(e^x)|_{x=0} = v_n K_0(1)-u_n K_1(1)$. First, note that the derivatives of I and K have the same form for all orders, except for order 0, where the derivative of K has a negative sign (\href{https://dlmf.nist.gov/10.29.E3}{10.29.3}) \cite{NIST:DLMF}:
\[
\begin{aligned}
I_0'(e^x) &= I_1(e^x) \\
K_0'(e^x) &= -K_1(e^x) \\
I_1'(e^x) &= e^x I_0(e^x) + I_1(e^x) \\
K_1'(e^x) &= e^x K_0(e^x) + K_1(e^x)
\end{aligned}
\]
This means that the n-th derivative of a $K$ function will have the same form as the n-th derivative of an $I$ function, except that the coefficient in front of $K_1(1)$ will be the negated coefficient of $I_1(1)$. Since $I_0'(e^x)|_{x=0} = v_n I_0(1) + u_n I_1(1)$ it follows that $K_0'(e^x)|_{x=0} = v_n K_0(1)-u_n K_1(1)$.
\end{proof}

Finally, by combining Theorem~\ref{thm-bessel-dobiski} and Theorem~\ref{thm-besselk-dobinski}, we have:
\[
\begin{aligned}
p I_0(e^x) + q K_0(e^x) = \sum_{n=0}^\infty \frac{x^n}{n!} \bigg[ v_n (p I_0(1) + q K_0(1)) + u_n (p I_1(1) - q K_1(1))\bigg]
\end{aligned}
\]
Within this equation, an appropriate choice of $p$ and $q$ will neutralize the modified Bessel functions in the series due to the Wronskian property in Equation \eqref{eq-bessel-wronskian}. Specifically, $p = K_1(1), q = I_1(1)$ makes the coefficient of $v_n$ 1 and the coefficient of $u_n$ 0 and gives us integer coefficients for series A351143. Similarly for the other two choices of p and q established in Theorem~\ref{thm-m2-egf}.

Thus the e.g.f. of 2-Bell sequences is similar to that of the regular 1-Bell sequence - it is the composition of an exponential-like function with the exponential function. The remaining terms are there just to ensure integer coefficients. With the integer‑coefficient property now rigorously settled, we shift to the combinatorial side: a triangular recurrence whose row sums reproduce precisely these 2‑Bell sequences.

\section{Bessel-Stirling numbers of the second kind}\label{sec-2-stirling}

The similarity of the Bessel-Bell numbers e.g.f. to the standard Bell numbers e.g.f. immediately invites the question if there is also a Bessel-Stirling equivalent to the triangular array of the Stirling numbers of the second kind, whose rows sum to the Bell numbers. Recall that the Stirling numbers appears as the coefficients of the Touchard polynomials, which themseleves are the coefficients of a bi-variate generalization of the Bell e.g.f. (Equations\eqref{eq-touchard-def} and \eqref{eq-touchard-egf}). Turns out a analogous structure arises from a bi-variate generalization of the Bessel-Bell e.g.f.

\begin{theorem}[Bessel-Bell polynomials and Bessel-Stirtling numbers]\label{thm-bb-polynomials}
The Bessel-Bell numbers are the row sums of a Stirling-like triangular array, whose row entries arise as the coefficients of x of Touchard-like polynomials that are the coefficients of $t^n$ in the expansion of the following bi-variate e.g.f.:
    \begin{equation*}
        \begin{aligned}
        G(t,x) &= x\big[K_0(x)+K_1(x)\big]\, I_0(x e^t) + x\big[I_1(x)-I_0(x)\big]\,K_0(xe^t)\\
                       &= \sum_{n=0}^\infty \frac{t^n}{n!}\mathscr{B}_n(x) \\
                       & = \sum_{n=0}^\infty\frac{t^n}{n!}\sum_{k=0}^{n}\mStirling{n}{k}_2 x^k, \\
        \end{aligned}
    \end{equation*}
    where $\mathscr{B_n}(1)$ gives us the n-th Bessel-Bell number (A007472). $\mStirling{n}{k}_2$ uses a similar notation as for the Binomial coefficients $\binom{n}{k}$ or the standard Stirling numbers $\Stirling{n}{k}$ popularized by \cite{graham_concrete_nodate}.
\end{theorem}

\begin{proof}
The claim that $\mathscr{B}_n(1)$ represents the n-th Bessel-Bell number follows simply from the facty that $G(t,1)$ reduces to the e.g.f. of the Bessel-Bell numbers (Theorem \ref{thm-m2-egf}). Therefore what we need to prove is that $\mathscr{B}(x)$ is an n-th degree polynomial in x. As we saw in the previous section, we can work with a much simpler version of the e.g.f., namely, $I_0(xe^t)$ and show that the remaining components are present only to cancel the carrier special functions. The standard series expansion of $I_0(xe^t)$ gives:
\[
\begin{aligned}
I_0(x e^t) = \sum_{k=0}^\infty \frac{(x/2)^{2k} e^{2kt}}{k!k!} = \sum_{k=0}^\infty \frac{(x/2)^{2k}}{k!k!} \sum_{n=0}^\infty \frac{(2k)^n t^n}{n!} = \sum_{n=0}^\infty \frac{t^n}{n!} \sum_{k=0}^\infty \frac{(x/2)^{2k}(2k)^n}{k!k!},
\end{aligned}
\]
As noted in Equation \eqref{eq-scaling-operator}, we can represent this series also by the action of the exponential scaling operator on $I_0(x)$, which in this case provides a more fruitful approach:
\begin{equation}
   I_0(x e^t)= e^{txD}I_0(x)= \sum_{n=0}^{\infty}\frac{t^n(xD)^n}{n!}I_0(x)
\end{equation}

Combining our knowledge of the derivative of the 0-th order modified Bessel function $I_0(x)$ with that of the 1-st order function $I_0(x)$ (\href{https://dlmf.nist.gov/10.29.E2}{10.29.2} \cite{NIST:DLMF}), we can express a recurrence relation for the operator $(xD)$ acting on modified Bessel functions of the first kind:
\begin{equation}
    \begin{aligned}
        (xD)I_0(x) &= x I_1(x) \\
        (xD)I_1(x) &= x I_0(x)-I_1(x)
    \end{aligned}
\end{equation}
This recurrence involves only x, $I_0(x)$ and $I_1(x)$ functions and by the chain rule, each application of the $(xD)$ operator adds a single degree x to the resulting polynomial. Then it is clear that we can express $(xD)^nI_0(x)$ as an $n$-th degree polynomial in x, where each degree is "carried" by a Bessel function: 
\begin{equation}
    \begin{aligned}
        I_0(xe^t) &= \sum_{n=0}^\infty\frac{t^n}{n!}V_n(x) \\
        V_n(x) &= \sum_{k=0}^{n} \mStirling{n}{k}_2 x^k\, I_{k \bmod 2}(x)
    \end{aligned}
\end{equation}
Here are the first few $V$ polynomials:
\[
\begin{aligned}
V_0(x) &= I_0(x) \\
V_1(x) &= x\,I_1(x) \\
V_2(x) &= x^2\,I_0(x) \\
V_3(x) &= 2x^2\,I_0(x) + x^3\,I_1(x) \\
V_4(x) &= 4x^2\,I_0(x) + 4x^3\,I_1(x) + x^4\,I_0(x) \\
V_5(x) &= 8x^2\,I_0(x) + 12x^3\,I_1(x) + 8x^4\,I_0(x) + x^5\,I_1(x) \\
\end{aligned}
\]
Notice two things. First, these polynomials are generated in very much the same way as Touchard polynomials, but by applying the exponential scaling operator to $I_0(x)$ instead of to $e^x$. Second, each mononomial $x^k$ is accompanied by a modified Bessel carrier function - $I_0$ "carries" the even degrees of x and $I_1$ "carries" the odd degrees of x. The proper analogue to the Touchard polynomials stated in Theorem \ref{thm-bb-polynomials} would not have these carrier functions. Thankfully, the solution is the same as that for the e.g.f. of the 2-Bell numbers derived in Section \ref{sec-m2} - to use the general Wronskian property of Bessel functions \eqref{eq-bessel-wronskian-general} to construct a bi-variate e.g.f. that leads to the cancellation of all carrier functions. Some straightforward but tedious algebra, mirroring the steps in Section~\ref{2-bell-integer-coefs}, confirms that the chosen form of $G(t,x)$ precisely cancels the carrier functions. We omit the details.
\end{proof}

The numbers $\mStirling{n}{k}_2$ form a triangular array whose first few rows are:
\begin{center}
\begin{tabular}{|l|*{10}{c}|c|}
\hline
      & k=0 & k=1 & k=2 & k=3 & k=4 & k=5 & k=6 & k=7 & k=8 & k=9 & $\sum_{k=0}^{n}$ \\
\hline
n=0 & 1   &     &     &     &     &     &     &     &     &     & 1                \\
n=1 & 0   & 1   &     &     &     &     &     &     &     &     & 1                \\
n=2 & 0   & 0   & 1   &     &     &     &     &     &     &     & 1                \\
n=3 & 0   & 0   & 2   & 1   &     &     &     &     &     &     & 3                \\
n=4 & 0   & 0   & 4   & 4   & 1   &     &     &     &     &     & 9                \\
n=5 & 0   & 0   & 8   & 12  & 8   & 1   &     &     &     &     & 29               \\
n=6 & 0   & 0   & 16  & 32  & 44  & 12  & 1   &     &     &     & 105              \\
n=7 & 0   & 0   & 32  & 80  & 208 & 92  & 18  & 1   &     &     & 431              \\
n=8 & 0   & 0   & 64  & 192 & 912 & 576 & 200 & 24  & 1   &     & 1969             \\
n=9 & 0   & 0   & 128 & 448 & 3840& 3216& 1776& 344 & 32  & 1   & 9785             \\
\hline
\end{tabular}
\smallskip
\end{center}
where the values in the upper triangle are all 0. The row sums of this array yield the Bessel-Bell numbers \href{https://oeis.org/A007472}{A007472}
\[
B_n^{(2)} = \sum_{k=0}^{n}\genfrac{\lfloor}{\rfloor}{0pt}{}{n}{k}_2 = (1, 1, 1, 3, 9, 29, 105, 431, 1969, 9785, 52145, 296155, 1787385, ...)
\]
Examining the coefficients of this triangle, which I will refer to as the Bessel-Stirling numbers of the second kind, leads to the following conjecture about their recurrence relation:
\begin{conjecture}[Recurrence relation of the Bessel-Stirling numbers]\label{conj-bs-recurrence}
\[
\mStirling{n+1}{k}_2 = 2\floor{k/2} \, \mStirling{n}{k}_2 + \mStirling{n}{k-1}_2, \,\,\,\,\, (n, k \ge 1)\\
\]    
\end{conjecture}
This recurrence relation is almost identical to that for the Stirling numbers of the second kind, except that the coefficient in front of $\mStirling{n}{k}$ has parity - it is always even, with k rounded down to the nearest even integer. Computational results with Mathematica confirm that the coefficients defined in Theorem \ref{thm-bb-polynomials} and those in Conjecture \ref{conj-bs-recurrence} produce the triangle up to reasonably large $n$. This triangle is now available as OEIS entry \href{https://oeis.org/A383235}{A383235}.

The Bessel-Sitrling numbers are a novel generalization of the Stirling numbers and joins a broader family of two-term recurrences (\cite{mansour_general_2012},\cite{neuwirth_recursively_2001},\cite{barbero_g_bivariate_2014},\cite{spiveySolutionsGeneralCombinatorial2011}). I will return to the connection between this new array and previous generalizations of the Stirling numbers in Section \ref{sec-final-remarks}.

Due to the form of their recurrence, the Bessel-Stirling numbers admit many identities common to other such two-term recurrence triangular arrays. Here I list the most interesting of them.
\begin{theorem}[Explicit formula for the Bessel-Stirling numbers of the second kind]
The Bessel-Stirling numbers can be calculated explicitly for $n \ge 2, k \geq 2$ via  the following formula:
\begin{equation*}\label{z-d-c}
    \begin{aligned}
        \mStirling{n}{k}_2&= \sum_{1 \leq j \leq \floor{\frac{k}{2}}}(2j)^{n-k}\left (C_{k,j} +D_{k,j}(n-k+1) \right )\\
            D_{k,r} &= \frac{(-r)^{k-1}}{r!r!(\floor{\frac{k}{2}}-r)!(\floor{\frac{k-1}{2}}-r)!} \\
            C_{k,r} &= \begin{cases}
                - D_{k,r} (r(2H_{r-1}-H_{\floor{\frac{k}{2}}-r}-H_{\floor{\frac{k-1}{2}}-r})-k+3) & \text{if} \, 1 \leq r < k/2\\
                \frac{r^{2r}}{r!r!} & \text{if}  \, r = k/2\\
                0 & \text{otherwise}
            \end{cases} 
    \end{aligned}
\end{equation*}
Or via the following limit definition, valid for all $n\ge0, k\ge 0$:
$$
\begin{aligned}
    (-1)^{k-1} 2^{n-k} \sum_{j=0}^{\floor{k/2}} \lim_{t \to j} \left[t^{n-1}
        \frac
            {n-2-t\,\left(2\psi(t)-\psi\left(\floor{\frac{k}{2}}-t+1\right)-\psi\left(\floor{\frac{k-1}{2}}-t+1\right)\right)}
            {\Gamma(t+1) \Gamma(t+1) \Gamma(\floor{\frac{k}{2}}-t+1) \Gamma(\floor{\frac{k-1}{2}}-t+1)}\right]
\end{aligned}
$$
\end{theorem}

\section{Combinatorial interpretation}\label{sec-combinatorial}

\section{The general case}\label{sec-general-case}

\begin{conjecture}m-Bell numbers are the row sums of m-Stirling numbers of the second kind triangular arrays.
\[
B_{n}^{(m)} = \sum_{k=0}^n \mStirling{n}{k}_m
\]
where $\mStirling{n}{k}_m$ is a number satisfying the following two-term recurrence relation:
\[
\mStirling{n+1}{k}_m = m\floor{k/m} \, \genfrac{\lfloor}{\rfloor}{0pt}{}{n}{k}_m + \genfrac{\lfloor}{\rfloor}{0pt}{}{n}{k-1}_m, \,\,\,\,\,
\]
for $n, k \ge 0$ and initial conditions $\genfrac{\lfloor}{\rfloor}{0pt}{}{n}{k}_m = \delta_{n,k}$ for $k\ge n$.
\end{conjecture}

\section{Moments of the Conway-Maxwell-Poisson distribution}\label{sec-cmp}

\section{Final remarks}\label{sec-final-remarks}

\printbibliography
\end{document}