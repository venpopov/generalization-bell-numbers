% Options for packages loaded elsewhere
\PassOptionsToPackage{unicode}{hyperref}
\PassOptionsToPackage{hyphens}{url}
\PassOptionsToPackage{dvipsnames,svgnames,x11names}{xcolor}
%
\documentclass[
  letterpaper,
  DIV=11,
  numbers=noendperiod]{scrartcl}

\usepackage{amsmath,amssymb}
\usepackage{iftex}
\ifPDFTeX
  \usepackage[T1]{fontenc}
  \usepackage[utf8]{inputenc}
  \usepackage{textcomp} % provide euro and other symbols
\else % if luatex or xetex
  \usepackage{unicode-math}
  \defaultfontfeatures{Scale=MatchLowercase}
  \defaultfontfeatures[\rmfamily]{Ligatures=TeX,Scale=1}
\fi
\usepackage{lmodern}
\ifPDFTeX\else  
    % xetex/luatex font selection
\fi
% Use upquote if available, for straight quotes in verbatim environments
\IfFileExists{upquote.sty}{\usepackage{upquote}}{}
\IfFileExists{microtype.sty}{% use microtype if available
  \usepackage[]{microtype}
  \UseMicrotypeSet[protrusion]{basicmath} % disable protrusion for tt fonts
}{}
\makeatletter
\@ifundefined{KOMAClassName}{% if non-KOMA class
  \IfFileExists{parskip.sty}{%
    \usepackage{parskip}
  }{% else
    \setlength{\parindent}{0pt}
    \setlength{\parskip}{6pt plus 2pt minus 1pt}}
}{% if KOMA class
  \KOMAoptions{parskip=half}}
\makeatother
\usepackage{xcolor}
\setlength{\emergencystretch}{3em} % prevent overfull lines
\setcounter{secnumdepth}{-\maxdimen} % remove section numbering
% Make \paragraph and \subparagraph free-standing
\makeatletter
\ifx\paragraph\undefined\else
  \let\oldparagraph\paragraph
  \renewcommand{\paragraph}{
    \@ifstar
      \xxxParagraphStar
      \xxxParagraphNoStar
  }
  \newcommand{\xxxParagraphStar}[1]{\oldparagraph*{#1}\mbox{}}
  \newcommand{\xxxParagraphNoStar}[1]{\oldparagraph{#1}\mbox{}}
\fi
\ifx\subparagraph\undefined\else
  \let\oldsubparagraph\subparagraph
  \renewcommand{\subparagraph}{
    \@ifstar
      \xxxSubParagraphStar
      \xxxSubParagraphNoStar
  }
  \newcommand{\xxxSubParagraphStar}[1]{\oldsubparagraph*{#1}\mbox{}}
  \newcommand{\xxxSubParagraphNoStar}[1]{\oldsubparagraph{#1}\mbox{}}
\fi
\makeatother


\providecommand{\tightlist}{%
  \setlength{\itemsep}{0pt}\setlength{\parskip}{0pt}}\usepackage{longtable,booktabs,array}
\usepackage{calc} % for calculating minipage widths
% Correct order of tables after \paragraph or \subparagraph
\usepackage{etoolbox}
\makeatletter
\patchcmd\longtable{\par}{\if@noskipsec\mbox{}\fi\par}{}{}
\makeatother
% Allow footnotes in longtable head/foot
\IfFileExists{footnotehyper.sty}{\usepackage{footnotehyper}}{\usepackage{footnote}}
\makesavenoteenv{longtable}
\usepackage{graphicx}
\makeatletter
\newsavebox\pandoc@box
\newcommand*\pandocbounded[1]{% scales image to fit in text height/width
  \sbox\pandoc@box{#1}%
  \Gscale@div\@tempa{\textheight}{\dimexpr\ht\pandoc@box+\dp\pandoc@box\relax}%
  \Gscale@div\@tempb{\linewidth}{\wd\pandoc@box}%
  \ifdim\@tempb\p@<\@tempa\p@\let\@tempa\@tempb\fi% select the smaller of both
  \ifdim\@tempa\p@<\p@\scalebox{\@tempa}{\usebox\pandoc@box}%
  \else\usebox{\pandoc@box}%
  \fi%
}
% Set default figure placement to htbp
\def\fps@figure{htbp}
\makeatother
% definitions for citeproc citations
\NewDocumentCommand\citeproctext{}{}
\NewDocumentCommand\citeproc{mm}{%
  \begingroup\def\citeproctext{#2}\cite{#1}\endgroup}
\makeatletter
 % allow citations to break across lines
 \let\@cite@ofmt\@firstofone
 % avoid brackets around text for \cite:
 \def\@biblabel#1{}
 \def\@cite#1#2{{#1\if@tempswa , #2\fi}}
\makeatother
\newlength{\cslhangindent}
\setlength{\cslhangindent}{1.5em}
\newlength{\csllabelwidth}
\setlength{\csllabelwidth}{3em}
\newenvironment{CSLReferences}[2] % #1 hanging-indent, #2 entry-spacing
 {\begin{list}{}{%
  \setlength{\itemindent}{0pt}
  \setlength{\leftmargin}{0pt}
  \setlength{\parsep}{0pt}
  % turn on hanging indent if param 1 is 1
  \ifodd #1
   \setlength{\leftmargin}{\cslhangindent}
   \setlength{\itemindent}{-1\cslhangindent}
  \fi
  % set entry spacing
  \setlength{\itemsep}{#2\baselineskip}}}
 {\end{list}}
\usepackage{calc}
\newcommand{\CSLBlock}[1]{\hfill\break\parbox[t]{\linewidth}{\strut\ignorespaces#1\strut}}
\newcommand{\CSLLeftMargin}[1]{\parbox[t]{\csllabelwidth}{\strut#1\strut}}
\newcommand{\CSLRightInline}[1]{\parbox[t]{\linewidth - \csllabelwidth}{\strut#1\strut}}
\newcommand{\CSLIndent}[1]{\hspace{\cslhangindent}#1}

\KOMAoption{captions}{tableheading}
\makeatletter
\@ifpackageloaded{caption}{}{\usepackage{caption}}
\AtBeginDocument{%
\ifdefined\contentsname
  \renewcommand*\contentsname{Table of contents}
\else
  \newcommand\contentsname{Table of contents}
\fi
\ifdefined\listfigurename
  \renewcommand*\listfigurename{List of Figures}
\else
  \newcommand\listfigurename{List of Figures}
\fi
\ifdefined\listtablename
  \renewcommand*\listtablename{List of Tables}
\else
  \newcommand\listtablename{List of Tables}
\fi
\ifdefined\figurename
  \renewcommand*\figurename{Figure}
\else
  \newcommand\figurename{Figure}
\fi
\ifdefined\tablename
  \renewcommand*\tablename{Table}
\else
  \newcommand\tablename{Table}
\fi
}
\@ifpackageloaded{float}{}{\usepackage{float}}
\floatstyle{ruled}
\@ifundefined{c@chapter}{\newfloat{codelisting}{h}{lop}}{\newfloat{codelisting}{h}{lop}[chapter]}
\floatname{codelisting}{Listing}
\newcommand*\listoflistings{\listof{codelisting}{List of Listings}}
\usepackage{amsthm}
\theoremstyle{plain}
\newtheorem{theorem}{Theorem}[section]
\theoremstyle{remark}
\AtBeginDocument{\renewcommand*{\proofname}{Proof}}
\newtheorem*{remark}{Remark}
\newtheorem*{solution}{Solution}
\newtheorem{refremark}{Remark}[section]
\newtheorem{refsolution}{Solution}[section]
\makeatother
\makeatletter
\makeatother
\makeatletter
\@ifpackageloaded{caption}{}{\usepackage{caption}}
\@ifpackageloaded{subcaption}{}{\usepackage{subcaption}}
\makeatother

\usepackage{bookmark}

\IfFileExists{xurl.sty}{\usepackage{xurl}}{} % add URL line breaks if available
\urlstyle{same} % disable monospaced font for URLs
\hypersetup{
  pdftitle={On a New Family of Stirling-like triangular arrays and Bell-like numbers},
  colorlinks=true,
  linkcolor={blue},
  filecolor={Maroon},
  citecolor={Blue},
  urlcolor={Blue},
  pdfcreator={LaTeX via pandoc}}


\title{On a New Family of Stirling-like triangular arrays and Bell-like
numbers}
\author{Vencislav Popov}
\date{}

\begin{document}
\maketitle


\usepackage{amsmath}
\usepackage{logix}
\usepackage{hyperref}
\newcommand{\Stirling}[0]{\genfrac\{\}{0pt}{}}
\newcommand{\mStirling}[0]{\genfrac{\lfloor}{\rfloor}{0pt}{}}

\subsection{1. Introduction}\label{introduction}

Given a sequence \((a_n)_{n \geq 0} = (a_0, a_1, \ldots)\), the binomial
transform is the procedure that maps it to a new sequence
\((b_n)_{n \geq 0} = (b_0,b_1,...)\) as follows:

\[
b_n = \sum_{k=0}^{n} \binom{n}{k} a_k
\]

Repeated applications of the binomial transform on the resulting
sequence can be summarized with a single sum:

\[
b_n = \sum_{k=0}^{n} \binom{n}{k} m^{n-k} a_k,
\]

where \(m\) is an integer that represents the number of times the
binomial transform has been applied. Bernstein and Sloane {[}1{]}
studied a number of what they call ``Eigen sequences'' of various such
transformations - sequences \((a_n)\) which when transformed one or more
times shift by one or more places but are otherwise preserved. Such
sequences show a ``self-similarity'' under an iterated transform and
understanding why this self-similarity occurs often reveals new
properties or relations between different integer sequences.

Perhaps the most famous case of a binomial-transform invariant sequence
is that of the Bell numbers (\href{https://oeis.org/A000110}{A000110}:
\(1, 1, 2, 5, 15, 52, 203, \ldots\)), which shift one place to the left
after a single binomial transformation:

\begin{equation}\phantomsection\label{eq-bell-recurrence}{
B_{n+1} = \sum_{k=0}^{n} \binom{n}{k} B_k
}\end{equation}

The Bell numbers count the total number of partitions of an n-element
set and they are part of a rich combinatorial structure that involves
the Stirling numbers the of the second kind
(\href{https://oeis.org/A008277}{A008277}), Touchard polynomials
{[}2{]}, the exponential function and linear operators acting on it
{[}3{]}.

To simplify the notation, following Bernstein and Sloane {[}1{]}, define
the following operators on sequences:

\[
\begin{aligned}
\mathrm{BINOM} \circ [a_0,a_1,a_2, \dots] &= \bigg[\binom{0}{0}a_0,\binom{1}{0}a_0+\binom{1}{1}a_1, \ldots, \sum_{k=0}^n\binom{n}{k}a_k, \ldots \bigg] \\
L \circ [a_0,a_1,a_2, \ldots] &= [a_1,a_2, \ldots]
\end{aligned}
\]

where \(\mathrm{BINOM}\) is the binomial transform operator applied to
sequence \(a\), whereas \(L\) is the left-shift operator. Then the Bell
numbers form the unique sequence that satisfies the following equality
with initial condition \(a_0 = 1:\)

\[
\mathrm{BINOM} \circ a = \mathrm{L} \circ a
\]

In this paper I study a new generalization of the Bell numbers defined
by the following property:

\begin{equation}\phantomsection\label{eq-mbell-recurrence}{
B^{(m)}_{n+m} = \sum_{k=0}^n \binom{n}{k} m^{n-k} B_k^{(m)}
}\end{equation}

where the upper index \((m)\) should be read as the ``m-Bell'' number.
Equivalently, these sequences satisfy the operator equation:

\[
\mathrm{BINOM^m} \circ a = \mathrm{L}^m \circ a
\]

These are sequences that shift to the left by \(m\) places after \(m\)
applications of the binomial transform. The case \(m=1\) corresponds to
the Bell numbers, and \(m=2\) corresponds to sequences
\href{https://oeis.org/A007472}{A007472},
\href{https://oeis.org/search?q=1,0,1,2,5,16&language=english&go=Search}{A351143}
and \href{https://oeis.org/A351028}{A351028}, which shift by 2 places
left after 2 binomial transformations (sequences for \(m>2\) are not
currently present in OEIS). Although the sequences for \(m=2\) are
listed in OEIS, little is known about their properties.

In the remainder of this paper I show that the m-fold
binomial--shift-invariance property that characterizes these sequences
arises from a combinatorial structure that mirrors that of the regular
Bell numbers. Each such \emph{m-Bell} sequence corresponds to the row
sums of new Stirling-like arrays, motivating the name \emph{m-Stirling
numbers} (which come in dual pairs). The m-Bell numbers have explicit
exponential generating functions (e.g.f.) that are the solutions of
ordinary differential equations of order \(m\). Each of the associated
m-Stirling arrays arises as the coefficients of polynomials that result
from the application of the exponential shift operator to hypergeometric
functions, which have simple forms for \(m=1\) (in terms of the
exponential function) and \(m=2\) (in terms of modified Bessel functions
of the first and second kinds). I conclude by providing a combinatorial
interpretation for the general case as well as showing a connection to
the Conway-Maxwell-Poisson distribution {[}4{]}.

To clearly establish the analogue between the Bell-Stirling-Touchard
framework and the novel results, I begin with a review of standard
results and notation {[}5{]} {[}6{]}.

\subsection{2. Background}\label{background}

\subsubsection{2.1. Exponential generating functions and effects of
sequence
transformations}\label{exponential-generating-functions-and-effects-of-sequence-transformations}

The exponential generating function (e.g.f.) of a sequence
\((a_n)_{n \geq 0}\) is a formal power series in \(x\):

\[
\mathcal{A}(x) = \sum_{n=0}^\infty a_n \frac{x^n}{n!}
\]

Both the binomial transform and the left-shift operator acting on \(a\)
have known effects on the e.g.f., \(\mathcal{B}(x)\), of the resulting
sequence \(b\) {[}1{]}:

\begin{equation}\phantomsection\label{eq-egf-transform}{
\begin{aligned}
b = \mathrm{BINOM} \circ a \,\,\, &\iff \mathcal{B}(x) = e^x \mathcal{A}(x) \\
b = \mathrm{L} \circ a \,\,\, &\iff \mathcal{B}(x) = \mathcal{A}'(x)
\end{aligned}
}\end{equation}

These relations can be used to establish an ordinary differential
equation (ODE) which the e.g.f. of sequences like those defined by
Equation~\ref{eq-mbell-recurrence} must satisfy. In the case of the Bell
numbers, this is simply:

\[
\mathcal{A}'(x)-e^x \mathcal{A(x)} = 0
\]

\subsubsection{2.1. Bell numbers, Stirling numbers, Touchard polynomials
and operator
calculus}\label{bell-numbers-stirling-numbers-touchard-polynomials-and-operator-calculus}

The solution of the ODE for the Bell numbers is the well known e.g.f.:

\[
e^{e^t-1} = \sum_{n=0}^{\infty} B_n \frac{t^n}{n!}
\]

Recall that the Bell numbers are the row sums of a triangle array formed
by the Stirling numbers of the second kind:

\[
\begin{align}
B_n = \sum_{k=0}^{n}\genfrac\{\}{0pt}{}{n}{k},
\end{align}
\]

where the Stirling numbers of the second kind satisfy the recurrence:

\[
\genfrac\{\}{0pt}{}{n+1}{k} = k \, \genfrac\{\}{0pt}{}{n}{k} + \genfrac\{\}{0pt}{}{n}{k-1}, \,\,\,\,\, n, k ≥ 1\\
\]

The Stirling numbers of the second kind count the number of ways to
partition n labeled objects into k unlabeled subsets and are the
coefficients of the Touchard (also known as Bell or exponential)
polynomials:

\[
T_n(x) = \sum_{k=0}^{n} \genfrac\{\}{0pt}{}{n}{k} x^k,
\]

whose e.g.f. is a more general case of the Bell numbers e.g.f.::

\[
\sum_{n=0}^{\infty} T_n(x) \frac{t^n}{n!} = e^{x(e^t-1)}
\]

Many useful identities involving Touchard polynomials and Stirling
numbers can be shown via the action of the exponential scaling operator:

\[
e^{txD_x}f(x)=f(xe^t)
\]

where \(D_x\) is the derivative operator with respect to \(x\).
Specifically, by applying this operator to the standard exponential
function \(e^x\) we get precisely the e.g.f. for Touchard polynomials
{[}3{]}:

\[
e^{-x}e^{txD_x}e^x = e^{x(e^t-1)}
\]

A final useful relation is the \textbf{Dobiński} formula that let us
express \(T_n(x)\) and the Bell numbers \(B_n = T_n(1)\) as an infinite
sum:

\begin{equation}\phantomsection\label{eq-bell-dobinski}{
\begin{aligned}
T_n(x) & =e^{-x}\sum_{k=0}^{\infty}\frac{x^k k^n}{k!} \\
B_n & = \frac{1}{e}\sum_{k=0}^{\infty}\frac{k^n}{k!}
\end{aligned}
}\end{equation}

In the remainder of the paper we will see that the sequences for
\(m = 2\) have many analogous properties as those defined above.

\subsection{3. The case m = 2}\label{the-case-m-2}

\subsubsection{3.1. Exponential generating
functions}\label{exponential-generating-functions}

Before solving the general case, let us focus on the \(m=2\)
generalization of the Bell numbers. The following three sequences all
share the property that they shift by 2 places left after two binomial
transformations:

\begin{itemize}
\item
  \href{https://oeis.org/A007472}{A007472}:
  \(1, 1, 1, 3, 9, 29, 105, 431, 1969, \ldots\)
\item
  \href{https://oeis.org/A351143}{A351143}:
  \(1, 0, 1, 2, 5, 16, 61, 258, 1177, \ldots\)
\item
  \href{https://oeis.org/A351028}{A351028}:
  \(0, 1, 0, 1, 4, 13, 44, 173, 792, 4009, \ldots\)
\end{itemize}

The main difference between them are the initial conditions
\((a_0, a_1)\): \((1, 1); (1,0); (0,1)\). They all satisfy the property:

\begin{equation}\phantomsection\label{eq-2bell-recurrence}{
B^{(2)}_{n+2} = \sum_{k=0}^n \binom{n}{k} 2^{n-k} B_k^{(2)}
}\end{equation}

\begin{theorem}[]\protect\hypertarget{thm-m2-egf}{}\label{thm-m2-egf}

\emph{The exponential generating functions for 2-Bell numbers are a
linear combination of modified Bessel functions of the first and second
kind of order 0 whose weights are uniquely determined by the first two
elements of the series. Specifically:}

\[
\begin{aligned}
\mathcal{A}_{351143}(x) &= K_1(1)I_0(e^x) + I_1(1)K_0(e^x) \\
\mathcal{A}_{351008}(x) &= K_0(1)I_0(e^x) - I_0(1)K_0(e^x) \\
\mathcal{A}_{007472}(x) = \mathcal{A}_{351143}(x) + \mathcal{A}_{351008}(x) &= [K_0(1) + K_1(1)]I_0(e^x) + [I_0(1) - I_1(1)]K_0(e^x)
\end{aligned}
\]

\end{theorem}

\begin{proof}
Using the operator notation introduced in section 1, 2-Bell numbers
satisfy the operator equality
\(\mathrm{BINOM}^2 \circ a = \mathrm{L}^2 \circ a\). Using
Equation~\ref{eq-egf-transform}, we see that the e.g.f. of these
sequences must satisfy the following differential equation:

\[
\mathcal{A}''(x)-e^{2x}A(x) =0
\]

To solve the ODE, first, use the substitutions
\(t=e^x, \frac{dt}{dx}=t, f(t)=\mathcal{A}(x)\) to simplify it:

\[
\begin{aligned}
\mathcal{A}'(x) & = \frac{d\mathcal{A}}{dx}=\frac{d\mathcal{A}}{dt}\frac{dt}{dx} = t \,f'(t) \\
\mathcal{A}''(x) &= \frac{d}{dx}\mathcal{A}'(x) = \frac{d}{dx}t f'(t) = \frac{dt}{dx} f'(t) + t \frac{d}{dx}f'(t) = t f'+t^2 f'' \\
e^{2x} \mathcal{A}(x) &= t^2f\\
\end{aligned}
\]

We can now express the ODE as follows:

\begin{equation}\phantomsection\label{eq-m2-ode}{
t^2f''+tf'-t^2f = 0
}\end{equation}

In this standard form, it is easy to see that Equation~\ref{eq-m2-ode}
is the modified Bessel equation for the case of \(n=0\) (see
\href{https://dlmf.nist.gov/10.25.E1}{10.25.1}){[}7{]}, whose general
solution is:

\[
f(t)=p\,I_0(t)+q\,K_0(t)
\]

with \(I_n\) and \(K_n\) being the modified Bessel functions of the
first and second kind, and \(p\) and \(q\) are constants determined by
the initial conditions. Therefore, the general solution for
\(\mathcal{A}(x)\), the e.g.f. of sequences
\href{https://oeis.org/A007472}{A007472},
\href{https://oeis.org/A351143}{A351143} and
\href{https://oeis.org/A351028}{A351028} is:

\begin{equation}\phantomsection\label{eq-m2-ODE-solution}{
\mathcal{A}(x)=p\, I_o(e^x) + q\,K_0(e^x)
}\end{equation}

To determine \(p\) and \(q\) we use the known initial conditions.
Sequence A007472 is the element-wise sum of A351143 and A351028, so it
is sufficient to determine \(p\) and \(q\) for the latter 2 only. For
A351143 we have \(a_0 = 1\) and \(a_1 = 0\). Thus, we have the following
system of equations:

\[
\begin{aligned}
A(0) & = p\, I_0(1) + q\,K_0(1) = 1 \\
A'(x)\bigr|_{x=0} &=p\, I_0'(e^x)\bigr|_{x=0} + q\,K_0'(e^x)\bigr|_{x=0} = 0
\end{aligned}
\]

The derivative of \(I_0\) and \(K_0\) are thankfully straightforward
(\href{https://dlmf.nist.gov/10.29.E3}{10.29.3}) {[}7{]}:

\[
\begin{aligned}
I_0'(x) &= I_1(x) \\ K_0'(x) &= -K_1(x) \\I_0'\bigl(e^x\bigr) &= e^xI_1\bigl(e^x\bigr) \\ K_0'\bigl(e^x\bigr) &= -e^xK_1\bigl(e^x\bigr) \\
\end{aligned}
\]

Evaluating at \(x=0\):

\[
\begin{aligned}
A(0) & = p \,I_0(1)+q\,K_0(1) = 1 \\
A'(0) & = p\,I_1(1)-q K_1(1) = 0
\end{aligned}
\]

After some algebraic torture we get expressions for p and q:

\[
\begin{aligned}
p &= \frac{1-qK_0(1)}{I_0(1)} \\
q &= p\frac{I_1(1)}{K_1(1)} = \frac{I_1(1)-qI_1(1)K_0(1)}{I_0(1)K_1(1)} \\
q &= \frac{I_1(1)}{I_1(1)K_0(1)+I_0(1)K_1(1)}
\end{aligned}
\]

These expressions can be simplified due to the following Bessel identity
concerning the Wronskian of the modified Bessel functions (see
\href{https://dlmf.nist.gov/10.28.E2}{10.28.2}):

\[
I_v(z)K_{v+1}(z)+I_{v+1}(z)K_v(z) = 1/z
\]

which holds for any complex \(v\) and \(z\). In the special case when
\(v = 0\) and \(z=1\):

\begin{equation}\phantomsection\label{eq-bessel-wronskian}{
I_0(1)K_1(1)+I_1(1)K_0(1) = 1
}\end{equation}

Therefore the constants for the e.g.f. of sequence A351143 are:

\[
\begin{aligned}
q &= I_1(1) \\
p &= K_1(1)
\end{aligned}
\]

With a very similar manipulation for the initial conditions of A351028
and A007472 we get the coefficients in Theorem~\ref{thm-m2-egf} which
concludes the proof.
\end{proof}

\subsubsection{3.2. Series expansion of the
e.g.f.}\label{series-expansion-of-the-e.g.f.}

The relatively complicated expression for the e.g.f. of 2-Bell numbers
made me question how it can result in integer coefficients. Let us
explore the series expansion for the derved e.g.f.s. Consider first the
case of sequence A351143:

\[
\mathcal{A}_{351143}(x) = K_1(1)I_0(e^x) + I_1(1)K_0(e^x)
\]

We have the following standard series for \(I_0\) and \(K_0\):

\[
\begin{aligned}
I_0(z) &= \sum_{k=0}^{\infty} \frac{(\frac{1}{2}z)^{2k}}{k!k!} \\
K_0(z) &= -\log\biggl(\frac{z}{2}\biggr)I_0(z) + \sum_{k=0}^{\infty}\frac{\psi(k+1)(\frac{1}{2}z)^{2k}}{k!k!}
\end{aligned}
\]

where \(\psi\) is the digamma function. Let us focus on the \(I_0\)
function. Substituting \(z=e^x\) and then expanding the Taylor series
for the exponential function we get:

\[
\begin{aligned}
I_0(e^x) &= \sum_{k=0}^{\infty} \frac{e^{2xk}}{2^{2k}k!k!} = \sum_{k=0}^{\infty} \frac{1}{2^{2k}k!k!}\sum_{n=0}^{\infty}\frac{x^n(2k)^n}{n!} = \sum_{n=0}^{\infty}\frac{x^n}{n!}\sum_{k=0}^{\infty} \frac{(2k)^n}{2^{2k}k!k!}
\end{aligned}
\]

Let the inner sum be represented by
\(S(n) = \sum_{k=0}^{\infty} \frac{(2k)^n}{2^{2k}k!k!}\). This formula
is reminiscent of the \textbf{Dobiński} formula for the standard Bell
numbers (Equation~\ref{eq-bell-dobinski}) with an extra factorial in the
denominator and extra powers of 2. Indeed, we can state an equivalent
theorem:

\begin{theorem}[]\protect\hypertarget{thm-bessel-dobiski}{}\label{thm-bessel-dobiski}

\[
\begin{aligned}
S(n) &= \sum_{k=0}^{\infty} \frac{(2k)^n}{(2^{k}k!)^2}=v_n I_0(1) + u_n I_1(1) \\
A351143(n) &= v_n \\
A351028(n) &= u_n \\
A007472(n) &= v_n+u_n
\end{aligned}
\]

\end{theorem}

\phantomsection\label{.proof}
Proceed by induction. First, establish the base cases:

\[
\begin{aligned}
S(0) &= \sum_{k=0}^{\infty} \frac{1}{2^{2k}k!k!} = I_0(1) \\
S(1) &= \sum_{k=0}^{\infty} \frac{2k}{2^{2k}k!k!} = \sum_{k=1}^{\infty} \frac{1}{2^{2k-1}(k-1)!k!} = \sum_{k=0}^{\infty} \frac{1}{2^{2k+1}k!(k+1)!} = I_1(1)\\
S(2) &= \sum_{k=0}^{\infty} \frac{2^2k^2}{2^{2k}k!k!} = \sum_{k=1}^{\infty} \frac{1}{2^{2k-2}(k-1)!(k-1)!} = \sum_{k=0}^{\infty} \frac{1}{2^{2k}k!k!} = I_0(1)\\
\end{aligned}
\]

Induction hypothesis: for every \(m \leq n\), it holds that
\(S(m) = v_m I_0(1) + u_m I_1(1)\). Then:

\begin{equation}\phantomsection\label{eq-2bell-dobinski-proof-step1}{
\begin{aligned}
S(n+2) &= \sum_{k=0}^{\infty} \frac{(2k)^{n+2}}{2^{2k}k!k!} = \sum_{k=1}^{\infty} \frac{(2k)^{n}}{2^{2k-2}(k-1)!(k-1)!} = \sum_{k=0}^{\infty} \frac{(2k+2)^{n}}{2^{2k}k!k!}\\
&= \sum_{k=0}^{\infty}\sum_{m=0}^{n} \binom{n}{m}2^{n-m}\frac{(2k)^m}{2^{2k}k!k!} = \sum_{m=0}^{n}\binom{n}{m}2^{n-m} \sum_{k=0}^{\infty}\frac{(2k)^m}{2^{2k}k!k!} \\
&= \sum_{m=0}^{n}\binom{n}{m}2^{n-m} S(m)
\end{aligned}
}\end{equation}

First, notice that Equation~\ref{eq-2bell-dobinski-proof-step1} has
exactly the same form as the recurrence relation for the 2-Bell numbers
as defined in Equation~\ref{eq-2bell-recurrence}. Second, use the
induction hypothesis and substitute \(S(m)\):

\[
\begin{aligned}
S(n+2) &= \sum_{m=0}^{n}\binom{n}{m}2^{n-m} (v_m I_0(1)+u_m I_1(1)) \\
&= I_0(1)\bigg[\sum_{m=0}^{n}\binom{n}{m}2^{n-m}v_m\bigg] + I_1(1)\bigg[\sum_{m=0}^{n}\binom{n}{m}2^{n-m}u_m\bigg] \\
&= v_{n+2}I_0(1)+u_{n+2}I_1(1)
\end{aligned}
\]

which completes the proof. Indeed, if we were to enumerate a few more
cases, we would see that \(v_n\) and \(u_n\) precisely match the
sequences A351143 and A351028.

\begin{longtable}[]{@{}lllll@{}}
\toprule\noalign{}
\(n\) & \(S(n)\) & \(v_n\) & \(u_n\) & \(v_n+u_n\) \\
\midrule\noalign{}
\endhead
\bottomrule\noalign{}
\endlastfoot
0 & \(I_0(1)\) & 1 & 0 & 1 \\
1 & \(I_1(1)\) & 0 & 1 & 1 \\
2 & \(I_0(1)\) & 1 & 0 & 1 \\
3 & \(2I_0(1)+I_1(1)\) & 2 & 1 & 3 \\
4 & \(5I_0(1)+4I_1(1)\) & 5 & 4 & 9 \\
5 & \(16I_0(1)+13I_1(1)\) & 16 & 13 & 27 \\
\end{longtable}

The major difference between Dobinski-like
Equation~\ref{eq-2bell-dobinski-proof-step1} and the standard Dobinski
formula for the Bell numbers is that unlike the exponential function,
modified Bessel functions do not have simple inverses and therefore we
cannot simply multiply the Equation~\ref{eq-2bell-dobinski-proof-step1}
by \(I_0(1)^{-1}\) to get rid of the Bessel functions. The recurrence
relation and the coefficients we just derived show that in principle the
e.g.f. for these sequences can be as simple as \(I_0(e^x)\), which is
very close to the e.g.f. for the Bell numbers \(\exp(-1)\exp(e^x)\). A
careful examination of the coefficients in the full series expansion of
the e.g.f.s in Theorem~\ref{thm-m2-egf} will reveal that the added
factors in the e.g.f. serve the same role as \(\exp(-1)\) in the Bell
numbers e.g.f. - due to Wronskian identity described in
Equation~\ref{eq-bessel-wronskian}, all Bessel functions get canceled
out and we remain with a pure power series in \(x\) without any special
functions.

\[
\begin{aligned}
I_0(e^x) &= \sum_{n=0}^{\infty} \frac{e^{2xn}}{2^{2n}n!n!} \\
K_0(e^x) &= (\log2-x)\sum_{n=0}^{\infty} \frac{e^{2xn}}{2^{2n}n!n!}+ \sum_{n=0}^{\infty}\frac{\psi(n+1)e^{2xn}}{2^{2n}n!n!} \\
&= \sum_{n=0}^{\infty} \frac{e^{2xn}}{2^{2n}n!n!}(\log2-x+\psi(n+1)) \\
\mathcal{A}_{351143}(x) &= K_1(1)\sum_{n=0}^{\infty} \frac{e^{2xn}}{2^{2n}n!n!}+I_1(1)\sum_{n=0}^{\infty} \frac{e^{2xn}}{2^{2n}n!n!}(\log2-x+\psi(n+1)) \\
&= \sum_{n=0}^{\infty} \frac{e^{2xn}}{2^{2n}n!n!}(K_1(1) + I_1(1) +\log2-x+\psi(n+1))
\end{aligned}
\]

\subsubsection{3.1. Stirling-like array}\label{stirling-like-array}

Define the following two-term recurrence relation where
\(\lfloor k/m\rfloor\) is the floor function:

\[
\genfrac{\lfloor}{\rfloor}{0pt}{}{n+1}{k}_m = m\lfloor k/m \rfloor \, \genfrac{\lfloor}{\rfloor}{0pt}{}{n}{k}_m + \genfrac{\lfloor}{\rfloor}{0pt}{}{n}{k-1}_m, \,\,\,\,\, n, k ≥ 1
\]

with initial conditions
\(\genfrac{\lfloor}{\rfloor}{0pt}{}{n}{k}_m = \delta_{n,k}\)

For \(m=1\) we obtain the standard Stirling numbers of the first kind.
Let us first consider the case for \(m=2\) in detail:

\[
\genfrac{\lfloor}{\rfloor}{0pt}{}{n+1}{k}_2 = 2\lfloor k/2 \rfloor \, \genfrac{\lfloor}{\rfloor}{0pt}{}{n}{k}_2 + \genfrac{\lfloor}{\rfloor}{0pt}{}{n}{k-1}_2, \,\,\,\,\, (n, k ≥ 1)\\
\]

This recurrence relation is almost identical to that for the Stirling
numbers of the second kind, except that the coefficient in front of
Z(n,k) has parity - it is always even, with k rounded down to the
nearest even integer.

The first few rows of this array are:

\begin{longtable}[]{@{}
  >{\raggedright\arraybackslash}p{(\linewidth - 18\tabcolsep) * \real{0.0794}}
  >{\raggedright\arraybackslash}p{(\linewidth - 18\tabcolsep) * \real{0.0794}}
  >{\raggedright\arraybackslash}p{(\linewidth - 18\tabcolsep) * \real{0.0794}}
  >{\raggedright\arraybackslash}p{(\linewidth - 18\tabcolsep) * \real{0.0794}}
  >{\raggedright\arraybackslash}p{(\linewidth - 18\tabcolsep) * \real{0.0794}}
  >{\raggedright\arraybackslash}p{(\linewidth - 18\tabcolsep) * \real{0.0794}}
  >{\raggedright\arraybackslash}p{(\linewidth - 18\tabcolsep) * \real{0.0794}}
  >{\raggedright\arraybackslash}p{(\linewidth - 18\tabcolsep) * \real{0.0794}}
  >{\raggedright\arraybackslash}p{(\linewidth - 18\tabcolsep) * \real{0.0794}}
  >{\raggedright\arraybackslash}p{(\linewidth - 18\tabcolsep) * \real{0.2857}}@{}}
\toprule\noalign{}
\begin{minipage}[b]{\linewidth}\raggedright
\end{minipage} & \begin{minipage}[b]{\linewidth}\raggedright
k=0
\end{minipage} & \begin{minipage}[b]{\linewidth}\raggedright
k=1
\end{minipage} & \begin{minipage}[b]{\linewidth}\raggedright
k=2
\end{minipage} & \begin{minipage}[b]{\linewidth}\raggedright
k=3
\end{minipage} & \begin{minipage}[b]{\linewidth}\raggedright
k=4
\end{minipage} & \begin{minipage}[b]{\linewidth}\raggedright
k=5
\end{minipage} & \begin{minipage}[b]{\linewidth}\raggedright
k=6
\end{minipage} & \begin{minipage}[b]{\linewidth}\raggedright
k=7
\end{minipage} & \begin{minipage}[b]{\linewidth}\raggedright
\(\sum_{k=0}^{n}\)
\end{minipage} \\
\midrule\noalign{}
\endhead
\bottomrule\noalign{}
\endlastfoot
n=0 & 1 & & & & & & & & 1 \\
n=1 & 0 & 1 & & & & & & & 1 \\
n=2 & 0 & 0 & 1 & & & & & & 1 \\
n=3 & 0 & 0 & 2 & 1 & & & & & 3 \\
n=4 & 0 & 0 & 4 & 4 & 1 & & & & 9 \\
n=5 & 0 & 0 & 8 & 12 & 8 & 1 & & & 29 \\
n=6 & 0 & 0 & 16 & 32 & 44 & 12 & 1 & & 105 \\
n=7 & 0 & 0 & 32 & 80 & 208 & 92 & 18 & 1 & 431 \\
\end{longtable}

The row sums of this array yield the sequence
\href{https://oeis.org/A007472}{A007472}

\[
z_n = \sum_{k=0}^{n}\genfrac{\lfloor}{\rfloor}{0pt}{}{n}{k}_2 = (1, 1, 1, 3, 9, 29, 105, 431, 1969, 9785, 52145, 296155, 1787385, ...)
\]

which as stated earlier has a property analogous to the Bell numbers: it
shifts by 2 places left after 2 applications of the binomial transform:

\[
\mathscr{z}_{n+2}=\sum_{k=0}^{n}\binom{n}{k}2^{n-k}z_k
\]

where we use the following identity that relates a double binomial
transform to a single transform with an extra factor of \(2^{n-k}\):

\[
\sum_{k=0}^{n}\binom{n}{k}\sum_{j=0}^{k}\binom{k}{j}a_j = \sum_{k=0}^{n}\binom{n}{k}2^{n-k}a_k
\]

\phantomsection\label{refs}
\begin{CSLReferences}{0}{1}
\bibitem[\citeproctext]{ref-bernstein1995}
\CSLLeftMargin{{[}1{]} }%
\CSLRightInline{Bernstein, M and Sloane, N J A (1995 ). Some canonical
sequences of integers. \emph{Linear Algebra and its Applications}.
\textbf{226-228} 57--72.
\url{https://linkinghub.elsevier.com/retrieve/pii/0024379594002459}}

\bibitem[\citeproctext]{ref-weisstein}
\CSLLeftMargin{{[}2{]} }%
\CSLRightInline{Weisstein, E W Bell Polynomial.
\url{https://mathworld.wolfram.com/BellPolynomial.html}}

\bibitem[\citeproctext]{ref-dattoliTouchardPolynomialsGeneralized2010}
\CSLLeftMargin{{[}3{]} }%
\CSLRightInline{Dattoli, G, Germano, B, Martinelli, M R and Ricci, P E
(2010 ). Touchard like polynomials and generalized {Stirling} numbers.
\url{http://arxiv.org/abs/1010.5934}}

\bibitem[\citeproctext]{ref-shmueliUsefulDistributionFitting2005}
\CSLLeftMargin{{[}4{]} }%
\CSLRightInline{Shmueli, G, Minka, T P, Kadane, J B, Borle, S and
Boatwright, P (2005 ). A {Useful Distribution} for {Fitting Discrete
Data}: {Revival} of the {Conway}--{Maxwell}--{Poisson Distribution}.
\emph{Journal of the Royal Statistical Society Series C: Applied
Statistics}. \textbf{54} 127--42.
\url{https://doi.org/10.1111/j.1467-9876.2005.00474.x}}

\bibitem[\citeproctext]{ref-comtet1974}
\CSLLeftMargin{{[}5{]} }%
\CSLRightInline{Comtet, L (1974 ). \emph{Advanced Combinatorics}.
Springer Netherlands, Dordrecht.
\url{http://link.springer.com/10.1007/978-94-010-2196-8}}

\bibitem[\citeproctext]{ref-suxe1ndor2004}
\CSLLeftMargin{{[}6{]} }%
\CSLRightInline{Sándor, J and Crstici, B (2004 ). Stirling, bell,
bernoulli, euler and eulerian numbers. Springer Netherlands, Dordrecht.
459--618. \url{https://doi.org/10.1007/1-4020-2547-5_5}}

\bibitem[\citeproctext]{ref-NIST:DLMF}
\CSLLeftMargin{{[}7{]} }%
\CSLRightInline{{\emph{NIST Digital Library of Mathematical Functions}}.
\url{https://dlmf.nist.gov/}, Release 1.2.4 of 2025-03-15.
\url{https://dlmf.nist.gov/}}

\end{CSLReferences}




\end{document}
