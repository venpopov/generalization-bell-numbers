\documentclass{article}
\usepackage{biblatex}
\usepackage{amsmath}
\usepackage{amsfonts}
\usepackage{hyperref}
\addbibresource{references.bib}
\bibliography{references}
\RequirePackage[a4paper, left=1.5in,right=1.5in,top=1in,bottom=1in]{geometry}
\newcommand{\floor}[1]{\left\lfloor #1 \right\rfloor}
\newcommand{\ceil}[1]{\left\lceil #1 \right\rceil}
\renewcommand{\theequation}{A\arabic{equation}}
\begin{document}
\section*{Appendix A: Z(n,k) formula derivation}
We have the recurrence
\begin{equation}\label{zet-recurrence}
Z(n,k) = 2\floor{k/2} Z(n-1,k)+Z(n-1,k-1),
\end{equation}
with initial conditions $Z(0,k) = \delta_{0,k}$, valid for $1 \leq k \leq n$, which forms a lower triangular array. 
Here I derive an explicit formula for $Z(n,k)$. Let $A_k$ be the ordinary generating function for the k-th column over $x^n$:
$$
A_k(x) = \sum_{n} Z(n,k) x^n
$$
We first multiply each side of the recurrence in Eq. \ref{zet-recurrence} by $x^n$ and sum over n:
$$
\begin{aligned}
\sum_{n \geq 1} Z(n,k)x^n &= 2\floor{k/2}\sum_{n \geq 1}Z(n-1,k)x^n + \sum_{n \geq 1} Z(n-1,k-1) x^n \\
A_k -Z(0,k)&=2\floor{k/2}x A_k+x A_{k-1}, \quad k \geq 1
\end{aligned}
$$
Since $Z(0,k) = 0$ when $k \neq 0$, then this simplifies to the following recurrence for the column generating function:
$$
A_k = \frac{x}{(1-2\floor{k/2}x)} A_{k-1}
$$
We know from the initial conditions that $A_0(x) = 1$, hence:
$$
\begin{aligned}
    A_0 &= 1 \\
    A_1 &= x \\
    A_2 &= \frac{x^2}{1-2x} \\
    A_3 &= \frac{x^3}{(1-2x)^2} \\
    A_4 &= \frac{x^4}{(1-2x)^2 (1-4x)} \\
    A_5 &= \frac{x^5}{(1-2x)^2 (1-4x)^2}, \\
\end{aligned}
$$
and so on. More generally, the ordinary generating function for the k-th column is:
\begin{equation}\label{column-ogf}
A_k(x) = \frac{x^k}{\prod_{j=1}^k(1-2\floor{j/2}x)}
\end{equation}
In order to get an explicit formula for $Z(n,k)$, use the technique on p.19 of \cite{wilfGeneratingfunctionologyThirdEdition2005}:
$$
Z(n,k) = [x^n]A(x),
$$
where $[x^n]$ is the coefficient extractor operator. Start with partial fraction decomposition of $A_k(x)$. First, denote:
$$
A_k(x) = x^kP_k(x),
$$
where
$$
P_k(x) = \prod_{1 \leq j \leq k}\frac{1}{1-2\floor{j/2}x}.
$$
Hence
\begin{equation}\label{znk-coef}
Z(n,k) = [x^n]x^kP_k(x) = [x^{n-k}]P_k(x)
\end{equation}
We can express $P_k(x)$ as a sum of partial fractions with unknown coefficients (note the slight difference in ranges for odd and even k):

\begin{equation}\label{pkx}
    \begin{aligned}
    P_{2m}(x) &= \sum_{1 \leq j \leq m} \frac{C_{2m,j}}{1-2jx} + \sum_{1 \leq j \leq m-1} \frac{D_{2m,j}}{(1-2jx)^2} \\
    P_{2m+1}(x) &= \sum_{1 \leq j \leq m} \frac{C_{2m+1,j}}{1-2jx} + \sum_{1 \leq j \leq m} \frac{D_{2m+1,j}}{(1-2jx)^2}
    \end{aligned}
\end{equation}

It is easy to calculate the $D$ coefficients with the standard Heaviside cover-up method:
$$
\begin{aligned}
    D_{k,r} &= (1-2rx)^2P_k(x)|_{x=\frac{1}{2r}} \\
    &= (1-2rx)^2 \prod_{1 \leq j \leq k} \frac{1}{1-2 \floor{j/2}x} \bigg|_{x=\frac{1}{2r}} \\
    &= \prod_{1 \leq j < 2r} \frac{1}{1-\floor{j/2}/r} \prod_{2r < j \leq k} \frac{1}{1- \floor{j/2}/r} \\
    &= \prod_{1 \leq j < 2r} \frac{r}{r-\floor{j/2}} \prod_{2r < j \leq k} \frac{r}{r- \floor{j/2}}, \\
\end{aligned}
$$
which simplifies to:
\begin{equation}\label{d-k-r}
        D_{k,r}=\frac{(-r)^{k-1}}{r!r!(\floor{\frac{k}{2}}-r)!(\floor{\frac{k-1}{2}}-r)!}.
\end{equation}
$D_{k,r}$ varies only slightly for even and odd k:
$$
\begin{aligned}
    D_{2m,r} &= \frac{-r^{2m-1}}{r!r!(m-r)!(m-1-r)!} \\
    D_{2m+1,r} &= \frac{r^{2m}}{r!r!(m-r)!(m-r)!} \\
\end{aligned}
$$
For the $C_{k,j}$ coefficients, we have two main cases to consider. When $k=2m$ is even and $j=m$, $P_k(x)$ has a simple pole at $1-2mx$, so we get $C_{2m,m}$ easily by the same method:
\begin{equation}\label{c-2m-m}
    C_{2m,m} = (1-2mx)P_{2m}(x)|_{x=\frac{1}{2m}} = \frac{m^{2m}}{m!m!}
\end{equation}
For the remaining C coefficients  we have a pole of $P_k(x)$ of multiplicity 2 at $1-2jx$ so we need to use the derivative method. Specifically:
$$
-2r C_{k,r} = \frac{\mathrm{d}}{\mathrm{d}x} (1-2rx)^2P_k(x)\bigg|_{x=\frac{1}{2r}}
$$
Let 
$$
\begin{aligned}
Q_r &= (1-2rx)^2 P_k(x) = \prod_{j} \frac{1}{(1-2jx)^{e_j}},\\
e_j &= 
    \begin{cases} 
    2,& \mathrm{if}\quad 1 \leq 2j < k \land  j \neq r \\ 
    1, & \mathrm{if} \quad 2j = k \land j \neq r \\
    0, & \mathrm{otherwise}
    \end{cases}
\end{aligned}
$$
Then $C_{k,r} = -\frac{1}{2r}Q'_r\left (\frac{1}{2r} \right)$ and we can use the logarithmic derivative to simplify the calculation:

$$ 
\begin{aligned}
    \frac{Q'_{r}}{Q_r} &= \frac{d}{dx}log(Q_r) = \frac{d}{dx}\left[ -\sum_{j} e_j\log(1-2jx) \right]=  \sum_{j} 2e_j\frac{j}{1-2jx}  \\
    C_{k,r} &= - \frac{1}{2r}Q'_r\left (\frac{1}{2r} \right) = -\log(Q_r(x))'Q_r(x)|_{x=1/{2r}} \\
    &= -\frac{1}{2r} Q_r\left(\frac{1}{2r}\right) 2\sum_{j} e_j\frac{j}{1-j/r} \\ 
    &= -\frac{1}{2r} 2 D_{k,r}\sum_{j}e_j\frac{rj }{r-j} \\
    &= - \frac{1}{2r}2r D_{k,r}\sum_{j}e_j \frac{j}{r-j} \\
    &= - D_{k,r}\sum_{j}e_j \frac{j}{r-j} \\
\end{aligned}
$$
Let $S_k=\sum_{j}e_j\frac{j}{r-j}$ and simplify S. First, consider the case $k=2m+1$. 
$$
\begin{aligned}
    S_{2m+1} &=  2\sum_{\substack{1 \leq j \leq m \\ j \neq r}}\frac{j}{r-j} \\ 
    &= 2\left[\frac{1}{r-1}+\frac{2}{r-2} + \cdots  + \frac{r-1}{1}-\frac{r+1}{1}-\frac{r+2}{2}-\cdots-\frac{m}{m-r}\right]\\
    &= 2\sum_{1 \leq j \leq r-1}\frac{r-j}{j} -2\sum_{1 \leq j \leq m-r}\frac{j+r}{j} \\
    &= 2r\left(\sum_{1 \leq j \leq r-1}\frac{1}{j}\right) - 2(r-1) - 2r\left(\sum_{1 \leq j \leq m-r}\frac{1}{j}\right)-2(m-r)\\
    &= 2r(H_{r-1}-H_{m-r})-2m+2 \\
\end{aligned}
$$
where $H_n$ is the $n^{th}$ harmonic number. We proceed similarly for $k=2m$ with some care about $e_{m}=1$:
$$
\begin{aligned}
    S_{2m} &= \sum_{j}e_j \frac{j}{r-j} = \frac{m}{r-m} + 2\sum_{\substack{1 \leq j \leq m-1 \\ j \neq r}}\frac{j}{r-j}\\ 
    &= \frac{m}{r-m} + 2 \sum_{1 \leq j \leq r-1} \frac{j}{r-j} - 2\sum_{r+1 \leq j \leq m-1}\frac{j}{j-r} \\
    &= \frac{m}{r-m} + 2\sum_{1 \leq j \leq r-1}\frac{r-j}{j} -2\sum_{1 \leq j \leq m-r-1}\frac{j+r}{j} \\
    &= \frac{m}{r-m} + 2rH_{r-1}-2(r-1)-2rH_{m-r-1}-2(m-r-1) \\
    &=  2rH_{r-1} - 2rH_{m-r-1}-2(m-2)-\frac{m}{m-r}  \\
    &= 2rH_{r-1} - 2rH_{m-r-1} - 2(m-2) - \frac{r}{m-r} -1 \\
    &= r\left(2H_{r-1} - H_{m-r-1}-H_{m-r}\right) - 2m + 3
\end{aligned}
$$
Generally when $r \neq m$:
\begin{equation}\label{s-k}
    S_k = r(2H_{r-1}-H_{\floor{k/2}-r}-H_{\floor{(k-1)/2}-r})-k+3
\end{equation}
Combining everything we have:
\begin{equation}\label{d-s-c}
    \begin{aligned}
        D_{k,r} &= \frac{(-r)^{k-1}}{r!r!(\floor{k/2}-r)!(\floor{(k-1)/2}-r)!} \\
        S_{k,r} &= r(2H_{r-1}-H_{\floor{k/2}-r}-H_{\floor{(k-1)/2}-r})-k+3 \\
        C_{k,r} &= 
        \begin{cases}
            - D_{k,r} S_{k,r} & \text{if}\, 1 \leq r < k/2\\
            \frac{r^{2r}}{r!r!} & \text{if}\, r = k/2\\
            0 & \text{otherwise}
        \end{cases} 
    \end{aligned}
\end{equation}
Next, let us find an explicit formula for Z(n,k). From \ref{znk-coef} and \ref{pkx}:
$$
\begin{aligned}
    Z(n,k) &=[x^{n-k}]P_k(x) \\
    &= [x^{n-k}]\left[ \sum_{1 \leq j \leq \floor{k/2}} \frac{C_{k,j}}{1-2jx} + \sum_{1 \leq j \leq \floor{(k-1)/2}} \frac{D_{k,j}}{(1-2jx)^2} \right]\\
     &= \sum_{1 \leq j \leq \floor{k/2}}C_{k,j}[x^{n-k}]  \frac{1}{1-2jx} + \sum_{1 \leq j \leq \floor{(k-1)/2}} D_{k,j}[x^{n-k}]\frac{1}{(1-2jx)^2} \\ 
\end{aligned}
$$
We have the standard geometric series expansions:
$$
\begin{aligned}
     & [x^{n-k}]\frac{1}{1-2jx} = [x^{n-k}]\sum_{i\geq0}(2j)^ix^i = (2j)^{n-k}\\ 
     & [x^{n-k}]\frac{1}{(1-2jx)^2} = [x^{n-k}]\sum_{i\geq0}(i+1)(2j)^ix^i = (n-k+1)(2j)^{n-k}.\\ 
\end{aligned}
$$
Which results in the following final set of formulas valid for $n \geq 2, k \geq 2$:
\begin{equation}\label{z-d-c}
    \begin{aligned}
        Z(n,k &)= \sum_{1 \leq j \leq \floor{\frac{k}{2}}}(2j)^{n-k}\left (C_{k,j} +D_{k,j}(n-k+1) \right )\\
            D_{k,r} &= \frac{(-r)^{k-1}}{r!r!(\floor{\frac{k}{2}}-r)!(\floor{\frac{k-1}{2}}-r)!} \\
            C_{k,r} &= \begin{cases}
                - D_{k,r} (r(2H_{r-1}-H_{\floor{\frac{k}{2}}-r}-H_{\floor{\frac{k-1}{2}}-r})-k+3) & \text{if} \, 1 \leq r < k/2\\
                \frac{r^{2r}}{r!r!} & \text{if}  \, r = k/2\\
                0 & \text{otherwise}
            \end{cases} 
    \end{aligned}
\end{equation}

This piecewise construction is necessary because the Harmonic numbers have a singularity at negative integer indices, which occurs when $r \geq k/2$ or $r\leq 0$. This problem can be avoided by replacing the Harmonic numbers with the identity $H(n) = \psi(n+1)+\gamma$ where $\gamma$ is the Euler-Mascheroni constant and $\psi(x)$ is the Digamma function, and by replacing the factorials with the Gamma function. Since we have digamma functions in the numerator and gamma functions in the denominator of the resulting expression, the limit of their ratio as the argument approaches non-positive integer indices is well defined. Specifically, using the residues of the digamma and gamma functions around the negative integer poles it can be easily shown that
$$
\lim_{x \to -n}\frac{\psi(x)}{\Gamma(x)} = (-1)^{n+1}n!, \quad n \in\mathbb{N}^{+}.
$$
After incorporating the coefficient definitions into the expression for $Z(n,k)$ (\ref{z-d-c}) and performing simplifications using properties of these special functions, we arrive at the following expression. 
$$
\begin{aligned}
    Z(n,k) =\\
    (-1)^{k-1} 2^{n-k} \sum_{j=0}^{\floor{k/2}} \lim_{t \to j} \left[t^{n-1}
        \frac
            {n-2-t\,\left(2\psi(t)-\psi\left(\floor{\frac{k}{2}}-t+1\right)-\psi\left(\floor{\frac{k-1}{2}}-t+1\right)\right)}
            {\Gamma(t+1) \Gamma(t+1) \Gamma(\floor{\frac{k}{2}}-t+1) \Gamma(\floor{\frac{k-1}{2}}-t+1)}\right]
\end{aligned}
$$
This formula provides a single, unified expression for $Z(n,k)$ that is valid for all integers $n,k \geq 0$. The inclusion of the $j=0$ term in the sum is essential, as the limit process correctly evaluates its contribution, ensuring validity for boundary cases like $Z(0,0)=1$, $ Z(1,1)=1$, and ensuring $Z(n,k)=0$ for $k>n$.

Here are explicit formulas for columns 2-5 and related OEIS sequences:
\[
    \begin{aligned}
        Z(n,2) &= 2^{n-2}, & \quad n \ge 2\, (\text{rel. } \href{https://oeis.org/A000079}{A000079})\\
        Z(n,3) &= (n-2)2^{n-3}, & \quad n \ge 3\, (\text{rel. } \href{https://oeis.org/A001787}{A001787})\\
        Z(n,4) &= 2^{2n-6}-(n-1)\,2^{n-4}, & n\geq 4 \, (\text{rel. } \href{https://oeis.org/A100575}{A100575})\\
        Z(n,5) &= n\,2^{n-5}+(n-6) 2^{2n-8}, & n\geq 5 \, (\text{rel. } \href{https://oeis.org/A158681}{A158681})\\
    \end{aligned}
\]

(with the caveat that the sequences listed at the end of each line are offset or scaled). The first subdiagonal gives \href{https://oeis.org/A007590}{A007590}.

\pagebreak
\section*{Appendix B: Bessel-Stirling numbers identities}

\begin{proposition}[Hockey-stick-like identity]
    \[
        \mStirling{n}{k}_2 = \sum_{j=k}^{n}(2\floor{k/2})^{n-j} \mStirling{j-1}{k-1}_2
    \]
\end{proposition}


\[
    \mStirling{n+m}{k}_2 = b^m \mStirling{n}{k}_2 +\sum_{j=0}^{m-1}b^{m-1-j} \mStirling{n+j}{k-1}_2
\]
where $b=2\floor{k/2}$

\printbibliography
\end{document}