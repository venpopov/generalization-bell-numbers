\documentclass{article}
\usepackage{biblatex}
\addbibresource{references.bib}
\usepackage{amsmath}
\usepackage[left=2.5cm, right=2cm]{geometry}
\newcommand{\floor}[1]{\left\lfloor #1 \right\rfloor}
\newcommand{\ceil}[1]{\left\lceil #1 \right\rceil}
\begin{document}

\title{Appendix A}
\maketitle
We have the recurrence
$$
Z(n,k) = 2\floor{k/2} Z(n-1,k)+Z(n-1,k-1),
$$
% which can be split into two cases for even and odd k:
% $$
% \begin{aligned}
% Z(n,2m) &= 2m Z(n-1,2m)+Z(n-1,2m-1) \\
% Z(n,2m+1) &= 2m Z(n-1,2m+1)+Z(n-1,2m)
% \end{aligned}
% $$
Let $A_k$ be the ordinary generating function over $x^n$:
$$
A_k(x) = \sum_{n} Z(n,k) x^n
$$
We multiply each side of the recurrence by $x^n$ and sum over n:
$$
\begin{aligned}
\sum_{n \geq 1} Z(n,k)x^n &= 2\floor{k/2}\sum_{n \geq 1}Z(n-1,k)x^n + \sum_{n \geq 1} Z(n-1,k-1) x^n \\
A_k -Z(0,k)&=2\floor{k/2}x A_k+x A_{k-1}, \quad k \geq 1
\end{aligned}
$$
Since $Z(0,k) = 0$ when $k \neq 0$, then this simplifies to
$$
A_k = \frac{x}{(1-2\floor{k/2}x)} A_{k-1}
$$
We know that $A_0(x) = 1$, hence:
$$
\begin{aligned}
    A_0 &= 1 \\
    A_1 &= x \\
    A_2 &= \frac{x^2}{1-2x} \\
    A_3 &= \frac{x^3}{(1-2x)^2} \\
    A_4 &= \frac{x^4}{(1-2x)^2 (1-4x)} \\
    A_5 &= \frac{x^5}{(1-2x)^2 (1-4x)^2}, \\
\end{aligned}
$$
and so on. More generally:
$$
A_k(x) = \frac{x^k}{\prod_{j=0}^k(1-2\floor{m/2}x)}
$$
In order to get an explicit formula for $Z(n,k)$ use the technique on p.19 of \cite{wilf_generatingfunctionology_2005}. Start with partial fraction decomposition of $A_k(x)$. But $A_k$ has too high a degree x in the numerator so denote:
$$
A_k(x) = x^kP_k(x),
$$
where
$$
P_k(x) = \prod_{1 \leq m \leq k}\frac{1}{1-2\floor{m/2}x}
$$
We can can express this as a sum of partial fractions with unknown coefficients:
$$
P_k(x) = \sum_{1 \leq j \leq k} \frac{c_{k,j}}{1-2jx} + \sum_{1 \leq j < k} \frac{d_{k,j}}{(1-2jx)^2}
$$
It is easy to calculate the $d$'s, which vary only slightly for even and odd k:
$$
\begin{aligned}
    d_{2m,j} &= -\frac{r^{2m-1}}{r!r!(m-r)!(m-1-r)!} \\
    d_{2m+1,j} &= \frac{r^{2m}}{r!r!(m-r)!(m-r)!} \\
\end{aligned}
$$





\end{document}