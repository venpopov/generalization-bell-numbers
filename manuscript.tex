\documentclass[a4paper]{amsart}
\usepackage{tikz}
\usetikzlibrary{quotes}
\usepackage{biblatex}
\usepackage{amsaddr}
\usepackage{hyperref}
\usepackage{mathrsfs} % Needed for \mathscr
\usepackage{array} % Needed for table column specifiers
\usepackage{amssymb} % for \varnothing
\usepackage{pdflscape}  % For PDF output (rotates the page in the PDF viewer)
\usepackage{logix}
\usepackage{tikz-cd}

% Custom commands
\newcommand{\Stirling}[0]{\genfrac\{\}{0pt}{}}
\newcommand{\Stirlingone}[0]{\genfrac[]{0pt}{}}
\newcommand{\Lah}[0]{\genfrac{\lfloor}{\rfloor}{0pt}{}}
\newcommand{\mStirling}[0]{\genfrac{\OpnDblBrac}{\ClsDblBrac}{0pt}{}}
\newcommand{\mStirlingone}[0]{\genfrac{\OpnBrktBar}{\ClsBrktBar}{0pt}{}}
\newcommand{\mLah}[0]{\genfrac{\OpnDblFloor}{\ClsDblFloor}{0pt}{}}
\newcommand{\uuline}[1]{\underline{\underline{#1}}}
\newcommand{\ooline}[1]{\overline{\overline{#1}}}



\newcommand{\floor}[1]{\left\lfloor #1 \right\rfloor}
\newcommand{\ceil}[1]{\left\lceil #1 \right\rceil}
\newcommand{\R}{\mathbf{R}} % The real numbers.
%\DeclareMathOperator{\dist}{dist} % The distance.

% Formatting
\RequirePackage[left=1.5in,right=1.5in,top=1in,bottom=1in]{geometry}
%\setlength{\parindent}{0mm}
\newtheorem{theorem}{Theorem}
\newtheorem{definition}[theorem]{Definition}
\newtheorem{proposition}[theorem]{Proposition}
\newtheorem{conjecture}[theorem]{Conjecture}
\newtheorem{lemma}[theorem]{Lemma}
\newtheorem{corollary}[theorem]{Corollary}
\newtheorem{remark}[theorem]{Remark}


%% Meta
\title[The theory of hypergeometric Stirling and Bell numbers]{The theory of hypergeometric Stirling and Bell numbers}
\author{Vencislav Popov}
\email{vencislav.popov@gmail.com}
\address{Department of Psychology, University of Zurich}
\thanks{Computational results and other supporting information is available on \href{https://github.com/venpopov/generalization-bell-numbers}{GitHub} and \href{https://www.wolframcloud.com/obj/vpopov0/Published/marray-recurrence-triangle.nb}{Wolfram Cloud}}
\date{}

% Biblio
\addbibresource{references.bib}
% \bibliography{references} % Removed as addbibresource is used with biblatex

\begin{document}
\begin{abstract}
I introduce a natural generalization of the Bell numbers, the $m$-Bell numbers $B^{(m)}_{n}$, characterized by invariance under m binomial transforms followed by an m-place left shift.  Their exponential generating functions satisfy $m$-th‑order ordinary differential equations whose solutions are hypergeometric functions, specializing to the exponential function when $m=1$ (classical Bell numbers) and to modified Bessel functions when $m=2$ (yielding "Bessel-Bell" numbers). Analogous to the Bell–Stirling correspondence, there exist \(m\)-Stirling triangular arrays arising from the two‑term recurrence  $S_{n+1,k}=m\!\lfloor k/m\rfloor\,S_{n,k}+S_{n,k-1},$ whose row sums reproduce $B^{(m)}_{n}$. These structures serve as conversion operators between polynomial bases, generalizing the role of classical Stirling numbers, and they admit combinatorial interpretations through parity‑constrained partitions. Finally, the $m-$Bell numbers are linked to moments of the Conway–Maxwell–Poisson distribution.

\bigskip
\noindent\textbf{Keywords.} Bell numbers, Stirling numbers, Conway–Maxwell–Poisson distribution, modified Bessel functions, hypergeometric functions
\end{abstract}

\maketitle

\section{Introduction}\label{sec-introduction}
\noindent Given a sequence of numbers $(a_n)_{n \geq 0} = (a_0, a_1, \ldots)$, the binomial transform maps it to a new sequence $(b_n)_{n \geq 0} = (b_0,b_1,...)$ as follows:
\begin{equation*}
    b_n = \sum_{k=0}^{n} \binom{n}{k} a_k.
\end{equation*}

Repeated applications of the binomial transform on the resulting sequence can be represented with a single sum (see \cite{spiveyKbinomialTransformsHankel2006}):
\begin{equation*}
    b_n = \sum_{k_1=0}^{n} \binom{n}{k_1}\sum_{k_2=0}^{k_1} \binom{k_1}{k_2} \dots \sum_{k_{m}=0}^{k_{m-1}} \binom{k_{m-1}}{k_m} a_{k_m} = \sum_{k=0}^{n} \binom{n}{k} m^{n-k} a_k,
\end{equation*}
where $m$ is an integer that represents the number of times the binomial transform has been applied. Bernstein and Sloane \cite{bernstein1995} studied a number of what they call "Eigen sequences" of various such transformations - sequences $(a_n)$ which when transformed one or more times shift by one or more places but are otherwise preserved. Such sequences show a "self-similarity" under an iterated transform and understanding why this self-similarity occurs often reveals new properties or relations between different integer sequences and the combinatorial structures they enumerate.

Perhaps the most famous case of a binomial-transform invariant sequence is that of the Bell numbers, entry \href{https://oeis.org/A000110}{A000110} in the Online Encyclopedia of Integer Sequences (OEIS): $1, 1, 2, 5, 15, 52, 203, 877, 4140, \ldots$ Applying the binomial transformation to the Bell numbers produces an identical sequence, with the first element omitted and each subsequent element's index shifted once to the left:
\begin{equation}\label{eq-bell-recurrence}
B_{n+1} = \sum_{k=0}^{n} \binom{n}{k} B_k
\end{equation}

The Bell numbers count the total number of partitions of an n-element set and they are part of a rich combinatorial structure that involves the Stirling numbers the of the second kind (\href{https://oeis.org/A008277}{A008277}), Touchard polynomials \cite{weisstein}, the exponential function and linear operators acting on it \cite{dattoliTouchardPolynomialsGeneralized2010}.

To simplify the notation, following Bernstein and Sloane \cite{bernstein1995}, define the following operators on sequences:
\[
\begin{aligned}
\mathrm{BINOM} \circ [a_0,a_1,a_2, \dots] &= \bigg[\binom{0}{0}a_0,\binom{1}{0}a_0+\binom{1}{1}a_1, \ldots, \sum_{k=0}^n\binom{n}{k}a_k, \ldots \bigg] \\
L \circ [a_0,a_1,a_2, \ldots] &= [a_1,a_2, \ldots]
\end{aligned}
\]
where $\mathrm{BINOM}$ is the binomial transform operator applied to sequence $a$, whereas $L$ is the left-shift operator. Then the Bell numbers form the unique sequence that satisfies the following equality with initial condition $a_0 = 1:$
\[
\mathrm{BINOM} \circ a = \mathrm{L} \circ a
\]

Consider a new generalization of the Bell numbers defined by the following property:
\begin{equation}\label{eq-mbell-recurrence}
B^{(m)}_{n+m} = \sum_{k=0}^n \binom{n}{k} m^{n-k} B_k^{(m)}
\end{equation}
where the upper index $(m)$ has no algebraic meaning and should be read merely as an "m-Bell" number. Equivalently, these sequences satisfy the operator equation:
\begin{equation}\label{eq-mbell-operator}
\mathrm{BINOM^m} \circ a = \mathrm{L}^m \circ a
\end{equation}

These are sequences that shift to the left by $m$ places after $m$ applications of the binomial transform. The case $m=1$ corresponds to the Bell numbers, and $m=2$ corresponds to sequences \href{https://oeis.org/A007472}{A007472}, \href{https://oeis.org/search?q=1,0,1,2,5,16&language=english&go=Search}{A351143} and \href{https://oeis.org/A351028}{A351028}, which shift by 2 places left after 2 binomial transformations (sequences for $m>2$ are not currently present in OEIS). Although the sequences for $m=2$ are listed in OEIS, little is known about their properties.

In this paper I show that the m-fold binomial–shift-invariance property that characterizes these sequences arises from a combinatorial structure that mirrors that of the regular Bell numbers (Section \ref{sec-background}). The $m$-Bell numbers have exponential generating functions (e.g.f.) that are the solutions of ordinary differential equations of order $m$. The solutions to these equations are a class of hypergeometric functions, which reduce to the exponential function for $m=1$ (the 1-Bell numbers) and modified Bessel functions of the first and second kinds for $m=2$ (motivating the name Bessel-Bell or 2-Bell numbers; Section \ref{sec-m2}; Theorem \ref{thm-m2-egf}). These generalized Bell numbers admit Dobiński-like formula representations (Theorem \ref{thm-bessel-dobiski}). Each such \textit{m-Bell} sequence corresponds to the row sums of new Stirling-like arrays (Section \ref{sec-2-stirling}), motivating the name \textit{m-Stirling }numbers, which, as we will see, come in dual pairs just like the regular Stirling numbers of the first and second kind. Each of the associated $m$-Stirling arrays arises as the coefficients of polynomials that result from the application of the exponential shift operator to the hypergeometric e.g.f's. The $m$-Stirling and $m$-Bell numbers count parity-constrained partitions, resulting in a novel general combinatorial structure that greatly expands the Bell-Stirling-Touchard story (Section \ref{sec-combinatorial}). Finally, just like their standard counterparts are related to moments of the Poisson distribution, the $m$-Bell numbers turn out to be related to the moments of a generalized Poisson distribution known as the Conway-Maxwell-Poisson distribution \cite{shmueliUsefulDistributionFitting2005} (Section \ref{sec-cmp}).

To clearly ground the analogue between the Bell-Stirling-Touchard framework and the novel results, I begin with a review of standard results and notation \cite{comtet1974, sándor2004}.

\section{Background}\label{sec-background}
\noindent The exponential generating function (e.g.f.) $\mathcal{A}(x)$ of a sequence $(a_n)_{n \geq 0}$ is a formal power series in $x$:
\[
\mathcal{A}(x) = \sum_{n=0}^\infty a_n \frac{x^n}{n!}
\]

\begin{proposition}[Functional equation for the e.g.f. of $m$-Bell numbers]\label{prop:mbell-egf}
Let $\mathcal{A}(x)$ be the exponential generating function of a sequence \( a = (a_n) \) satisfying \eqref{eq-mbell-operator}.
Then \( \mathcal{A}(x) \) satisfies the $m$-th order ODE:
\[
\left(\frac{\mathrm{d}}{\mathrm{d}x}\right)^m \mathcal{A}(x) - e^{mx} \mathcal{A}(x) = 0.
\]
\end{proposition}

\begin{proof}
It is well known that both the binomial transform and the left-shift operator acting on $a$ have simple effects on the e.g.f., $\mathcal{B}(x)$, of the resulting sequence $b$ (e.g. \cite{bernstein1995}):
\begin{equation}\label{eq-egf-transform}
\begin{aligned}
b = \mathrm{BINOM} \circ a \quad &\iff\quad \mathcal{B}(x) = e^x \mathcal{A}(x) \\
b = \mathrm{L} \circ a \quad &\iff\quad \mathcal{B}(x) = \mathcal{A}'(x)
\end{aligned}
\end{equation}
Applying $\mathrm{BINOM}$ $m$ times corresponds to multiplying $\mathcal{A}(x)$ by $e^{mx}$. Applying $\mathrm{L}$ $m$ times corresponds to taking the $m$-th derivative. Therefore, if \(\mathrm{BINOM}^m \circ a = \mathrm{L}^m \circ a\), then:
\[
\mathcal{A}^{(m)}(x) = e^{mx} \mathcal{A}(x).
\]
Rewriting this gives the result.
\end{proof}

The solution of the ODE for $m=1$ gives the well known offset double exponential generating function for the Bell numbers (with $x = e^t$):
\begin{equation}\label{eq-bell-egf}
    e^{e^t-1} = \sum_{n=0}^{\infty} B_n \frac{t^n}{n!}.
\end{equation}

Recall that the Bell numbers are the row sums of a triangle array formed by the Stirling numbers of the second kind:
\begin{align*}
B_n = \sum_{k=0}^{n}\genfrac\{\}{0pt}{}{n}{k},
\end{align*}
where the Stirling numbers of the second kind satisfy the two-term recurrence:
\[
\genfrac\{\}{0pt}{}{n+1}{k} = k \, \genfrac\{\}{0pt}{}{n}{k} + \genfrac\{\}{0pt}{}{n}{k-1}, \,\,\,\,\, n, k \ge 1\\.
\]

The Stirling numbers of the second kind count the number of ways to partition $n$ labeled objects into $k$ unlabeled subsets and are the coefficients of the Touchard (also known as Bell or exponential) polynomials:
\begin{equation}\label{eq-touchard-def}
    T_n(x) = \sum_{k=0}^{n} \genfrac\{\}{0pt}{}{n}{k} x^k,
\end{equation}

whose e.g.f. is a bi-variate generalization of the Bell numbers' e.g.f.:
\begin{equation}\label{eq-touchard-egf}
    \sum_{n=0}^{\infty} T_n(x) \frac{t^n}{n!} = e^{x(e^t-1)}.
\end{equation}
Many useful identities involving Touchard polynomials and Stirling numbers can be shown via the action of the exponential scaling operator:
\begin{equation}\label{eq-scaling-operator}
    e^{txD_x}f(x)=f(xe^t)
\end{equation}

where $D_x$ is the derivative operator with respect to $x$. Specifically, by applying this operator to the standard exponential function $e^x$ we get precisely the e.g.f. for Touchard polynomials \cite{dattoliTouchardPolynomialsGeneralized2010}:
\[
e^{-x}e^{txD_x}e^x = e^{x(e^t-1)}
\]

A final useful relation is the celebrated Dobiński formula\cite{wilfGeneratingfunctionologyThirdEdition2005} that lets us express $T_n(x)$ and the Bell numbers $B_n = T_n(1)$ as an infinite sum:
\begin{equation}\label{eq-bell-dobinski}
\begin{aligned}
T_n(x) & =e^{-x}\sum_{k=0}^{\infty}\frac{x^k k^n}{k!} \\
B_n & = \frac{1}{e}\sum_{k=0}^{\infty}\frac{k^n}{k!}
\end{aligned}
\end{equation}

In the remainder of the paper we will see that the sequences for $m$-Bell numbers have many analogous properties as those defined above, starting with the special case $m = 2$.

\section{The Bessel-Bell numbers ($m=2$)}\label{sec-m2}
\subsection{Deriving the e.g.f.}
\noindent Before solving the general case, let us focus on the $m=2$ generalization of the Bell numbers, which satisfy the property that they shift by 2 places left after two binomial transformations:
\begin{equation}\label{eq-2bell-recurrence}
B^{(2)}_{n+2} = \sum_{k=0}^n \binom{n}{k} 2^{n-k} B_k^{(2)}
\end{equation}
The following three sequences listed in OEIS satisfy this property:
\begin{itemize}
    \item \href{https://oeis.org/A007472}{A007472}: $1, 1, 1, 3, 9, 29, 105, 431, 1969, \ldots$
    \item \href{https://oeis.org/A351143}{A351143}: $1, 0, 1, 2, 5, 16, 61, 258, 1177, \ldots$
    \item \href{https://oeis.org/A351028}{A351028}: $0, 1, 0, 1, 4, 13, 44, 173, 792, 4009, \ldots$
\end{itemize}

The main difference between the three sequences are the initial conditions $(a_0, a_1)$: $(1, 1); (1,0); (0,1)$. We will see in Section \ref{sec-general-case} that for any $m$ there are $m$ \textbf{primitive} solutions, whose initial conditions are a sequence of $m$ values, exactly one of which is 1 and the rest are 0. In the $m=2$ case the primitive solutions are A351143 and A351028, whose element-wise sum produces the sequence A007472. Owing to the form of their e.g.f.s, I will refer to the composite sequence A007472 as the Bessel-Bell numbers (not to be confused with Bessel numbers \cite{cheonGeneralizedBesselNumbers2013a}).

\begin{theorem}\label{thm-m2-egf}
The exponential generating functions for 2-Bell numbers are solutions to the modified Bessel ODE - a linear combination of modified Bessel functions of the first ($I_0$) and second kind ($K_0$) of order 0, whose weights are uniquely determined by the first two elements of the corresponding sequence. Specifically:
\[
\mathcal{A}(x) = p\,I_0(e^x)+q\,K_0(e^x)
\]

\begin{center}
\begin{tabular}{lrr}
\hline
Sequence & Coefficient \(p\) & Coefficient \(q\) \\
\hline
351143 & \(K_1(1) \approx 0.601907\) & \(I_1(1) \approx 0.565159\) \\
351008 & \(K_0(1) \approx 0.421024\) & \(-I_0(1) \approx 1.266066\) \\
007472 & \(K_0(1) + K_1(1)\approx 1.022932\) & \(I_1(1) - I_0(1)\approx -0.700907\) \\
\hline
\end{tabular}
\end{center}
\smallskip
\end{theorem}

\begin{proof}
By Proposition \ref{prop:mbell-egf}, the e.g.f. $\mathcal{A}(x)$ satisfies the linear ODE
\[
\mathcal{A}''(x)-e^{2x}\,\mathcal{A}(x)=0.
\]
Set $t=e^{x}$ (so $\mathrm{d}t/\mathrm{d}x=t$) and define $f(t)=\mathcal{A}(x)$.  By the chain rule,
\[
\begin{aligned}
\mathcal{A}'(x) &= t\,f'(t),\\
\mathcal{A}''(x)&= t\,f'(t)+t^{2}f''(t),\\
e^{2x}\,\mathcal{A}(x) &= t^{2}f(t).
\end{aligned}
\]
Substituting in yields the standard form of the modified Bessel equation for the case of $n=0$ (see \href{https://dlmf.nist.gov/10.25.E1}{10.25.1}) \cite{NIST:DLMF}
\begin{equation}\label{eq-m2-ode}
t^{2}f''+t\,f'-t^{2}f=0,
\end{equation}
whose general solution is:
\[
f(t)=p\,I_0(t)+q\,K_0(t).
\]
with $I_n$ and $K_n$ being the modified Bessel functions of the first and second kind, and $p$ and $q$ are constants determined by the initial conditions. Therefore, we obtain the general solution for $\mathcal{A}(x)$, the e.g.f. of sequences \href{https://oeis.org/A007472}{A007472}, \href{https://oeis.org/A351143}{A351143} and \href{https://oeis.org/A351028}{A351028}, by substituting back $t=e^x$:
\begin{equation}\label{eq-m2-ODE-solution}
\mathcal{A}(x)=p\, I_o(e^x) + q\,K_0(e^x)
\end{equation}
To determine $p$ and $q$ we use the known initial conditions. Sequence A007472 is the element-wise sum of A351143 and A351028, so it is sufficient to determine $p$ and $q$ for the latter 2 sequences only. For A351143 we have $a_0 = 1$ and $a_1 = 0$. Evaluating the Maclaurin series for $\mathcal{A}(x)$ yields the following system of equations:
\[
\begin{aligned}
\mathcal{A}(0) & = p\, I_0(1) + q\,K_0(1) = 1 \\
\mathcal{A}'(x)\bigr|_{x=0} &=p\, I_0'(e^x)\bigr|_{x=0} + q\,K_0'(e^x)\bigr|_{x=0} = 0
\end{aligned}
\]
where $f'(x)\bigr|_{x=0}$ is the first derivative of $f$ evaluated at 0. The derivative of $I_0$ and $K_0$ are thankfully straightforward (\href{https://dlmf.nist.gov/10.29.E3}{10.29.3}) \cite{NIST:DLMF}:
\begin{equation}\label{eq-bessel-deriv1}
\begin{aligned}
I_0'(x) &= I_1(x) \\ 
K_0'(x) &= -K_1(x). \\
\end{aligned}
\end{equation}
By the chain rule we get:
\begin{equation*}
    \begin{aligned}
        I_0'\bigl(e^x\bigr) &= e^xI_1\bigl(e^x\bigr) \\
        K_0'\bigl(e^x\bigr) &= -e^xK_1\bigl(e^x\bigr) \\
    \end{aligned}
\end{equation*}
Evaluating at $x=0$ gives:
\[
\begin{aligned}
A(0) & = p \,I_0(1)+q\,K_0(1) = 1 \\
A'(0) & = p\,I_1(1)-q K_1(1) = 0
\end{aligned}
\]
After some standard algebraic torture we get the following expressions for p and q:
\[
\begin{aligned}
p &= \frac{1-qK_0(1)}{I_0(1)} \\
q &= \frac{I_1(1)}{I_1(1)K_0(1)+I_0(1)K_1(1)}
\end{aligned}
\]
These expressions can be simplified further due to the following Bessel identity concerning the Wronskian of the modified Bessel functions (see \href{https://dlmf.nist.gov/10.28.E2}{10.28.2}):
\begin{equation}\label{eq-bessel-wronskian-general}
    I_v(z)K_{v+1}(z)+I_{v+1}(z)K_v(z) = 1/z
\end{equation}
which holds for any complex $v$ and $z$. In the special case when $v = 0$ and $z=1$:
\begin{equation}\label{eq-bessel-wronskian}
I_0(1)K_1(1)+I_1(1)K_0(1) = 1
\end{equation}
Therefore the constants for the e.g.f. of sequence A351143 are:
\[
\begin{aligned}
q &= I_1(1) \\
p &= \frac{1 - I_1(1)K_0(1)}{I_0(1)} = \frac{I_0(1)K_1(1)+I_1(1)K_0(1) - I_1(1)K_0(1)}{I_0(1)} = K_1(1)
\end{aligned}
\]
With a very similar manipulation for the initial conditions of A351028 and A007472 we get the coefficients in Theorem~\ref{thm-m2-egf} which concludes the proof.
\end{proof}

\subsection{Why do the e.g.f.s of the 2-Bell numbers produce integer coefficients?}\label{2-bell-integer-coefs}
The e.g.f.s of the 2-Bell numbers are admittedly more complicated than the e.g.f. $e^{e^x-1}$ of the Bell numbers \eqref{eq-bell-egf} (although the form is quite similar, as I discuss at the end of this section). How is it that such relatively complicated expression, involving four different special functions, produce integer coefficients? Let us explore the series expansion for the derived e.g.f.s. Consider first the case of sequence A351143:
\[
\mathcal{A}_{351143}(x) = K_1(1)I_0(e^x) + I_1(1)K_0(e^x)
\]
We have the following standard series for $I_0$ and $K_0$ (see \href{https://dlmf.nist.gov/10.25.E2}{10.25.2} and \href{https://dlmf.nist.gov/10.31.E2}{10.31.2}) \cite{NIST:DLMF}:
\begin{equation}\label{eq-i-k-0-series}
    \begin{aligned}
        I_0(z) &= \sum_{k=0}^{\infty} \frac{(\frac{1}{2}z)^{2k}}{k!k!} \\
        K_0(z) &= -\log\biggl(\frac{z}{2}\biggr)I_0(z) + \sum_{k=0}^{\infty}\frac{\psi(k+1)(\frac{1}{2}z)^{2k}}{k!k!}
    \end{aligned}
\end{equation}
where $\psi$ is the digamma function. Let us focus on the $I_0$ function. Substituting $z=e^x$ and then expanding the Taylor series for the exponential function we get:
\begin{equation}\label{eq-besseli-exp-maclaurin}
\begin{aligned}
I_0(e^x) &= \sum_{k=0}^{\infty} \frac{e^{2xk}}{2^{2k}k!k!} = \sum_{k=0}^{\infty} \frac{1}{2^{2k}k!k!}\sum_{n=0}^{\infty}\frac{x^n(2k)^n}{n!} = \sum_{n=0}^{\infty}\frac{x^n}{n!}\sum_{k=0}^{\infty} \frac{(2k)^n}{2^{2k}k!k!}
\end{aligned}
\end{equation}
Let the inner sum be represented by $S(n) = \sum_{k=0}^{\infty} \frac{(2k)^n}{2^{2k}k!k!}$. This formula is reminiscent of the Dobiński formula for the standard Bell numbers \eqref{eq-bell-dobinski} with an extra factorial in the denominator and extra powers of 2. Indeed, we can state an equivalent theorem:

\begin{theorem}[Dobiński-like formula for the Bessel-Bell numbers]\label{thm-bessel-dobiski}
\[
\begin{aligned}
I_0(e^x) &= \sum_{n=0}^\infty S(n) \frac{x^n}{n!} \\
S(n) &= \sum_{k=0}^{\infty} \frac{(2k)^n}{(2^{k}k!)^2}=v_n I_0(1) + u_n I_1(1) \\
A351143(n) &= v_n \\
A351028(n) &= u_n \\
A007472(n) &= v_n+u_n
\end{aligned}
\]
\end{theorem}

\begin{proof}
Proceed by induction. First, establish the base cases:
\[
\begin{aligned}
S(0) &= \sum_{k=0}^{\infty} \frac{1}{2^{2k}k!k!} = I_0(1) \\
S(1) &= \sum_{k=0}^{\infty} \frac{2k}{2^{2k}k!k!} = \sum_{k=1}^{\infty} \frac{1}{2^{2k-1}(k-1)!k!} = \sum_{k=0}^{\infty} \frac{1}{2^{2k+1}k!(k+1)!} = I_1(1)\\
S(2) &= \sum_{k=0}^{\infty} \frac{2^2k^2}{2^{2k}k!k!} = \sum_{k=1}^{\infty} \frac{1}{2^{2k-2}(k-1)!(k-1)!} = \sum_{k=0}^{\infty} \frac{1}{2^{2k}k!k!} = I_0(1)\\
\end{aligned}
\]
where in the case $S(1)$ we used the expansion for a general order modified Bessel function:
\[
I_v = \sum_{k=0}^\infty \frac{(x/2)^{2v+1}}{\Gamma(k+1)\Gamma(k+v+1)}
\]
Induction hypothesis: for every $m \leq n$, it holds that $S(m) = v_m I_0(1) + u_m I_1(1)$. Then:
\begin{equation}\label{eq-2bell-dobinski-proof-step1}
\begin{aligned}
S(n+2) &= \sum_{k=0}^{\infty} \frac{(2k)^{n+2}}{2^{2k}k!k!} = \sum_{k=1}^{\infty} \frac{(2k)^{n}}{2^{2k-2}(k-1)!(k-1)!} = \sum_{k=0}^{\infty} \frac{(2k+2)^{n}}{2^{2k}k!k!}\\
&= \sum_{k=0}^{\infty}\sum_{m=0}^{n} \binom{n}{m}2^{n-m}\frac{(2k)^m}{2^{2k}k!k!} = \sum_{m=0}^{n}\binom{n}{m}2^{n-m} \sum_{k=0}^{\infty}\frac{(2k)^m}{2^{2k}k!k!} \\
&= \sum_{m=0}^{n}\binom{n}{m}2^{n-m} S(m)
\end{aligned}
\end{equation}
First, notice that \eqref{eq-2bell-dobinski-proof-step1} has exactly the same form as the recurrence relation for the 2-Bell numbers as defined in \eqref{eq-2bell-recurrence}. Second, use the induction hypothesis and substitute $S(m)$:
\[
\begin{aligned}
S(n+2) &= \sum_{m=0}^{n}\binom{n}{m}2^{n-m} (v_m I_0(1)+u_m I_1(1)) \\
&= I_0(1)\bigg[\sum_{m=0}^{n}\binom{n}{m}2^{n-m}v_m\bigg] + I_1(1)\bigg[\sum_{m=0}^{n}\binom{n}{m}2^{n-m}u_m\bigg] \\
&= v_{n+2}I_0(1)+u_{n+2}I_1(1)
\end{aligned}
\]
which completes the proof. Indeed, if we were to enumerate a few more cases, we would see that $v_n$ and $u_n$ precisely match the sequences A351143 and A351028 (Table \ref{tab-vn-un-coefs}).

\begin{table}[]
    \centering
        \begin{tabular}{|c|l|c|c|c|}
        \hline
        $n$ & $S(n)$                & $v_n$ & $u_n$ & $v_n+u_n$ \\
        \hline
        0   & $I_0(1)$              & 1     & 0     & 1         \\
        1   & $I_1(1)$              & 0     & 1     & 1         \\
        2   & $I_0(1)$              & 1     & 0     & 1         \\
        3   & $2I_0(1)+I_1(1)$    & 2     & 1     & 3         \\
        4   & $5I_0(1)+4I_1(1)$   & 5     & 4     & 9         \\
        5   & $16I_0(1)+13I_1(1)$ & 16    & 13    & 29        \\
        6   & $61I_0(1)+44I_1(1)$ & 61    & 44 &    105 \\ 
        \hline
        \end{tabular}
        \bigskip
    \caption{Coefficients in the expansion of the Dobiński-like formula for the Bessel-Bell numbers \ref{thm-bessel-dobiski}}
    \label{tab-vn-un-coefs}
\end{table}
\end{proof}

The major difference between Dobiński-like Equation \eqref{eq-2bell-dobinski-proof-step1} and the standard Dobiński formula for the Bell numbers is that there are two different modified Bessel functions acting as \textbf{carriers} for the coefficients of the two primitive Bessel-Bell sequences, rather than a single exponential function. In the standard Dobiński formula one multiplies the infinite sum by $1/e$ to cancel the exponential carrier function, which here is not possible. The concept of a carrier function will play a fundamental role in understanding how $m$-Bell and $m$-Stirling numbers are generated, so we should make it explicit.

\begin{definition}[Carrier functions]
Let $\mathcal{L}$ be a linear operator on a function space $\mathcal{F}$ over a field $\mathbb{K}$. A set of functions $\{f_1, f_2, \ldots, f_m\} \subset \mathcal{F}$ is called a set of \textit{carrier functions} for $\mathcal{L}$ if:

(i) For each $n \in \mathbb{N}$, the function $\mathcal{L}^n f_1$ can be expressed as a linear combination of the form $\sum_{j=1}^m P_{n,j}(x) f_j(x)$, where each $P_{n,j}$ is a polynomial in $x$ with coefficients in $\mathbb{K}$;

(ii) The coefficients of these polynomials form sequences of combinatorial significance.

The function $f_1$ is called the \textit{primary carrier function} of the system.
\end{definition}

For example, $e^x$ is the primary carrier function for the operator $\mathcal{L} = x\frac{d}{dx}$, as $\mathcal{L}^n e^x$ produces polynomials whose coefficients are the Stirling numbers of the second kind. Similarly, $I_0(x)$ and $I_1(x)$ form a set of carrier functions for the same operator, generating polynomials whose coefficients are the Bessel-Stirling numbers of the second kind.

The recurrence relation and the coefficients just derived show that in principle the e.g.f. for these sequences can be as simple as $I_0(e^x)$, which is very close to the e.g.f. for the Bell numbers $\exp(-1)\exp(e^x)$. A careful examination of the coefficients in the full series expansion of the e.g.f.s in Theorem~\ref{thm-m2-egf} will reveal that the added factors in the e.g.f. serve the same role as $\exp(-1)$ in the Bell numbers e.g.f. - due to Wronskian identity described in Equation \eqref{eq-bessel-wronskian}, all Bessel functions get canceled out and we remain with a pure power series in $x$ without any special functions. 

Due to the complicated series expansion of $K_0$ shown in Equation \eqref{eq-i-k-0-series}, this cancellation property is easier to see if we use the Maclaurin series for $I$ and $K$. Let $f(x) = I_0(e^x)$ and $g(x) = K_0(e^x)$. Then
\[
\begin{aligned}
f(x) &= \sum_{n=0}^\infty \frac{x^n}{n!} f^{(n)}(x)|_{x=0} \\
g(x) &= \sum_{n=0}^\infty \frac{x^n}{n!} g^{(n)}(x)|_{x=0}
\end{aligned}
\]
By comparing the coefficients of $f(x)$ with \eqref{eq-besseli-exp-maclaurin}, we know that $f^{(n)}(x)|_{x=0} = S(n) = v_n I_0(1)+u_nI_1(1)$ where $v_n$ and $u_n$ are as defined in Theorem~\ref{thm-bessel-dobiski}. What about the coefficients in $g(x)$? They are derivatives of $K_0(e^x)$, and since derivatives of $K$ are very similar to those of $I$ (Equation \ref{eq-bessel-deriv1}), we will end up with similar coefficients up to a sign change. Specifically we have the partner theorem to Theorem~\ref{thm-bessel-dobiski}:

\begin{theorem}\label{thm-besselk-dobinski}
\[
K_0(e^x) = \sum_{n=0}^\infty \frac{x^n}{n!}(v_n K_0(1) - u_n K_1(1))
\]
\end{theorem}

\begin{proof}
We need to prove that $K_0^{(n)}(e^x)|_{x=0} = v_n K_0(1)-u_n K_1(1)$. First, note that the derivatives of I and K have the same form for all orders, except for order 0, where the derivative of K has a negative sign (\href{https://dlmf.nist.gov/10.29.E3}{10.29.3}) \cite{NIST:DLMF}:
\[
\begin{aligned}
I_0'(e^x) &= I_1(e^x) \\
K_0'(e^x) &= -K_1(e^x) \\
I_1'(e^x) &= e^x I_0(e^x) + I_1(e^x) \\
K_1'(e^x) &= e^x K_0(e^x) + K_1(e^x)
\end{aligned}
\]
This means that the n-th derivative of a $K$ function will have the same form as the n-th derivative of an $I$ function, except that the coefficient in front of $K_1(1)$ will be the negated coefficient of $I_1(1)$. Since $I_0'(e^x)|_{x=0} = v_n I_0(1) + u_n I_1(1)$ it follows that $K_0'(e^x)|_{x=0} = v_n K_0(1)-u_n K_1(1)$.
\end{proof}

Finally, by combining Theorem~\ref{thm-bessel-dobiski} and Theorem~\ref{thm-besselk-dobinski}, we have:
\[
\begin{aligned}
p I_0(e^x) + q K_0(e^x) = \sum_{n=0}^\infty \frac{x^n}{n!} \bigg[ v_n (p I_0(1) + q K_0(1)) + u_n (p I_1(1) - q K_1(1))\bigg]
\end{aligned}
\]
Within this equation, an appropriate choice of $p$ and $q$ will neutralize the modified Bessel functions in the series due to the Wronskian property in Equation \eqref{eq-bessel-wronskian}. Specifically, $p = K_1(1), q = I_1(1)$ makes the coefficient of $v_n$ 1 and the coefficient of $u_n$ 0 and gives us integer coefficients for series A351143. Similarly for the other two choices of p and q established in Theorem~\ref{thm-m2-egf}.

Thus the e.g.f. of 2-Bell sequences is similar to that of the regular 1-Bell sequence - it is the composition of an exponential-like function with the exponential function. The remaining terms are there just to ensure integer coefficients. With the integer‑coefficient property now rigorously settled, let us shift to the combinatorial side: a triangular recurrence whose row sums reproduce precisely the Bessel‑Bell sequence.

\section{Bessel-Stirling numbers of the second kind}\label{sec-2-stirling}

\noindent The similarity of the Bessel-Bell numbers e.g.f. to the standard Bell numbers e.g.f. immediately invites the question if there is also a Bessel-Stirling equivalent to the triangular array of the Stirling numbers of the second kind, whose rows sum to the Bell numbers. Recall that the Stirling numbers appears as the coefficients of the Touchard polynomials, which themseleves are the coefficients of a bi-variate generalization of the Bell e.g.f. (Equations\eqref{eq-touchard-def} and \eqref{eq-touchard-egf}). Turns out a analogous structure arises from a bi-variate generalization of the Bessel-Bell e.g.f..

\begin{theorem}[Bessel-Bell polynomials and Bessel-Stirtling numbers]\label{thm-bb-polynomials}
The Bessel-Bell numbers are the row sums of a Stirling-like triangular array, whose row entries arise as the coefficients of x of Touchard-like polynomials that are the coefficients of $t^n$ in the expansion of the following bi-variate e.g.f.:
    \begin{equation*}
        \begin{aligned}
        G(t,x) &= x\big[K_0(x)+K_1(x)\big]\, I_0(x e^t) + x\big[I_1(x)-I_0(x)\big]\,K_0(xe^t)\\
                       &= \sum_{n=0}^\infty \frac{t^n}{n!}\mathscr{B}_n(x) \\
                       & = \sum_{n=0}^\infty\frac{t^n}{n!}\sum_{k=0}^{n}\mStirling{n}{k}_2 x^k, \\
        \end{aligned}
    \end{equation*}
    where $\mathscr{B_n}(1)$ gives us the n-th Bessel-Bell number (A007472). $\mStirling{n}{k}_2$ uses a similar notation as for the Binomial coefficients $\binom{n}{k}$ or the standard Stirling numbers $\Stirling{n}{k}$ popularized by \cite{graham_concrete_nodate}.
\end{theorem}

\begin{proof}
The claim that $\mathscr{B}_n(1)$ represents the n-th Bessel-Bell number follows simply from the fact that $G(t,1)$ reduces to the e.g.f. of the Bessel-Bell numbers (Theorem \ref{thm-m2-egf}). Therefore what we need to prove is that $\mathscr{B}(x)$ is an n-th degree polynomial in x. As we saw in the previous section, we can work with a much simpler version of the e.g.f., namely, $I_0(xe^t)$ and show that the remaining components are present only to cancel the carrier special functions. The standard series expansion of $I_0(xe^t)$ gives:
\[
\begin{aligned}
I_0(x e^t) = \sum_{k=0}^\infty \frac{(x/2)^{2k} e^{2kt}}{k!k!} = \sum_{k=0}^\infty \frac{(x/2)^{2k}}{k!k!} \sum_{n=0}^\infty \frac{(2k)^n t^n}{n!} = \sum_{n=0}^\infty \frac{t^n}{n!} \sum_{k=0}^\infty \frac{(x/2)^{2k}(2k)^n}{k!k!},
\end{aligned}
\]
As noted in Equation \eqref{eq-scaling-operator}, we can represent this series also by the action of the exponential scaling operator on $I_0(x)$, which in this case provides a more fruitful approach:
\begin{equation}
   I_0(x e^t)= e^{txD}I_0(x)= \sum_{n=0}^{\infty}\frac{t^n(xD)^n}{n!}I_0(x)
\end{equation}

Combining our knowledge of the derivative of the 0-th order modified Bessel function $I_0(x)$ with that of the 1-st order function $I_0(x)$ (\href{https://dlmf.nist.gov/10.29.E2}{10.29.2} \cite{NIST:DLMF}), we can express a recurrence relation for the operator $(xD)$ acting on modified Bessel functions of the first kind:
\begin{equation}
    \begin{aligned}
        (xD)I_0(x) &= x I_1(x) \\
        (xD)I_1(x) &= x I_0(x)-I_1(x)
    \end{aligned}
\end{equation}
This recurrence involves only x, $I_0(x)$ and $I_1(x)$ functions and by the chain rule, each application of the $(xD)$ operator adds a single degree x to the resulting polynomial. Then it is clear that we can express $(xD)^nI_0(x)$ as an $n$-th degree polynomial in x, where each degree is "carried" by a Bessel function: 
\begin{equation}
    \begin{aligned}
        I_0(xe^t) &= \sum_{n=0}^\infty\frac{t^n}{n!}V_n(x) \\
        V_n(x) &= \sum_{k=0}^{n} \mStirling{n}{k}_2 x^k\, I_{k \bmod 2}(x)
    \end{aligned}
\end{equation}
Here are the first few $V$ polynomials:
\[
\begin{aligned}
V_0(x) &= I_0(x) \\
V_1(x) &= x\,I_1(x) \\
V_2(x) &= x^2\,I_0(x) \\
V_3(x) &= 2x^2\,I_0(x) + x^3\,I_1(x) \\
V_4(x) &= 4x^2\,I_0(x) + 4x^3\,I_1(x) + x^4\,I_0(x) \\
V_5(x) &= 8x^2\,I_0(x) + 12x^3\,I_1(x) + 8x^4\,I_0(x) + x^5\,I_1(x) \\
\end{aligned}
\]
Notice two things. First, these polynomials are generated in very much the same way as Touchard polynomials, but by applying the exponential scaling operator to $I_0(x)$ instead of to $e^x$. Second, each mononomial $x^k$ is accompanied by a modified Bessel carrier function - $I_0$ "carries" the even degrees of x and $I_1$ "carries" the odd degrees of x. The proper analogue to the Touchard polynomials stated in Theorem \ref{thm-bb-polynomials} would not have these carrier functions. Thankfully, the solution is the same as that for the e.g.f. of the 2-Bell numbers derived in Section \ref{sec-m2} - to use the general Wronskian property of Bessel functions \eqref{eq-bessel-wronskian-general} to construct a bi-variate e.g.f. that leads to the cancellation of all carrier functions. Some straightforward but tedious algebra, mirroring the steps in Section~\ref{2-bell-integer-coefs}, confirms that the chosen form of $G(t,x)$ precisely cancels the carrier functions. I omit the details.
\end{proof}

The numbers $\mStirling{n}{k}_2$ form a triangular array whose first few rows are listed in Table \ref{tab-bessel-stirling2}
\begin{table}[]
    \centering
        \begin{tabular}{|l|*{10}{c}|c|}
        \hline
              & k=0 & k=1 & k=2 & k=3 & k=4 & k=5 & k=6 & k=7 & k=8 & k=9 & $\sum_{k=0}^{n}$ \\
        \hline
        n=0 & 1   &     &     &     &     &     &     &     &     &     & 1                \\
        n=1 & 0   & 1   &     &     &     &     &     &     &     &     & 1                \\
        n=2 & 0   & 0   & 1   &     &     &     &     &     &     &     & 1                \\
        n=3 & 0   & 0   & 2   & 1   &     &     &     &     &     &     & 3                \\
        n=4 & 0   & 0   & 4   & 4   & 1   &     &     &     &     &     & 9                \\
        n=5 & 0   & 0   & 8   & 12  & 8   & 1   &     &     &     &     & 29               \\
        n=6 & 0   & 0   & 16  & 32  & 44  & 12  & 1   &     &     &     & 105              \\
        n=7 & 0   & 0   & 32  & 80  & 208 & 92  & 18  & 1   &     &     & 431              \\
        n=8 & 0   & 0   & 64  & 192 & 912 & 576 & 200 & 24  & 1   &     & 1969             \\
        n=9 & 0   & 0   & 128 & 448 & 3840& 3216& 1776& 344 & 32  & 1   & 9785             \\
        \hline
        \end{tabular}
        \bigskip
    \caption{The Bessel-Stirling numbers of the second kind, $\mStirling{n}{k}_2$. The values in the upper triangle are all 0.}
    \label{tab-bessel-stirling2}
\end{table}

\begin{table}[]
    \centering
    \begin{tabular}{|l|*{10}{c}|c|}
        \hline
          & k=0 & k=1 & k=2   & k=3   & k=4   & k=5   & k=6   & k=7   & k=8   & k=9   & $\sum_{k=0}^{n}$ \\
        \hline
        n=0 & 1   &     &       &       &       &       &       &       &       &       & 1                  \\
        n=1 & 0   & 1   &       &       &       &       &       &       &       &       & 1                  \\
        n=2 & 0   & 0   & 1     &       &       &       &       &       &       &       & 1                  \\
        n=3 & 0   & 0   & 2     & 1     &       &       &       &       &       &       & 3                  \\
        n=4 & 0   & 0   & 4     & 4     & 1     &       &       &       &       &       & 9                  \\
        n=5 & 0   & 0   & 16    & 20    & 8     & 1     &       &       &       &       & 45                 \\
        n=6 & 0   & 0   & 64    & 96    & 52    & 12    & 1     &       &       &       & 225                \\
        n=7 & 0   & 0   & 384   & 640   & 408   & 124   & 18    & 1     &       &       & 1575               \\
        n=8 & 0   & 0   & 2304  & 4224  & 3088  & 1152  & 232   & 24    & 1     &       & 11025              \\
        n=9 & 0   & 0   & 18432 & 36096 & 28928 & 12304 & 3008  & 424   & 32    & 1     & 99225              \\
        \hline
    \end{tabular}
    \bigskip
    \caption{Unsigned Bessel-Stirling numbers of the first kind, $\mStirlingone{n}{k}_2$.}
    \label{tab-bessel-stirling1}
\end{table}

\begin{table}[]
    \centering
    \begin{tabular}{|l|*{10}{c}|c|}
        \hline
          & k=0 & k=1 & k=2   & k=3   & k=4   & k=5   & k=6   & k=7   & k=8   & k=9   & $\sum_{k=0}^{n}$ \\
        \hline
        n=0 & 1   &     &       &       &       &       &       &       &       &       & 1                  \\
        n=1 & 0   & 1   &       &       &       &       &       &       &       &       & 1                  \\
        n=2 & 0   & 0   & 1     &       &       &       &       &       &       &       & 1                  \\
        n=3 & 0   & 0   & 4     & 1     &       &       &       &       &       &       & 5                  \\
        n=4 & 0   & 0   & 16    & 8     & 1     &       &       &       &       &       & 25                 \\
        n=5 & 0   & 0   & 96    & 64    & 16    & 1     &       &       &       &       & 177                \\
        n=6 & 0   & 0   & 576   & 480   & 192   & 24    & 1     &       &       &       & 1273               \\
        n=7 & 0   & 0   & 4608  & 4416  & 2400  & 432   & 36    & 1     &       &       & 11893              \\
        n=8 & 0   & 0   & 36864 & 39936 & 28416 & 6720  & 864   & 48    & 1     &       & 112849             \\
        n=9 & 0   & 0   & 368640& 436224& 380928& 109056& 18816 & 1536  & 64    & 1     & 1310305            \\
        \hline
    \end{tabular}
    \bigskip
    \caption{Bessel-Stirling numbers of the third kind.}
    \label{tab-bessel-stirling3}
\end{table}


The row sums of this array yield the Bessel-Bell numbers \href{https://oeis.org/A007472}{A007472}:
\[
B_n^{(2)} = \sum_{k=0}^{n} \mStirling{n}{k}_2 = (1, 1, 1, 3, 9, 29, 105, 431, 1969, 9785, 52145, 296155, 1787385, ...)
\]
Examining the coefficients of this triangle, which I will refer to as the Bessel-Stirling numbers of the second kind, leads to the following conjecture about their recurrence relation:
\begin{conjecture}[Recurrence relation of the Bessel-Stirling numbers]\label{conj-bs-recurrence}
\[
\mStirling{n+1}{k}_2 = 2\floor{k/2} \, \mStirling{n}{k}_2 + \mStirling{n}{k-1}_2, \,\,\,\,\, (n, k \ge 1)\\
\]    
\end{conjecture}
This recurrence relation is almost identical to that for the Stirling numbers of the second kind, except that the coefficient in front of $\mStirling{n}{k}$ has parity - it is always even, with k rounded down to the nearest even integer. Computational results with Mathematica confirm that the coefficients defined in Theorem \ref{thm-bb-polynomials} and those in Conjecture \ref{conj-bs-recurrence} produce the triangle up to reasonably large $n$. I recently submitted this triangle and is now available as OEIS entry \href{https://oeis.org/A383235}{A383235}.

The Bessel-Sitrling numbers are a novel generalization of the Stirling numbers and joins a broader family of two-term recurrences (\cite{mansour_general_2012},\cite{neuwirth_recursively_2001},\cite{barbero_g_bivariate_2014},\cite{spiveySolutionsGeneralCombinatorial2011}). I will return to the connection between this new array and previous generalizations of the Stirling numbers in Section \ref{sec-final-remarks}.

Due to the form of their recurrence, the Bessel-Stirling numbers admit many identities common to other such two-term recurrence triangular arrays (see \textbf{Appendix A} and \textbf{B}). The most important one is an explicit formula for calculating the Bessel-Stirling numbers directly (proof given in \textbf{Appendix A}):

\begin{theorem}[Explicit formula for the Bessel-Stirling numbers of the second kind]\label{thm-bs-formula}
The Bessel-Stirling numbers can be calculated explicitly for $n \ge 2, k \geq 2$ via  the following formula:
\[
    \mStirling{n}{k}_2 = \sum_{j=1}^{\floor{k/2}}(2j)^{n-k}\left (c_{k,j} +d_{k,j}(n-k+1) \right )
\]
where $d$ and $c$ are determined as follows:
\[
\begin{aligned}
    d_{k,r} &= \frac{(-r)^{k-1}}{r!r!(\floor{\frac{k}{2}}-r)!(\floor{\frac{k-1}{2}}-r)!} \\
    c_{k,r} &= \begin{cases}
                - d_{k,r} (r(2H_{r-1}-H_{\floor{\frac{k}{2}}-r}-H_{\floor{\frac{k-1}{2}}-r})-k+3) & \text{if} \, 1 \leq r < k/2\\
                \frac{r^{2r}}{r!r!} & \text{if}  \, r = k/2\\
                0 & \text{otherwise}
            \end{cases} 
\end{aligned}
\]
Alternatively, the following limit definition of the Bessel-Stirling numbers is valid for all $n\ge0, k\ge 0$:
$$
\begin{aligned}
    (-1)^{k-1} 2^{n-k} \sum_{j=0}^{\floor{k/2}} \lim_{t \to j} \left[t^{n-1}
        \frac
            {n-2-t\,\left(2\psi(t)-\psi\left(\floor{\frac{k}{2}}-t+1\right)-\psi\left(\floor{\frac{k-1}{2}}-t+1\right)\right)}
            {\Gamma(t+1) \Gamma(t+1) \Gamma(\floor{\frac{k}{2}}-t+1) \Gamma(\floor{\frac{k-1}{2}}-t+1)}\right]
\end{aligned}
$$
Where $H_n$ is the n-th harmonic number, $\psi(x)$ is the digamma function, and $\Gamma(x)$ is the gamma function. 
\end{theorem}

Now that we understand the analytic properties of Bessel-Stirling and Bessel-Bell numbers, and their basic recurrences, the natural next questions is - What do these numbers count?

\section{Combinatorial interpretation of the Bessel-Stirling numbers of the second kind and the Bessel-Bell numbers}\label{sec-combinatorial}
\noindent One possible interpretation of the Bessel-Stirling numbers $\mStirling{n}{k}_2$ is that they enumerate the number of ways to place $n$ labeled marbles in $k$ unlabeled compartments following a parity-constrained urn model.
\begin{theorem}[The two-compartment urn model]\label{thm-urn-model}
The Bessel-Stirling number of the second kind $\mStirling{n}{k}_2$ counts the number of ways to arrange $n$ labeled marbles in $k$ total compartments distributed across urns with two unlabeled compartments each, subject to the following constraints:
\begin{itemize}
    \renewcommand\labelitemi{--}
    \item You can place a marble in any compartment of any urn to begin
    \item You cannot place another marble in an urn's same compartment until the other compartment of that urn also contains at least one marble
    \item You can freely place marbles in any compartment of urns where both compartments have been used
    \item You cannot open a new urn until both compartments of all previously used urns contain at least one marble each
\end{itemize}
Then the Bessel-Bell numbers (\href{https://oeis.org/A007472}{A007472}) count the total number of ways to partition n labeled objects in such non-empty two-compartment urns.
\end{theorem}

Before diving into the proof, let us go through some examples to build intuition. Let $\frac{a}{b}\big|\frac{c}{d}$ represent 2 urns with vertical bars separating the urns and a horizontal bar separating the two compartments within an urn. The numbers that will replace the placeholder letters will represent the objects, labeled in the order in which they were placed in each compartment (one at a time).
For $\mStirling{3}{2}_2$ we have two ways to place the objects in two compartments of a single urn: $\frac{1\,3}{2}$ or $\frac{1}{2\,3}$.
For $\mStirling{3}{3}_2$ we have one way to place each object in its own compartment (2 urns, 3 compartments used): $\frac{1}{2}\big|\frac{3}{}$. All valid partitions up to 5 objects are presented in Table \ref{tab-bs-enumeration}.
\bgroup
\renewcommand{\arraystretch}{2}
\begin{table}[]
    \centering
        \begin{tabular}{|l|l|p{10.5cm}|}
        \hline
        T(n,k) & Count & Examples \\
        \hline
        T(1,1) & 1 & \texttt{$
            \frac{1}{}
        $}\\
        \hline
        T(2,2) & 1 & \texttt{$
            \frac{1}{2}
            \smallskip
        $}\\
        \hline
        T(3,2) & 2 & \texttt{$
            \frac{1}{2\,3},\
            \frac{1\,3}{2}
        $}\\
        T(3,3) & 1 & \texttt{$
            \frac{1}{2}\big|\frac{3}{}
            \smallskip
        $}\\
        \hline
        T(4,2) & 4 & {$
            \frac{1\,4}{2\,3},\ 
            \frac{1\,3\,4}{2},\
            \frac{1}{2\,3\,4},\ 
            \frac{1\,3}{2\,4}
        $} \\
        T(4,3) & 4 & \texttt{$
            \frac{1\,4}{2}\big|\frac{3}{},\
            \frac{1}{2\,4}\big|\frac{3}{},\
            \frac{1\,3}{2}\big|\frac{4}{},\
            \frac{1}{2\,3}\big|\frac{4}{}
        $}\\
        T(4,4) & 1 & \texttt{$
            \frac{1}{2}\big|\frac{3}{4}
            \smallskip
        $} \\
        \hline
        T(5,2) & 8 & \texttt{$
            \frac{1\,4\,5}{2\,3},\
            \frac{1\,4}{2\,3\,5},\
            \frac{1\,3\,4\,5}{2},\
            \frac{1\,3\,4}{2\,5},\
            \frac{1\,5}{2\,3\,4},\
            \frac{1}{2\,3\,4\,5},\
            \frac{1\,3\,5}{2\,4},\
            \frac{1\,3}{2\,4\,5}
        $}\\
        T(5,3) & 12 & \texttt{$
            \frac{1\,4\,5}{2}\big|\frac{3}{},\
            \frac{1\,4}{2\,5}\big|\frac{3}{\,},\
            \frac{1\,5}{2\,4}\big|\frac{3}{\,},\
            \frac{1}{2\,4\,5}\big|\frac{3}{\,},\
            \frac{1\,3\,5}{2}\big|\frac{4}{\,},\
            \frac{1\,3}{2\,5}\big|\frac{4}{\,},\
            \frac{1\,5}{2\,3}\big|\frac{4}{\,},\
            \frac{1}{2\,3\,5}\big|\frac{4}{\,},\
            \newline\newline
            \smallskip
            \frac{1\,3\,4}{2}\big|\frac{5}{\,},\
            \frac{1}{2\,3\,4}\big|\frac{5}{\,},\
            \frac{1\,3}{2\,4}\big|\frac{5}{\,},\
            \frac{1\,4}{2\,3}\big|\frac{5}{\,}
        $}\\
            
        T(5,4) & 8 & \texttt{$
            \frac{1\,3}{2} \big| \frac{4}{5},\
            \frac{1\,4}{2} \big| \frac{3}{5},\
            \frac{1\,5}{2} \big| \frac{3}{4},\
            \frac{1}{2\,3} \big| \frac{4}{5},\
            \frac{1}{2\,4} \big| \frac{3}{5},\
            \frac{1}{2\,5} \big| \frac{3}{4},\
            \frac{1}{2} \big| \frac{3\,5}{4},\
            \frac{1}{2} \big| \frac{3}{4\,5}
        $}\\
        T(5,5) & 1 & \texttt{$
            \frac{1}{2}\big|\frac{3}{4}\big|\frac{5}{\,}
            \smallskip
        $}\\
        \hline
        \end{tabular}
        \bigskip
    \caption{Valid Bessel-Stirling constrained partitions of $n$ objects into $k$ compartments of 2-compartment urns}
    \label{tab-bs-enumeration}
\end{table}
\egroup
\begin{proof}
We proceed by induction on $n$. The base cases are straightforward:
\begin{itemize}
\item $\mStirling{0}{0}_2 = 1$: There is exactly one way to arrange 0 marbles using 0 compartments.
\item $\mStirling{n}{0}_2 = 0$ for $n > 0$: It's impossible to arrange any positive number of marbles using 0 compartments.
\item $\mStirling{n}{k}_2 = 0$ for $k > n$: It's impossible to use more than $n$ compartments when placing $n$ marbles, since each compartment must contain at least one marble.
\item $\mStirling{n}{n}_2 = 1$ for $n > 0$: There is exactly one way to place $n$ marbles into $n$ different compartments.
\end{itemize}
For the inductive step, assume that for some $n \geq 1$, the number $\mStirling{n}{k}_2$ correctly counts the arrangements of $n$ labeled marbles into $k$ compartments following our constraints. We need to show that $\mStirling{n+1}{k}_2 = 2\lfloor k/2 \rfloor \mStirling{n}{k}_2 + \mStirling{n}{k-1}_2$ counts the arrangements for $n+1$ marbles in $k$ compartments.
Consider the placement options for the $(n+1)^{th}$ marble:
\begin{itemize}
    \item \textit{Place in a new compartment}: The $(n+1)$-th marble can be placed in a new (previously unused) compartment, making the total number of compartments used equal to $k$. This requires that the previous $n$ marbles were arranged in $k-1$ compartments, and there are $\mStirling{n}{k-1}_2$ such arrangements.
    \item \textit{Place in a previously used compartment}: The $(n+1)$-th marble can be placed in one of the compartments already containing marbles. Here, the key insight comes from the urn model's constraints.
\end{itemize}

We need to count how many valid placement options exist among the $k$ compartments already in use. This depends on the distribution of these compartments across urns:
\begin{itemize}
    \item \textit{When $k$ is even}, say $k = 2j$, the $k$ compartments must be distributed across exactly $j$ fully-filled urns (both compartments used in each). This is because our constraints prohibit opening a new urn until all previous urns have at least one marble in each compartment. With $j$ fully-filled urns, we have $2j = k$ valid placement options for the $(n+1)^{th}$ marble, since we can place it in any of the already-used compartments.
    \item \textit{When $k$ is odd}, say $k = 2j+1$, we must have $j$ fully-filled urns (accounting for $2j$ compartments) plus one urn with exactly one compartment used. According to the constraints, we cannot place another marble in this partially-filled urn's used compartment until its other compartment also contains a marble. Therefore, we have only $2j = 2\lfloor k/2 \rfloor$ valid placement options in previously used compartments.
\end{itemize}

In either case, the number of valid placement options in previously used compartments is $2\lfloor k/2 \rfloor$. Since there are $\mStirling{n}{k}_2$ ways to arrange the first $n$ marbles in $k$ compartments, we have $2\lfloor k/2 \rfloor \cdot \mStirling{n}{k}_2$ ways to place the $(n+1)^{th}$ marble in a previously used compartment.
Combining both cases, the total number of ways to arrange $n+1$ marbles in $k$ compartments is:
$\mStirling{n+1}{k}_2 = 2\lfloor k/2 \rfloor \cdot \mStirling{n}{k}_2 + \mStirling{n}{k-1}_2$
This matches exactly the recurrence relation for the Bessel-Stirling numbers, completing the proof.
\end{proof}

Note that this combinatorial interpretation easily generalizes to any $m$-Stirling numbers and $m$-Bell sequences. In the special case of $m=1$, each urn has only one compartment, so we are left with the standard combinatorial interpretation of counting partitions of labeled objects in unlabeled non-empty sets. This combinatorial interpretation may provide insight into why the sequence $a_n = \sum_{k=0}^n \mStirling{n}{k}_2$ shifts two places to the left after applying the binomial transform twice. The constraint requiring both compartments of an urn to be filled before opening a new urn creates a "pairing effect" that manifests mathematically as the shift-by-two property. This loose argument however requires further study. Finally, an easy way to enumerate the combinatorial object defined by the Bessel-stirling numbers of the second kind is via succession rules and a generating tree as introduced by West \cite{west_generating_1995}. See Figure \ref{fig:generating-tree} and Appendix C.

\begin{figure}[h]  % Use 'h' to place the figure here in the text
\centering
\begin{tikzpicture}[scale=0.5, level distance=1.1cm,
level 1/.style={sibling distance=8cm},
level 2/.style={sibling distance=7cm},
level 3/.style={sibling distance=6.8cm},
level 4/.style={sibling distance=2.2cm},
level 5/.style={sibling distance=0.6cm},
every node/.style={draw,circle,minimum size=0.25cm,inner sep=0pt,font=\scriptsize,}]
\node {0}
    child {node {1}
        child {node {2}
            child {node {2}
                child {node {2}
                    child {node {2}}
                    child {node {2}}
                    child {node {3}}
                }
                child {node {2}
                    child {node {2}}
                    child {node {2}}
                    child {node {3}}
                }
                child {node {3}
                    child {node {3}}
                    child {node {3}}
                    child {node {4}}
                }
            }
            child {node {2}
                child {node {2}
                    child {node {2}}
                    child {node {2}}
                    child {node {3}}
                }
                child {node {2}
                    child {node {2}}
                    child {node {2}}
                    child {node {3}}
                }
                child {node {3}
                    child {node {3}}
                    child {node {3}}
                    child {node {4}}
                }
            }
            child {node {3}
                child {node {3}
                    child {node {3}}
                    child {node {3}}
                    child {node {4}}
                }
                child {node {3}
                    child {node {3}}
                    child {node {3}}
                    child {node {4}}
                }
                child {node {4}
                    child {node {4}}
                    child {node {4}}
                    child {node {4}}
                    child {node {5}}
                }
            }
        }
    };
\end{tikzpicture}
\caption{First five levels of the generating tree for 2-Stirling numbers using the succession rule $\Omega$.}
\label{fig:generating-tree}
\end{figure}

\section{Bessel-Stirling numbers of the first kind and Bessel falling factorials}
\noindent The regular Stirling numbers come in a dual pair. In addition to the Stirling numbers of the second kind, which we explored at length, there exist a partner triangular array that bears the name Stirling numbers of the first kind. The two arrays are orthogonal to each other, or inverses of one another:
\[
\sum_{j=k}^n s(n,j)S(j,k)=\sum_{j=k}^n(-1)^{n-j} \Stirlingone{n}{j}\Stirling{j}{k} = \delta_{n,k}
\]

Where $s(n,k) = (-1)^{n-k
}\Stirlingone{n}{k}$ represents the \textit{signed} Stirling numbers of the first kind, $\Stirlingone{n}{k}$ represents the \textit{unsigned} Stirling numbers of the first kind, and $\delta_{n,k}$ is Konecker delta. When represented as lower-triangular matrices (as in Table \ref{tab-bessel-stirling2}), with $S$ representing the matrix of Stirling numbers of the second kind and $s$ matrix representing the Stirling numbers of the first kind, we have $s^TS = Ss^T= I$. The recurrence relation for the unsigned Stirling numbers of the first kind have a factor of $n$ rather than $k$:
\[
\Stirlingone{n+1}{k} = n\Stirlingone{n}{k} + \Stirlingone{n}{k-1}
\]

Similarly we find that the Bessel-Stirling numbers also have a first kind, defined by the following recurrence.
\begin{definition}[Unsigned Bessel-Stirling numbers of the first kind]
    \[
    \mStirlingone{n+1}{k}_2 = 2\floor{n/2}\mStirlingone{n}{k}_2 + \mStirlingone{n}{k-1}_2
    \]
\end{definition}

\begin{proposition}
    \[
        \sum_{j=k}^n (-1)^{n-j}\mStirlingone{n}{j}_2\mStirling{j}{k}_2=\delta_{k,n}
    \]
\end{proposition}

The first few rows of the unsigned Bessel-Stirling numbers of the first kind are shown in Table \ref{tab-bessel-stirling1}. Amazingly, although this triangular array is also not present in OEIS, its row sums yield a sequence with a rich combinatorial history: A000246, the number of permutations in the symmetric group $S_n$ that have odd order. A000246 is known to be the row sums of a couple of other triangles, neither of which is the one formed by the 2-Stirling numbers of the first kind.

This recurrence closely mirrors that of the standard Stirling numbers of the first kind, but with the factor $n$ replaced by $2\floor{n/2}$, reflecting the "even-step" structure inherent in the Bessel framework. Just as the ordinary Stirling numbers of the first kind arise in the context of change of basis from ordinary powers to falling powers, these generalized coefficients appear as the change-of-basis matrix between ordinary powers and a novel class of falling factorials.

\subsection{Bessel falling factorials}

\begin{figure}[h]  % Use 'h' to place the figure here in the text
\centering
\[
\begin{tikzcd}[column sep=9em, row sep=4.5em]
    x^{\overline{n}} 
        \arrow[rd, bend left=45, "(-1)^{n-k}\Stirling{n}{k}" description]
        \arrow[dd, bend left=15, "{\Lah{n}{k}}" description]
    & & 
    x^{\ooline{n}}
        \arrow[ld, bend left=-45, swap, "(-1)^{n-k}\mStirling{n}{k}" description]
        \arrow[dd, bend left=-15, swap, "{\mLah{n}{k}}" description]
    \\
    & 
    x^{n} 
        \arrow[ul, bend left=15, "\Stirlingone{n}{k}" description]
        \arrow[ur, bend left=-15, swap, "\mStirlingone{n}{k}" description]
        \arrow[dl, bend left=45, "(-1)^{n-k}\Stirlingone{n}{k}" description]
        \arrow[dr, bend left=-45, swap, "(-1)^{n-k}\mStirlingone{n}{k}" description]
    &\\
    x^{\underline{n}}
        \arrow[uu, bend left=45, "{(-1)^{n-k}\Lah{n}{k}}" description]
        \arrow[ur, bend left=15, "\Stirling{n}{k}" description]
    & & 
    x^{\uuline{n}}
        \arrow[uu, bend left=-45, swap, "{(-1)^{n-k}\mLah{n}{k}}" description]
        \arrow[ul, bend left=-15, swap, "\mStirling{n}{k}" description]
\end{tikzcd}
\]
\caption{Stirling numbers and Bessel-Stirling numbers are transformations between polynomial bases. The regular Stirling (and Lah) numbers convert between ordinary powers (middle), falling powers (bottom left), and rising powers (top left). The Bessel-Stirling numbers convert between ordinary powers, Bessel falling powers (bottom right), and Bessel rising powers (top right).}
\label{fig:commutative-diagram}
\end{figure}

The signed Stirling numbers of the first kind are well known to be the coefficients in the expansion of the falling factorial:
\[
x^{\underline{n}} =\prod_{k=0}^{n-1}(x - k)= x(x-1)(x-2)\cdots(x-n+1) = \sum_{k=0}^n (-1)^{n-k} \Stirlingone{n}{k} x^k.
\]

In the Bessel case, we discover a strikingly analogous structure. 
\begin{definition}
    Define the Bessel falling and rising factorials (or even-step falling factorials) as:
\[
\begin{aligned}
    x^{\uuline{n}} &= \prod_{k=0}^{n-1}(x-2\floor{k/2})= x^2(x-2)^2\cdots(x - 2\floor{(n-1)/2}) \\
    x^{\ooline{n}} &= \prod_{k=0}^{n-1}(x+2\floor{k/2})= x^2(x+2)^2\cdots(x + 2\floor{(n-1)/2}) 
\end{aligned}
\]  
\end{definition}
\begin{theorem}
The (signed) Bessel-Stirling numbers of the first kind appear as the coefficients in the expansion of the Bessel falling\ rising factorials:
\[
\begin{aligned}
    x^{\uuline{n}} &= \sum_{k=0}^n (-1)^{n-k} \mStirlingone{n}{k}_2 x^k\\
    x^{\ooline{n}} &= \sum_{k=0}^n \mStirlingone{n}{k}_2 x^k.
\end{aligned}
\]
\end{theorem}
Here is the expansion for the first 5 Bessel rising factorials (note that not all degrees on the left-hand side are square):
\[
\begin{aligned}
     x &= x &\\
     x^2 &= x^2 &\\
     x^2(x+2) &= 2x^2+x^3 &\\
     x^2(x+2)^2 &= 4x^2+4x^3+x^4 &\\
     x^2(x+2)^2(x+4) &= 16x^2+20x^3+8x^4+x^5 &\\
     x^2(x+2)^2(x+4)^2 &= 64x^2+96x^3+52x^4+12x^5 +x^6& \\
\end{aligned}
\]
Just like the regular falling powers form an alternative polynomial basis, which can be converted to and from using the Stirling numbers, the same is true for these Bessel falling powers. You can recover a monomial from them using the Bessel-Stirling numbers of the second kind:

\[
x^n =\sum_{k=0}^n \mStirling{n}{k} x^{\uuline{n}}
\]

Thus, the two Bessel-Stirling triangles again form a pair of lower-triangular inverse matrices—one converting monomials to generalized factorials, the other reversing the transformation. In addition, there exists an analogue to the Lah numbers, the Bessel-Bell numbers of the third kind, which converts between the two generalized falling/rising powers (see Figure \ref{fig:commutative-diagram}). Just like the Lah numbers, these final triangular array is simply the matrix product of the two Bessel-Bell number triangles. This fully mirrors the classical Stirling duality and confirms that the Bessel numbers of both kinds constitute a proper combinatorial and algebraic generalization of the Bell–Stirling–Touchard framework.

TODO: Combinatorial interpretation of the first kind numbers

\section{The general case}\label{sec-general-case}
Now that we have explored the case $m=2$ at depth, we can give the theory for the general case. 

\begin{proposition}
    The exponential generating functions for m-Bell numbers satisfy the m-th order ODE:
    \[
    \sum_{k=0}^m\Stirling{m}{k}t^kf^{(m)}-t^m
    \]
\end{proposition}

As for the case $m=2$, set $t=e^x$, $t=\frac{dt}{dx}$ and $f(t)=\mathcal{A}(x)$. Let $D_x$ be the derivative operator with respect to $x$ and $D_t$ be the derivative operator with respect to $t$. Then we have:
\[
    \begin{aligned}
        t&=D_xt \\
        D_xA(x)&=D_xf(x)=D_xt\,D_tf(t) = tD_tf(t)\\
        D_x^m &= (tD_t)^m\\
        D_x^{m}\mathcal{A}(x) - e^{mx} \mathcal{A}(x) &= (tD_t)^mf-t^mf=0 \\
    \end{aligned}
\]
\begin{lemma}
    \[
    (xD_x)^n = \sum_{k=0}^{n}\Stirling{n}{k}x^kD_x^k
    \]
\end{lemma}


To-be-done:

- [ ] explicit solution to the m-th order ODE in terms of \cite{tauberGeneralizationsExponentialFunction1960}.

- [ ] tables of m-Bell numbers

- [ ] Same combinatorial story, but with m compartments per urn

\begin{conjecture} m-Bell numbers are the row sums of m-Stirling numbers of the second kind triangular arrays.
\[
B_{n}^{(m)} = \sum_{k=0}^n \mStirling{n}{k}_m
\]
where $\mStirling{n}{k}_m$ is a number satisfying the following two-term recurrence relation:
\[
\mStirling{n+1}{k}_m = m\floor{k/m} \, \genfrac{\lfloor}{\rfloor}{0pt}{}{n}{k}_m + \genfrac{\lfloor}{\rfloor}{0pt}{}{n}{k-1}_m, \,\,\,\,\,
\]
for $n, k \ge 0$ and initial conditions $\genfrac{\lfloor}{\rfloor}{0pt}{}{n}{k}_m = \delta_{n,k}$ for $k\ge n$.
\end{conjecture}

The succession rule presented in Section \ref{sec-succession-rule} can also be generalized to $m$-Stirling numbers by adapting the pattern to reflect the $m$-compartment structure:
\begin{definition}[Succession Rule for $m$-Stirling Numbers]
\begin{align}
\Omega_m: \begin{cases}
(0) \
(mk+j) \to (mk+j)^{mk} , (mk+j+1) \quad \text{for } 0 \leq j < m
\end{cases}
\end{align}
\end{definition}
This representation further reinforces the structural unity of the $m$-Stirling family of sequences and provides an efficient way to generate and analyze these numbers through tree-based methods.

\section{Moments of the Conway-Maxwell-Poisson distribution}\label{sec-cmp}

Perhaps the most intriguing result from the current work is the connection between the $m$-Bell numbers and the moments of an existing generalization of the Poisson distribution - the Conway-Maxwell-Poisson distribution...

\section{Final remarks}\label{sec-final-remarks}
- What is the connection between the triangle of the Bessel-Stirling numbers of the first kind, and sequence A000246, which form its row sums?

\printbibliography

\pagebreak

\section*{Appendix A: Bessel-Stirling numbers formula (Proof of Theorem \ref{thm-bs-formula})}
Let $A_k$ be the ordinary generating function for the k-th column over $x^n$:
$$
A_k(x) = \sum_{n} \mStirling{n}{k}_2 x^n
$$
We first multiply each side of the recurrence in \ref{conj-bs-recurrence} by $x^n$ and sum over n:
$$
\begin{aligned}
\sum_{n \geq 1} \mStirling{n}{k}_2x^n &= 2\floor{k/2}\sum_{n \geq 1}\mStirling{n-1}{k}_2x^n + \sum_{n \geq 1} \mStirling{n-1}{k-1}_2 x^n \\
A_k -\mStirling{0}{k}_2&=2\floor{k/2}x A_k+x A_{k-1}, \quad k \geq 1
\end{aligned}
$$
Since $\mStirling{0}{k}_2 = 0$ when $k \neq 0$, then this simplifies to the following recurrence for the column generating function:
$$
A_k = \frac{x}{(1-2\floor{k/2}x)} A_{k-1}
$$
We know from the initial conditions that $A_0(x) = 1$, hence:
$$
\begin{aligned}
    A_0 &= 1 \\
    A_1 &= x \\
    A_2 &= \frac{x^2}{1-2x} \\
    A_3 &= \frac{x^3}{(1-2x)^2} \\
    A_4 &= \frac{x^4}{(1-2x)^2 (1-4x)} \\
    A_5 &= \frac{x^5}{(1-2x)^2 (1-4x)^2}, \\
\end{aligned}
$$
and so on. More generally, the ordinary generating function for the k-th column is:
\begin{equation}\label{column-ogf}
A_k(x) = \frac{x^k}{\prod_{j=1}^k(1-2\floor{j/2}x)}
\end{equation}
In order to get an explicit formula for $\mStirling{n}{k}_2$, use the technique on p.19 of \cite{wilfGeneratingfunctionologyThirdEdition2005}:
$$
\mStirling{n}{k}_2 = [x^n]A(x),
$$
where $[x^n]$ is the coefficient extractor operator. Start with partial fraction decomposition of $A_k(x)$. First, denote:
$$
A_k(x) = x^kP_k(x),
$$
where
$$
P_k(x) = \prod_{1 \leq j \leq k}\frac{1}{1-2\floor{j/2}x}.
$$
Hence
\begin{equation}\label{znk-coef}
\mStirling{n}{k}_2 = [x^n]x^kP_k(x) = [x^{n-k}]P_k(x)
\end{equation}
We can express $P_k(x)$ as a sum of partial fractions with unknown coefficients (note the slight difference in ranges for odd and even k):

\begin{equation}\label{pkx}
    \begin{aligned}
    P_{2m}(x) &= \sum_{1 \leq j \leq m} \frac{c_{2m,j}}{1-2jx} + \sum_{1 \leq j \leq m-1} \frac{d_{2m,j}}{(1-2jx)^2} \\
    P_{2m+1}(x) &= \sum_{1 \leq j \leq m} \frac{c_{2m+1,j}}{1-2jx} + \sum_{1 \leq j \leq m} \frac{d_{2m+1,j}}{(1-2jx)^2}
    \end{aligned}
\end{equation}

It is easy to calculate the $d$ coefficients with the standard Heaviside cover-up method:
$$
\begin{aligned}
    d_{k,r} &= (1-2rx)^2P_k(x)|_{x=\frac{1}{2r}} \\
    &= (1-2rx)^2 \prod_{1 \leq j \leq k} \frac{1}{1-2 \floor{j/2}x} \bigg|_{x=\frac{1}{2r}} \\
    &= \prod_{1 \leq j < 2r} \frac{1}{1-\floor{j/2}/r} \prod_{2r < j \leq k} \frac{1}{1- \floor{j/2}/r} \\
    &= \prod_{1 \leq j < 2r} \frac{r}{r-\floor{j/2}} \prod_{2r < j \leq k} \frac{r}{r- \floor{j/2}}, \\
\end{aligned}
$$
which simplifies to:
\begin{equation}\label{d-k-r}
        d_{k,r}=\frac{(-r)^{k-1}}{r!r!(\floor{\frac{k}{2}}-r)!(\floor{\frac{k-1}{2}}-r)!}.
\end{equation}
$d_{k,r}$ varies only slightly for even and odd k:
$$
\begin{aligned}
    d_{2m,r} &= \frac{-r^{2m-1}}{r!r!(m-r)!(m-1-r)!} \\
    d_{2m+1,r} &= \frac{r^{2m}}{r!r!(m-r)!(m-r)!} \\
\end{aligned}
$$
For the $c_{k,j}$ coefficients, we have two main cases to consider. When $k=2m$ is even and $j=m$, $P_k(x)$ has a simple pole at $1-2mx$, so we get $c_{2m,m}$ easily by the same method:
\begin{equation}\label{c-2m-m}
    c_{2m,m} = (1-2mx)P_{2m}(x)|_{x=\frac{1}{2m}} = \frac{m^{2m}}{m!m!}
\end{equation}
For the remaining $c$ coefficients  we have a pole of $P_k(x)$ of multiplicity 2 at $1-2jx$ so we need to use the derivative method. Specifically:
$$
-2r\,c_{k,r} = \frac{\mathrm{d}}{\mathrm{d}x} (1-2rx)^2P_k(x)\bigg|_{x=\frac{1}{2r}}
$$
Let 
$$
\begin{aligned}
Q_r &= (1-2rx)^2 P_k(x) = \prod_{j} \frac{1}{(1-2jx)^{e_j}},\\
e_j &= 
    \begin{cases} 
    2,& \mathrm{if}\quad 1 \leq 2j < k \land  j \neq r \\ 
    1, & \mathrm{if} \quad 2j = k \land j \neq r \\
    0, & \mathrm{otherwise}
    \end{cases}
\end{aligned}
$$
Then $c_{k,r} = -\frac{1}{2r}Q'_r\left (\frac{1}{2r} \right)$ and we can use the logarithmic derivative to simplify the calculation:

$$ 
\begin{aligned}
    \frac{Q'_{r}}{Q_r} &= \frac{d}{dx}log(Q_r) = \frac{d}{dx}\left[ -\sum_{j} e_j\log(1-2jx) \right]=  \sum_{j} 2e_j\frac{j}{1-2jx}  \\
    c_{k,r} &= - \frac{1}{2r}Q'_r\left (\frac{1}{2r} \right) = -\log(Q_r(x))'Q_r(x)|_{x=1/{2r}} \\
    &= -\frac{1}{2r} Q_r\left(\frac{1}{2r}\right) 2\sum_{j} e_j\frac{j}{1-j/r} \\ 
    &= -\frac{1}{2r} 2 d_{k,r}\sum_{j}e_j\frac{rj }{r-j} \\
    &= - \frac{1}{2r}2r d_{k,r}\sum_{j}e_j \frac{j}{r-j} \\
    &= - d_{k,r}\sum_{j}e_j \frac{j}{r-j} \\
\end{aligned}
$$
Let $s_k=\sum_{j}e_j\frac{j}{r-j}$ and simplify $s$. First, consider the case $k=2m+1$. 
$$
\begin{aligned}
    s_{2m+1} &=  2\sum_{\substack{1 \leq j \leq m \\ j \neq r}}\frac{j}{r-j} \\ 
    &= 2\left[\frac{1}{r-1}+\frac{2}{r-2} + \cdots  + \frac{r-1}{1}-\frac{r+1}{1}-\frac{r+2}{2}-\cdots-\frac{m}{m-r}\right]\\
    &= 2\sum_{1 \leq j \leq r-1}\frac{r-j}{j} -2\sum_{1 \leq j \leq m-r}\frac{j+r}{j} \\
    &= 2r\left(\sum_{1 \leq j \leq r-1}\frac{1}{j}\right) - 2(r-1) - 2r\left(\sum_{1 \leq j \leq m-r}\frac{1}{j}\right)-2(m-r)\\
    &= 2r(H_{r-1}-H_{m-r})-2m+2 \\
\end{aligned}
$$
where $H_n$ is the $n^{th}$ harmonic number. We proceed similarly for $k=2m$ with some care about $e_{m}=1$:
$$
\begin{aligned}
    s_{2m} &= \sum_{j}e_j \frac{j}{r-j} = \frac{m}{r-m} + 2\sum_{\substack{1 \leq j \leq m-1 \\ j \neq r}}\frac{j}{r-j}\\ 
    &= \frac{m}{r-m} + 2 \sum_{1 \leq j \leq r-1} \frac{j}{r-j} - 2\sum_{r+1 \leq j \leq m-1}\frac{j}{j-r} \\
    &= \frac{m}{r-m} + 2\sum_{1 \leq j \leq r-1}\frac{r-j}{j} -2\sum_{1 \leq j \leq m-r-1}\frac{j+r}{j} \\
    &= \frac{m}{r-m} + 2rH_{r-1}-2(r-1)-2rH_{m-r-1}-2(m-r-1) \\
    &=  2rH_{r-1} - 2rH_{m-r-1}-2(m-2)-\frac{m}{m-r}  \\
    &= 2rH_{r-1} - 2rH_{m-r-1} - 2(m-2) - \frac{r}{m-r} -1 \\
    &= r\left(2H_{r-1} - H_{m-r-1}-H_{m-r}\right) - 2m + 3
\end{aligned}
$$
Generally when $r \neq m$:
\begin{equation}\label{s-k}
    s_k = r(2H_{r-1}-H_{\floor{k/2}-r}-H_{\floor{(k-1)/2}-r})-k+3
\end{equation}
Combining everything we have:
\begin{equation}\label{d-s-c}
    \begin{aligned}
        d_{k,r} &= \frac{(-r)^{k-1}}{r!r!(\floor{k/2}-r)!(\floor{(k-1)/2}-r)!} \\
        s_{k,r} &= r(2H_{r-1}-H_{\floor{k/2}-r}-H_{\floor{(k-1)/2}-r})-k+3 \\
        c_{k,r} &= 
        \begin{cases}
            - d_{k,r} s_{k,r} & \text{if}\, 1 \leq r < k/2\\
            \frac{r^{2r}}{r!r!} & \text{if}\, r = k/2\\
            0 & \text{otherwise}
        \end{cases} 
    \end{aligned}
\end{equation}
Next, let us find an explicit formula for $\mStirling{n}{k}_2$. From \ref{znk-coef} and \ref{pkx}:
\[
\begin{aligned}
    \mStirling{n}{k}_2 &=[x^{n-k}]P_k(x) \\
    &= [x^{n-k}]\left[ \sum_{1 \leq j \leq \floor{k/2}} \frac{c_{k,j}}{1-2jx} + \sum_{1 \leq j \leq \floor{(k-1)/2}} \frac{d_{k,j}}{(1-2jx)^2} \right]\\
     &= \sum_{1 \leq j \leq \floor{k/2}}c_{k,j}[x^{n-k}]  \frac{1}{1-2jx} + \sum_{1 \leq j \leq \floor{(k-1)/2}} d_{k,j}[x^{n-k}]\frac{1}{(1-2jx)^2} \\ 
\end{aligned}
\]
We have the standard geometric series expansions:
\[
\begin{aligned}
     & [x^{n-k}]\frac{1}{1-2jx} = [x^{n-k}]\sum_{i\geq0}(2j)^ix^i = (2j)^{n-k}\\ 
     & [x^{n-k}]\frac{1}{(1-2jx)^2} = [x^{n-k}]\sum_{i\geq0}(i+1)(2j)^ix^i = (n-k+1)(2j)^{n-k}.\\ 
\end{aligned}
\]
Which results in the following final set of formulas valid for $n \geq 2, k \geq 2$:
\begin{equation}\label{z-d-c}
    \begin{aligned}
        \mStirling{n}{k}_2 &= \sum_{1 \leq j \leq \floor{\frac{k}{2}}}(2j)^{n-k}\left (c_{k,j} +d_{k,j}(n-k+1) \right )\\
            d_{k,r} &= \frac{(-r)^{k-1}}{r!r!(\floor{\frac{k}{2}}-r)!(\floor{\frac{k-1}{2}}-r)!} \\
            c_{k,r} &= \begin{cases}
                - d_{k,r} (r(2H_{r-1}-H_{\floor{\frac{k}{2}}-r}-H_{\floor{\frac{k-1}{2}}-r})-k+3) & \text{if} \, 1 \leq r < k/2\\
                \frac{r^{2r}}{r!r!} & \text{if}  \, r = k/2\\
                0 & \text{otherwise}
            \end{cases} 
    \end{aligned}
\end{equation}

This piecewise construction is necessary because the Harmonic numbers have a singularity at negative integer indices, which occurs when $r \geq k/2$ or $r\leq 0$. This problem can be avoided by replacing the Harmonic numbers with the identity $H(n) = \psi(n+1)+\gamma$ where $\gamma$ is the Euler-Mascheroni constant and $\psi(x)$ is the Digamma function, and by replacing the factorials with the Gamma function. Since we have digamma functions in the numerator and gamma functions in the denominator of the resulting expression, the limit of their ratio as the argument approaches non-positive integer indices is well defined. Specifically, using the residues of the digamma and gamma functions around the negative integer poles it can be easily shown that
\[
\lim_{x \to -n}\frac{\psi(x)}{\Gamma(x)} = (-1)^{n+1}n!, \quad n \in\mathbb{N}^{+}.
\]

After incorporating the coefficient definitions into the expression for $\mStirling{n}{k}_2$ (\ref{z-d-c}) and performing simplifications using properties of these special functions, we arrive at the following expression. 
$$
\begin{aligned}
    (-1)^{k-1} 2^{n-k} \sum_{j=0}^{\floor{k/2}} \lim_{t \to j} \left[t^{n-1}
        \frac
            {n-2-t\,\left(2\psi(t)-\psi\left(\floor{\frac{k}{2}}-t+1\right)-\psi\left(\floor{\frac{k-1}{2}}-t+1\right)\right)}
            {\Gamma(t+1) \Gamma(t+1) \Gamma(\floor{\frac{k}{2}}-t+1) \Gamma(\floor{\frac{k-1}{2}}-t+1)}\right]
\end{aligned}
$$
This formula provides a single, unified expression for $\mStirling{n}{k}_2$ that is valid for all integers $n,k \geq 0$. The inclusion of the $j=0$ term in the sum is essential, as the limit process correctly evaluates its contribution, ensuring validity for boundary cases like $\mStirling{0}{0}_2=\mStirling{1}{1}_2=1$, and ensuring $\mStirling{n}{k}_2=0$ for $k>n$.

Here are explicit formulas for columns 2-5 and related OEIS sequences:
\[
    \begin{aligned}
        Z(n,2) &= 2^{n-2}, & \quad n \ge 2\, (\text{rel. } \href{https://oeis.org/A000079}{A000079})\\
        Z(n,3) &= (n-2)2^{n-3}, & \quad n \ge 3\, (\text{rel. } \href{https://oeis.org/A001787}{A001787})\\
        Z(n,4) &= 2^{2n-6}-(n-1)\,2^{n-4}, & n\geq 4 \, (\text{rel. } \href{https://oeis.org/A100575}{A100575})\\
        Z(n,5) &= n\,2^{n-5}+(n-6) 2^{2n-8}, & n\geq 5 \, (\text{rel. } \href{https://oeis.org/A158681}{A158681})\\
    \end{aligned}
\]

(with the caveat that the sequences listed at the end of each line are offset or scaled). The first subdiagonal gives \href{https://oeis.org/A007590}{A007590}.

\pagebreak
\section*{Appendix B: Bessel-Stirling numbers identities}
\noindent In all cases below $b = 2\floor{k/2}$ is used to simplify the notation.
\begin{lemma}[Distant Hockey-stick-like identity]\label{lemma-dist-hockey}
\[
    \mStirling{n+m}{k}_2 = b^m \mStirling{n}{k}_2 +\sum_{j=0}^{m-1}b^{m-1-j}\mStirling{n+j}{k-1}_2
\]
\end{lemma}
\begin{proof}
The base case for $m=1$ comes from the basic recurrence formula (\ref{conj-bs-recurrence}):
\[
    \mStirling{n+1}{k}_2 = b \,\mStirling{n}{k}_2 + \mStirling{n}{k-1}_2
\]
Assume it holds for \textit{m} (induction hypothesis). Then from the basic recurrence we have:
\[
    \mStirling{n+m+1}{k}_2 = b \, \mStirling{n+m}{k}_2 +\mStirling{n+m}{k-1}_2
\]
From the induction hypothesis:
\[
    \begin{aligned}
        \mStirling{n+m+1}{k}_2 &= b \, \bigg(b^m \mStirling{n}{k}_2 +\sum_{j=0}^{m-1}b^{m-1-j}\mStirling{n+j}{k-1}_2\bigg) +\mStirling{n+m}{k-1}_2 \\
                   &= b^{m+1}\mStirling{n}{k}_2+\sum_{j=0}^mb^{m-j}\, \mStirling{n+j}{k-1}_2
    \end{aligned}
\]
which completes the proof.
\end{proof}
\begin{proposition}[Hockey-stick-like identity]
    \[
        \mStirling{n}{k}_2 = \sum_{j=k}^{n}b^{n-j} \mStirling{j-1}{k-1}_2
    \]
\end{proposition}
\begin{proof}
Set $n=0$ in Lemma \ref{lemma-dist-hockey}:
$$
    \mStirling{m}{k}_2 = b^m \mStirling{0}{k}_2 +\sum_{j=0}^{m-1}b^{m-1-j}\mStirling{j}{k-1}_2
$$
Due to the initial conditions $\mStirling{0}{0}_2=1$, $\mStirling{n}{0}_2 = \mStirling{0}{k}_2 =0$, the first term of the right-hand side is always 0. Then we can simplify it, rewriting the indices:
$$
\mStirling{n}{k}_2 = \sum_{j=0}^{n-1}b^{n-j-1}\mStirling{j}{k-1}_2
$$
and since $\mStirling{n}{k}_2 =0$ when $k > n$, we can start $j$ from $k$, and also shift it by 1 to match the Proposition's form:
\[
        \mStirling{n}{k}_2 = \sum_{j=k}^{n}b^{n-j} \mStirling{j-1}{k-1}_2
\]
\end{proof}

\pagebreak
\section*{Appendix C: Succession rules and a generating tree for Bessel-Stirling numbers}\label{sec-succession-rule}
\noindent The Bessel-Stirling numbers of the second kind can also be elegantly described using the formalism of succession rules and generating trees, as introduced by West \cite{west_generating_1995}. This representation provides yet another perspective on the structure of these numbers and their recurrence relation.
\begin{definition}[Succession Rule for 2-Stirling Numbers]
The Bessel-Stirling triangle of the second kind can be generated using the following succession rule:
\[
    \begin{aligned}
        \Omega: \begin{cases}
        (0) \\
        (2k) \to (2k)^{2k} (2k+1) \\
        (2k+1) \to (2k+1)^{2k}(2k+2)
        \end{cases}
    \end{aligned}
\]
where the notation $(a) \to (b)^c , (d)$ means that a node labeled $(a)$ produces $c$ children labeled $(b)$ and one child labeled $(d)$.
\end{definition}
Starting from the root node labeled $(0)$, we apply this rule iteratively to build a generating tree. The labels at level $n$ of this tree encode the 2-Stirling numbers $\mStirling{n}{k}_2$. Specifically, the number of nodes (the exponent) with label $(k)$ at level $n$ corresponds precisely to $\mStirling{n}{k}_2$.
For example, the first few levels of the generating tree produce the following multisets of labels:
\[
    \begin{aligned}
    \text{Level 0}: &{(0)} \\
    \text{Level 1}: &{(1)} \\
    \text{Level 2}: &{(2)} \\
    \text{Level 3}: &{(2)^2(3)} \\
    \text{Level 4}: &{(2)^4(3)^4(4)} \\
    \text{Level 5}: &{(2)^8(3)^{12}(4)^8(5)} \\
    \end{aligned}
\]
The total number of labels at level $n$ is precisely the $n$-th Bessel-Bell number (sequence A007472). This succession rule representation directly encodes the recurrence relation for 2-Stirling numbers. For even values $k = 2j$, a node labeled $(2j)$ produces $2j$ children with the same label plus one child with label $(2j+1)$, corresponding to the term $2j \cdot \mStirling{n}{2j}_2$ in the recurrence relation. Similarly, for odd values $k = 2j+1$, a node produces $2j$ children with the same label, corresponding to the term $2j \cdot \mStirling{n}{2j+1}_2$, plus one child with label $(2j+2)$.

This succession rule formalism provides a direct connection to the combinatorial interpretation: the label $(k)$ represents using $k$ compartments, and the number of children with the same label reflects the number of valid placements in already-used compartments, while the single child with a new label represents opening a new compartment.

\pagebreak
\section*{Appendix D: Comparison of properties of the Bell-Stirling-Touchard framework and the 2-Bell/2-Stirling generalization}

PUT A TABLE HERE TO HIGHLIGHT THE PARALLELS AND TO SERVE AS EASY REFERENCE

\end{document}